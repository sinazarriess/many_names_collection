\begin{table*}[t]
	\begin{tabular}{|l|l|llll|}
		\hline
		Resource & Purpose & \multicolumn{4}{c|}{Useful For}\\
				& 			& Basic-level Names & Specificity & Bla & Bli \\
		\hline \hline
		RefCOCO & XXX & 
	\end{tabular}
	\caption{\label{tab:summary_resources} Intended to provide an overview of resources, their purpose/use and shortcomings wrt object naming}
\end{table*}

\subsection{Computer Vision Resources}
\paragraph{Methods: Object Detectors, Image Classifiers}

\paragraph{ImageNet and ILSVRC}
\paragraph{MSCOCO} \cite{mscoco}

\subsection{V\ \&\ L Resources}
\paragraph{VisualGenome}
\paragraph{(ReferIt) RefCOCO and alike}
Based on MSCOCO; Other MSCOCO-based resources: Captions; Visual Genome in parts\\

We analyze the RefCOCO and RefCOCO+ datasets which contain referring expressions to objects in MSCOCO \cite{mscoco} images.
These data collections were performed via crowdsourcing with the ReferIt Game \cite{Kazemzadeh2014}  where two players were paired and a director needed to refer to a predetermined object to a matcher, who then selected it.
RefCOCO and RefCOCO+ contain 3 referring expressions on average per object, and overall 150K expressions for 50K objects. 
The two datasets have been collected for an (almost) identical set of objects, but in RefCOCO+, players were asked not to use location words (\textit{on the left}, etc.).
See \cite{Yu2016} for more details. 

\paragraph{Flickr30k Entities}

\subsection{\cs{Language Resources? (WordNet)}}


\subsection{Analysis of Data wrt Our Requirements(?)}

\subsubsection{Pre-processing}
We parse referring expressions and captions with the Stanford Dependency Parser.
We extract heads/object names as follows: TODO.

\subsubsection{Level of Specificity} 
Variability of reference level in existing data sets for language \& vision?
Are resources appropriate for defining reference level?
\paragraph{WordNet?}

We hypothesize that the distance of a name's synset to the root node (entity) relates to its specificity.
We estimate this distance as the minimal path length of all synsets of a word  to the root node.

Table \ref{tab:specnames} shows the estimated levels of specificity for object names in the RefCoco data set.
We observe distances to the root between 2 and 17, meaning that there is a much more fine-grained distinction of levels than the three-way classification.

Unfortunately, the levels of specificity predicted by WordNet do not seem to reflect linguistic intuitions, here are some problematic examples from Table \ref{tab:specnames}:

\begin{itemize}
\item elephant (10) is more specific than panda (14)? horse is less specific than elephant (10)?
\end{itemize}


\begin{table}
\centering
\setlength{\tabcolsep}{4pt}
\begin{tabular}{rrl}
\toprule
 specificity &  rel.freq. &                          top 5 names \\
\midrule
          -1 &   0.071697 &      NONE,brocolli,zeb,broc,girafe \\
           2 &   0.003898 &                         thing,things \\
           3 &   0.001182 &   object,group,set,substance,objects \\
           4 &   0.140633 &           man,person,piece,head,part \\
           5 &   0.100739 &       player,glass,baby,front,corner \\
           6 &   0.208590 &              woman,girl,kid,boy,bowl \\
           7 &   0.238708 &            guy,right,chair,lady,bear \\
           8 &   0.110613 &           horse,bus,cow,pizza,batter \\
           9 &   0.097390 &         shirt,car,bike,donut,catcher \\
          10 &   0.048368 &   elephant,couch,truck,vase,suitcase \\
          11 &   0.008828 &    motorcycle,clock,mom,dad,scissors \\
          12 &   0.002822 &  oven,airplane,suv,taxi,refrigerator \\
          13 &   0.005253 &   laptop,fridge,canoe,orioles,pigeon \\
          14 &   0.000414 &  panda,freezer,penguin,rooster,rhino \\
          15 &   0.030870 &   zebra,giraffe,zebras,giraffes,deer \\
          16 &   0.000083 &       bison,mooses,orang,elks,sambar \\
          17 &   0.000143 &           ox,cattle,gnu,mustang,orca \\
\bottomrule
\end{tabular}\caption{Levels of specificity for naming choices in RefCOCO: for each level, relative frequency and 5 most frequent names are shown}
\label{tab:specnames}
\end{table}

\paragraph{No. of images in ImageNet?}