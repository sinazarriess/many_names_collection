%\cite{graf2016animal}
%\cite{Rosch1978}
%\subsection{Intro to Problem}
The act of naming an object amounts to that of picking out a nominal to be employed to refer to it (e.g., ``the \textit{dog}'', ``the white \textit{dog} to the left'') and can be seen as a subtask of generating a referring expression. The lexical choice involved in naming is a non-trivial one \cite{brown1958shall}. Indeed, since an object is simultaneously a member of multiple categories (e.g., a young beagle is at once a dog, a beagle, an animal, a puppy etc.), all the various names that lexicalize these constitute a valid alternative: in fact, their denotation includes the target object. The study of object naming focuses on those factors that lead to the choice of a particular name in communicative situations. 
%These include properties of the object and candidate names, as well as contextual requirements.

Lexical alternatives in naming differ in their \textbf{level of specificity} (e.g., \textit{dog} is less specific than \textit{beagle})\cite{cruse1977pragmatics}: the categories they denote can indeed be organized in a hierarchical fashion according to class inclusion relations (e.g., a beagle is a dog, a dog is an animal etc.) \cite{murphy2004big}. Such a conceptual structure gives rise to a taxonomy, or ontology. %where locating an object at a certain category (e.g., beagle) allows to inductively access all its superordinate ones (e.g., dog, mammal, animal etc.).
It was observed that each type of object exhibits a preferred level of specificity which it is more naturally named at, called the \textbf{entry-level}. This typically corresponds to an intermediate level of specificity, i.e., \textbf{basic level} (e.g, \textit{bird}, \textit{car}) \cite{rosch1976basic}, as opposed to more generic (i.e., \textbf{super-level}; e.g., \textit{animal}, \textit{vehicle}) or specific categories (i.e., \textbf{sub-level}; e.g., \textit{sparrow}, \textit{convertible}). However, less prototypical members of basic-level categories tend to be instead identified with sub-level categories (e.g., a penguin is typically called a \textit{penguin} and not a \textit{bird}) \cite{jolicoeur1984pictures}. This out-of-context preference towards a certain taxonomic level is often referred to as \textbf{lexical availability}. 

Contextual factors also affect object naming. Scenarios where multiple objects are available induce a pressure for generating names which uniquely identify the target, thus excluding the competing alternatives (or \textit{distractors})\cite{olson1970language}. For example, in presence of more than one dog, the name \textit{dog} is ambiguous in terms of the specific one it is intended to refer to. In these cases, a sub-level category (e.g., \textit{rottweiler}, \textit{beagle}) is more informative: it provides more specific information, though it is possibly more costly to produce (it may be a non-default, infrequent or long alternative). It was shown that in use speakers tend to choose a name given a trade-off between its \textbf{cost}, on one side, and its contextual \textbf{informativeness}, on the other; for example, they may still opt for an ambiguous expression if less costly for them to produce it \cite{rohde2012communicating} \cite{graf2016animal}. 
%Note that reasoning about the informativeness of a name require categorization of all the potential referents in that context (e.g., to know that \textit{dog} is ambiguous require having recognized that there are multiple dogs in the scene). 
Other contextual factors affecting lexical choice include the \textbf{perceptual salience} of the object, such as its size or location \cite{clark1983common}.

%\begin{itemize}
%\item Naming (lexical choice of nominal reference) is not trivial: many names for an object \cite{brown1958shall}. If seen as the act of picking out an applicable level: each object simultaneously belong to multiple categories, which reflect different conceptualizations and may be more or less appropriate across contexts, 

% \cite{olson1970language}. Lexical alternatives: near-synonyms, hyponyms, hypernyms etc.\cite{edmonds2002near} 

%\item Typically, variation across levels of specificity \cite{cruse1977pragmatics} Concepts are organized in a hierarchical fashion according to class inclusion relations, giving rise to a taxonomy \cite{murphy2004big}. Note that if accessing the most specific category in such taxonomy we can inherit knowledge about all the other categories it is also part of, by backtracking the path up to the root in the ontology (inductive access to lexical alternatives\begin{center}
	
%\item  Out-of-context preference for each object towards a certain specificity level, called \textbf{entry-level} category. Typically, this correspond to a intermediate level of specificity, called \textbf{basic level} \cite{rosch1976basic}, but for less typical members of the class may be more specific \cite{jolicoeur1984pictures}. 

%\item Context: distractors, competing alternatives, pressure for unique reference. Overriding the out-of-context preference and may require the use of other levels of specificity e.g. \cite{murphy1989categorizing}. 
%In context, balancing a preference for low-cost, on one side, and informative, on the other, expressions: reduce cost while keeping interpretability (an ambiguous expression may be chosen if less costly (more frequent and short)) \cite{rohde2012communicating, graf2016animal}. In interaction: coherence and alignment. 
%\end{itemize}
%\subsection{Definition Sub-, basic-, super-level Categories}
%\cite{graf2016animal} investigate object naming with respect to reference level. They distinguish and manually annotate 3 levels: (i) sub-level (\textit{dalmatian}), (ii) basic-level (\textit{dog}), (iii) super-level (\textit{animal}).

%\section{Requirements: What Do We Need? \cs{[Subsection of \ref{sec:object_naming}?]}}
%For large-scale studies of object naming, we need to be able to automatically define the level of specificity of a name, given an ontology. 
%
%\paragraph{Requirements regarding the factors}
%%Given an image which depicts an object~$o^t$, the target object of the naming task, along with $k$~distractor objects~\mbox{$d^i, i\in{1,...,k}$} (the perceptual context of~$o_t$, understood as the real-world situation in which a unique name for~$o_t$ is to be produced).
%\cs{Note: I am aware of the inconsistency caused by "leaf category". Needs to be replaced later.}

\section{Requirements}
In order to study object naming as prompted by real-world images, not only do we need to collect names naturally generated by speakers, but also to quantify those factors whose interaction with naming we want to analyze. As we saw, phenomena that impacts on object naming are both perceptual and linguistic in nature: since we here focus on perception as vision, we distinguish between those that pertain language, vision or both.

\la{Note that I swapped the order of contextual informativeness and lexical availability. What does R1 refer to? the requirement of sub-level info?}
\begin{itemize}
	\item[---] \textbf{Visual-Linguistic factors}
	\begin{itemize}
		\item[R1] \textbf{Lexical availability}: \\
		In object naming, there is a preference for each object type towards a certain level of specificity. To detect this, we need to have access to a taxonomy, reflecting  hierarchical relations among the various categories, and map both names and target objects to their location in such a structure. For instance, if we want to check whether a convertible is more often referred to with a basic-level category, e.g., \textit{car}, or a more specific one, e.g., \textit{convertible}, we need to know that \textit{car} lexicalizes a category which is super-ordinate to that lexicalized by \textit{convertible}. Moreover, in the first place, we need to know which categories our target object belongs to (e.g., convertible, a car, a vehicle etc.) and hence which names could apply to it. Note that for due to class inclusion relations, it is sufficient to know the most specific category in the taxonomy, i.e., the sub-level, to deduce all the other others (e.g., convertibles $\subset$ cars $\subset$ vehicles ... ).
		
		For this purpose, one can use existing lexical resources as a taxonomy: for example, in WordNet \cite{fellbaum1998wordnet} words denoting the same category are grouped and linked to others by taxonomic relations. The repertory of categories, and hence candidate names, for a target object can be queried by taking the sub-level category, i.e., the \textsl{leaf category} in the taxonomy.
		
		\item[R2] \textbf{Contextual informativeness}: \\
		The same object may be named differently depending on the distractors it is combined with: a particular name may not be sufficiently specific to identify its referent if applicable to more than one object. In order to analyze this, we are required to know the repertory of categories that each of the potential referents is an instance of: for instance, we need to know that there are multiple cars, to judge whether \textit{car} is ambiguous. As before, we also need to map names to a taxonomy, in order to check which level of specificity is chosen. We then require similar resources as in R1, but extending information about the sub-level category to each object in the scene.
				
	\end{itemize}
	\item[---] \textbf{Linguistic factors}:
	\begin{itemize}
		\item[R3] \textbf{Cost}: \\
		The cost of generation of a nominal may impact on its likelihood to be chosen in naming. Previous work \cite{graf2016animal} operationalized such cost in terms of length and frequency. These can readily be coded as functions of respectively the number of characters and the relative frequency (estimated from a text corpus) of a name. \la{Actually, they also include typicality in the cost but I would leave it out as we kind of put that phenomenon more under the umbrella of lexical availability. And in general, I am not sure what to say about that, actually ... Do we want typicality ratings?} However, reasoning about the cost of lexical alternatives requires having access to such alternatives, hence names applicable to the object. As before, these can be obtained by mapping the object to the taxonomy and derive the set of categories it is a member of.
	\end{itemize}
	\item[---] \textbf{Visual factors}
	\begin{itemize}
		\item[R4] \textbf{Perceptual salience}: \\
		The prominence of an object in a scene may affect naming.	This can be estimated geometrically on the basis of the relative position and size of an object in the image.
	\end{itemize}
\end{itemize}
%
To summarize, to carry out an analyses of object naming in real-world images datasets we require:
\begin{itemize}
	\item Names naturally generated by speakers to refer to a target object in an image
	\item Sub-level category of each object in the image (including the target object)
	\item Set of categories mapped to their lexical realizations and organized according to taxonomic relations (e.g., WordNet)
	\item Coordinates of the image region showing the target object.
\end{itemize}
%\begin{itemize}
%	\item[---] Visual-Linguistic Factors
%	\begin{itemize}
%		\item[R1] Contextual informativeness\\
%		To reason about the contextual informativeness, the  repertory of categories of which an object can be an instance is required. \\
%		%What are the lexical alternatives of a target object which \textit{uniquely} refer to it?\\
%		%For each pair of target object and distractor, we need to know whether they are from the same most specific category. (I.e.,~are they of the same sub, basic or super-level?)\\
%		%Example pizza: is there another calzone? Is there another pizza? Is there another kind of food?)\\
%		$\Rightarrow$ The repertory of categories for an object can be queried from an existing taxonomy, such as WordNet \cite{fellbaum1998wordnet}, by its most specific name (the \textsl{leaf category} in the taxonomy). 
%		Hence, for \textit{each} object in the image, its leaf category is required.
%		\item[R2] The preferred name of an object\\
%		The entry-level/basic-level name.\\
%		$\Rightarrow$ ?
%	\end{itemize}
%	\item[---] Linguistic (lexical) Factors
%	\begin{itemize}
%		\item[R3] Lexical cost\\
%		$\Rightarrow$ Can be estimated, e.g.,~by the length of an expression. Needs to be estimated for all lexical alternatives, i.e.,~requires R1.
%	\end{itemize}
%	\item[---] Perceptual Factors
%	\begin{itemize}
%		\item[R4] Perceptual salience\\
%		This can be estimated geometrically on the basis of the relative position and size of an object in the image.\\
%		$\Rightarrow$ The coordinates (e.g.,~bounding box) of the image region which shows object~$o$. 
%	\end{itemize}
%\end{itemize}