%\cite{graf2016animal}
%\cite{Rosch1978}
\subsection{Intro to Problem}
The act of naming an object amounts to that of picking out a nominal to be employed to refer to that particular object (e.g., ``the \textit{dog}'', ``the white \textit{dog} to the left'') and can be seen as a subtask of generating a referring expression.  The lexical choice involved in naming is a non-trivial one \cite{brown1958shall}\cite{olson1970language}. Since an object is simultaneously a member of multiple categories (e.g., a young beagle is at once a dog, a beagle, an animal, a puppy etc.), all the various names that lexicalize these constitute a valid alternative. Indeed, for all such names their denotation includes the target object. However, in communicative situations, a particular name is selected for an object considering conceptual information about the object and contextual requirements.

To start with, lexical alternatives differ in their \textbf{level of specificity} (e.g., \textit{dog} is less specific than \textit{beagle})\cite{cruse1977pragmatics}. The categories and concepts they denote are organized in a hierarchical fashion according to class inclusion relations (e.g., a beagle is a dog, a dog is an animal etc.)  \cite{murphy2004big}. They can be seen as giving rise to a taxonomy, or ontology.
%(note that, in principle, knowing the most specific category of an object in such a structure gives us inductive access to all its other categories) 
It was shown that for each type of object, there is a preferred level of specificity which is employed in naming, called the \textbf{entry-level}. This typically corresponds to categories with an intermediate level of specificity, i.e., \textbf{basic level} (e.g, \textit{dog}, \textit{car}) \cite{rosch1976basic}; however, for less prototypical members of such categories a more specific level is usually chosen (e.g., a penguin is typically called a \textit{penguin} and not a \textit{bird}) \cite{jolicoeur1984pictures}. This factor is typically referred to as \textbf{lexical availability}.
Moreover, contextual factors also affect object naming. For example, scenarios where the target object is presented with other objects create a pressure for generating names which uniquely identify the target, thus excluding the competing alternatives (or \textit{distractors}). To do so, a speaker may be required to the use of more specific expressions than the entry-level, in spite of these being less default or generally more costly (e.g., long, infrequent). A series of study showed that speakers tend to balance the preferences for reducing the \textbf{cost} of a name, on one side, and \textbf{ambiguity}, on the other (for example, they may opt for an ambiguous expression if less costly) \cite{rohde2012communicating} \cite{graf2016animal}. Other contextual factors affecting lexical choice include the \textbf{perceptual salience} of the object, such as its size or prominence \cite{}

%\begin{itemize}
%\item Naming (lexical choice of nominal reference) is not trivial: many names for an object \cite{brown1958shall}. If seen as the act of picking out an applicable level: each object simultaneously belong to multiple categories, which reflect different conceptualizations and may be more or less appropriate across contexts, 

% \cite{olson1970language}. Lexical alternatives: near-synonyms, hyponyms, hypernyms etc.\cite{edmonds2002near} 

%\item Typically, variation across levels of specificity \cite{cruse1977pragmatics} Concepts are organized in a hierarchical fashion according to class inclusion relations, giving rise to a taxonomy \cite{murphy2004big}. Note that if accessing the most specific category in such taxonomy we can inherit knowledge about all the other categories it is also part of, by backtracking the path up to the root in the ontology (inductive access to lexical alternatives\begin{center}
	
%\item  Out-of-context preference for each object towards a certain specificity level, called \textbf{entry-level} category. Typically, this correspond to a intermediate level of specificity, called \textbf{basic level} \cite{rosch1976basic}, but for less typical members of the class may be more specific \cite{jolicoeur1984pictures}. 

%\item Context: distractors, competing alternatives, pressure for unique reference. Overriding the out-of-context preference and may require the use of other levels of specificity e.g. \cite{murphy1989categorizing}. 
%In context, balancing a preference for low-cost, on one side, and informative, on the other, expressions: reduce cost while keeping interpretability (an ambiguous expression may be chosen if less costly (more frequent and short)) \cite{rohde2012communicating, graf2016animal}. In interaction: coherence and alignment. 


%\end{itemize}
\subsection{Definition Sub-, basic-, super-level Categories}
\cite{graf2016animal} investigate object naming with respect to reference level. They distinguish and manually annotate 3 levels: (i) sub-level (\textit{dalmatian}), (ii) basic-level (\textit{dog}), (iii) super-level (\textit{animal}).

\section{Requirements: What Do We Need? \cs{[Subsection of \ref{sec:object_naming}?]}}
For large-scale studies of object naming, we need to be able to automatically define the level of specificity of a name, given an ontology. 