\section{Data Collection}
\label{sec:task}

%describe the YouNameIt task here

describe the task here

\subsection{Visual Genome data}

\gbt{to be refined -- taken from sivl submission as is}

\vgenome \cite{krishna2016visualgenome} aims to provide a full set of descriptions of the scenes which images depict in order to spur complete scene understanding. 
It contains a dense region-based labeling of $108k$~images with textual expression of the attributes and references of objects, their relationships as well as question answer pairs, all linked to WordNet synsets \cite[see below]{fellbaum1998wordnet}. 


\begin{itemize}
     		\item[(1)] \textbf{Specific categories}: are not available, as object categories and names are not consistently annotated (and even conflated)
				\item[(2)] \textbf{Exhaustive annotations}: are available, which is a huge advantage of this data sets
		   \item[(3)] \textbf{Natural names}: are available, though object names might not be fully discriminative (as in referring expressions)

\end{itemize}

describe sampling of images, category selection

Table with:

% \begin{itemize}
% \item rows: domains (if we go for long paper: then one row per collection node?)
% \item columns:
%   \begin{enumerate}
%   \item \# collection nodes
%   \item collection nodes (list)
%   \item \# unique VG names
%   \item example VG names
%   \item \# unique objects
%   \item \# unique images (? not sure if necessary; maybe only one of unique {objects, images})
%   \end{enumerate}
% \end{itemize}

\subsection{Procedure} describe the crowdsourcing set-up and the task

\subsection{Data} give an overview of the collected data

%%% Local Variables:
%%% mode: latex
%%% TeX-master: "main"
%%% End:
