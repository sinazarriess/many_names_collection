\section{Related Work}
\label{sec:related}

\paragraph{Cognition: Concepts and categorization}

 Seminal work on concepts by Rosch suggests that object names typically exhibit a preferred level of specificity called the \textbf{entry-level}. This typically corresponds to an intermediate level of specificity, i.e., \textbf{basic level} (e.g, \refexp{bird}, \refexp{car}) \cite{rosch1976basic}, as opposed to more generic (i.e., \textbf{super-level}; e.g., \refexp{animal}, \refexp{vehicle}) or specific categories (i.e., \textbf{sub-level}; e.g., \refexp{sparrow}, \refexp{convertible}). However, less prototypical members of basic-level categories tend to be instead identified with sub-level categories (e.g., a \cat{penguin} is typically called a \refexp{penguin} and not a \refexp{bird}) \cite{jolicoeur1984pictures}. 
%This out-of-context preference towards a certain taxonomic level is often referred to as \textbf{lexical availability}. 
While the traditional notion of entry-level categories suggests that objects tend to be named by a \refexp{single} preferred concept, research on pragmatics has found that speakers are flexible in  
%with respect to the chose level of specificity. 
their choice of the level of specificity. 
Scenarios where multiple objects (of the same category) are present induce a pressure for generating names which uniquely identify the target \cite{olson1970language}, such that sub-level names can be systematically elicited in these cases %\cite{rohde2012communicating} \cite{graf2016animal}. 
\cite{rohde2012communicating,graf2016animal}.

\paragraph{Vision: Object Recognition}

State-of-the-art computer vision systems are able to classify images into thousands of different categories (e.g.\  \newcite{googlenet}). These object recognition systems are now widely used in vision \& language research.
Nevertheless, the way the treat object recognition is conceptually very simple (if not to say, naive):  standard object classification schemes are inherently ``flat'', and treat object labels as mutually exclusive \cite{deng2014large}, ignoring all kinds of linguistic relations between these labels and ignoring the fact that an object can easily be an instance of several categories.\cs{I would make this statement stronger and argue that \textbf{object recognition is merely a labeling of objects  with human interpretable symbols}, and that a system would probably fail if it had to decide whether an object labeled as, e.g.\ \refexp{fig} may also be labeled as \refexp{food}.}


\paragraph{Vision \& language: Naming and Referring}

\newcite{Ordonez:2016} have studied the problem of deriving appropriate object names, or so-called entry-level
 categories, from the output of an object recognizer. Their approach focusses on linking abstract object categories in ImageNet to actual words via translation procedures that e.g. involve corpus frequencies. 
 \newcite{zarriess-schlangen:2017} learn a model of object naming on a corpus of referring expressions paired with objects in real-world images, but focus on combining visual and distributional information and on zero-shot learning.
 Thus, object naming is an important task for referring expression generation, though most research in this area has focussed on content and attribute selection \cite{Kazemzadeh2014,gkatzia:2015,zarrieschlang:easy-pre,Maoetal:cocorefexp}.

\paragraph{Existing resources and their shortcomings}
Moreover, existing resources in L\&V hardly provide any consistent taxonomic information on objects and their categories, e.g. object labels are typically quite general as in Flickr30k \cite[e.g.,~\cat{people, animals, bodyparts, clothing}]{plummer2015flickr30kentities} or taxonomically heterogeneous as in MS COCO \cite[e.g.,~\cat{people, baseball glove, bird}]{mscoco}.


%%% Local Variables:
%%% mode: latex
%%% TeX-master: "main"
%%% End:
