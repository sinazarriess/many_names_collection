\documentclass[10pt, a4paper]{article}
\usepackage{lrec}
%\usepackage{multibib}
%\newcites{languageresource}{Language Resources}
\usepackage{graphicx}
\usepackage{tabularx}
\usepackage{soul}
% for eps graphics
\usepackage[usenames, dvipsnames]{color}

\usepackage{epstopdf}
\usepackage[latin1]{inputenc}
\usepackage{xspace}
\usepackage{booktabs}
\usepackage{hyperref}
\usepackage{xstring}
\usepackage{xspace}
\usepackage{multirow}
\usepackage{colortbl}
\usepackage{xcolor}

\newcommand{\secref}[1]{\StrSubstitute{\getrefnumber{#1}}{.}{ }}

\newcommand{\refexp}[1]{\textsl{#1}}
\newcommand{\word}[1]{\textsl{#1}}
\newcommand{\cat}[1]{\textsc{#1}}
\newcommand{\vgenome}{VisualGenome\xspace}
\newcommand{\vg}{VG\xspace}
\newcommand{\ra}{$\rightarrow$}
\newcommand{\referit}{ReferIt\xspace}
\newcommand{\refcoco}{RefCOCO\xspace}
\newcommand{\refcocop}{RefCOCO+\xspace}
\newcommand{\flickr}{Flickr30k Entities\xspace}

\newcommand{\sz}[1]{\textcolor{blue}{\emph{//sz: #1//}}}
\newcommand{\gbt}[1]{\textcolor{orange}{\emph{//g: #1//}}}
\newcommand{\cs}[1]{\textcolor{green!60!black}{\emph{//cs: #1//}}}

\definecolor{lightgray}{gray}{0.85}


%\title{Naming Objects in Real-world Images: A survey and A new \sz{linguistically motivated????} collection}
\title{Object Naming: A Survey and a New Dataset}

\name{Author1, Author2, Author3}

\address{Affiliation1, Affiliation2, Affiliation3 \\
         Address1, Address2, Address3 \\
         author1@xxx.yy, author2@zzz.edu, author3@hhh.com\\
         \{author1, author5, author9\}@abc.org\\}


\abstract{
Massive data collections for applications in language \& vision are nowadays available. In principle, these could constitute valuable resources also for research in computational linguistics (CL), besides task-specific modeling. However, in practice, very few studies have tested linguistic hypotheses using these large-scale datasets. In this paper, we illustrate the challenges of using this type of corpora for CL research, by focusing on a case study, namely naming of objects in real-world images. Our analysis of existing resources in language \& vision reveals that available datasets do not generally provide the right type and consistent quality of object naming data,  for being able to conduct a linguistically motivated study of this phenomenon. We contribute a new dataset on top of Visual Genome that  \sz{... continue}
 \\ \newline \Keywords{keyword1, keyword2, keyword3} }

\begin{document}

\maketitleabstract

\section{Introduction}
\begin{figure}[tbp]
\scriptsize
\begin{tabular}{p{4.3cm}p{2cm}}
%VisualGenome+ManyNames & \cite{snodgrass}\\
\centering
\includegraphics[scale=0.15]{figures/2390077_1254219_supercat_unique.png} &
\includegraphics[scale=0.4]{figures/snodgrass_vanderwart_cake_042.png}\\
 cake\ (53),  food\ (19), bread\ (8), burger\ (6), dessert\ (6), snacks\ (3), muffin\ (3),  pastry\ (3) & \hspace{.9cm} cake (83)
\end{tabular}
\caption{Names for a cake object in ManyNames (left) and in \citet{snodgrass} (right), percentages of responses in parentheses.}
\label{fig:cake}
\vspace{-0.5cm}
\end{figure}

Generally, research in Language \& Vision (L\&V) is interested in modeling how speakers \textit{naturally} name, refer to or talk about visual objects and scenes, in contrast to predicting abstract object labels as e.g.\ in Computer Vision.
This typically entails that data collections and models need to account for linguistic variation, as there can hardly ever be a single ground-truth utterance when referring to some visual entity. 
While variation has indeed been accounted for in L\&V tasks like image captioning \cite{vedantam2015cider,Bernardietal:automatic,dai2017towards}, object naming has been addressed in a comparatively simplistic way: 
even corpora with massive annotation like VisualGenome~\cite[\vg henceforth]{krishna2016visualgenome} typically provide a single label for each object assumed to be its canonical category or name.

%While many of these tasks are well-known to 
%\gbt{I've left only one image in the figure to highlight that we study variation for the same instance.}
%However, so far research in NLP has had surprisingly little to say about object naming.
%Neighboring areas
%, specifically Computer Vision and Cognitive Science, 
%have addressed related tasks, with simplifications that NLP is equipped to address:
%In Computer Vision, naming is usually equated with object categorization, and addressed as a classification problem in which a single label is provided for each object~\cite{googlenet}.
%In a typical evaluation, alternative names for the object in Figure \ref{fig:cake} such as \refexp{cake}, \refexp{dessert}, \refexp{sweet}, or \refexp{food} would be counted as incorrect.
%In Cognitive Science, object naming has received more attention, but it has been studied with stylized drawings instead of realistic images.
%, such that it is unclear how findings generalize to tasks in language \& vision.

We take a first step at studying natural naming of objects in real-world images and contribute a new dataset, ManyNames, that contains 36 crowd-sourced names for 25K instances from \vg.
Thus, our images show objects in complex visual contexts,
unlike the ``clean'' ImageNet data~\cite{imagenet_cvpr09} that has been previously used to train object classifiers \cite{ILSVRC15}, and unlike stylized line drawings used in picture naming experiments in Cognitive Science (see Figure \ref{fig:cake}).

As illustrated for an object of the class ``cake'' in Figure \ref{fig:cake}, our data reveals clear naming preferences (53\% of the annotators prefer the basic-level \textit{cake}) and also rich variation (the remaining annotators prefer other options like \textit{food, dessert, bread}) which is not restricted to taxonomical relations studied in previous work on naming \cite{rosch1976basic,Ordonez:2016,graf2016animal}. 
%We find both consistency and variation in naming: In instances, the relative frequency of the most common name is 75\% on average, which is remarkable for a task where subjects are allowed to produce whatever name they like. However, 
%there are still around 3 names being produced for each instance on average.
%Moreover, 
%we find a high level of variation within what would be considered a class in visual object recognition, with around 30 names per class.
%standard Computer Vision approaches (objects assigned the same synset in VisualGenome)
%Interestingly, most of this variation comes from alternative names that do not stand in a taxonomic relation, but range from 
%We also find a very high standard deviation in the agreement measures, which suggests that visual features (how prototypical the instance is of a given category, how clearly delimited the object is) are crucial factors determining variation.
%Our data also show that object identification via bounding boxes is not a trivial matter even for humans, with name variants ranging from clearly different objects in 
%cross-classifification  (\textit{cheesecake}-\textit{dessert}) to cases
% where it is visually impossible to distinguish between the two (\textit{bed}-\textit{bedsheet} when the bed is only partially shown) to cases 
%of near-metonymy (\textit{floor}-\textit{carpet}) and issues in object idenitification due to bounding boxes (\textit{man}-\textit{helmet}).
%Given these findings, we ask whether current models implicitly capture object naming variation of the sort provided by humans.
Interestingly, when testing a model of object recognition, that has been trained on \vg, on ManyNames, we observe better performance than on \vg.
%: It has 77\% accuracy when taking the most frequent ManyNames name as gold standard, vs.\ only 69\% on the VisualGenome name.
This suggests that object names annotated by many speakers provide a more robust ground for testing computational models, while also offering rich data for studying linguistic variation in naming.

% \item most of the variation in our dataset comes from alternative names that do not stand in a taxonomic relation, suggesting that the previous work in Cognitive Science is missing much of the empirical ground.
% %while previous work has mostly focused on variation in the level of generality within a taxonomy (\emph{penguin} vs. \emph{bird}), 

% our datasets contains a lot of variability for names coming from different parts of the taxonomy (\emph{dessert} vs. \emph{cake}, \emph{bottle} vs. \emph{wine})
% \end{itemize}

% Moreover, we analyze whether current models implicitly encode the variation in naming, by doing XXX. We find YYY.


% Our paper puts together two strands of research that have mostly been pursued independently to date.
% On the one hand, state-of-the-art computer vision systems are able to accurately classify images into thousands of different categories (e.g.\  \newcite{googlenet}), where the task is often to predict the class for a given object. 
%name for a given object. \gbt{Is this true? Imagenet task asks for synsets, which can be taken to be categories\dots To refine}
% However, they mostly adopt very simple assumptions with respect to the underlying lexicon, which is implemented by using single "ground truth" labels: 
%as a simple, flat labeling scheme:
 % A standard object recognition system would be trained to classify the left object in Figure~\ref{fig:cake} as \emph{cheesecake}, the right one as \emph{dessert}, and using \emph{dessert} for the left picture would be considered incorrect. 
% On the other hand, research on object naming in Cognitive Science has shown that people choose different names depending on the circumstances, with factors such as context or the prototypicality of the object with respect to the category playing a role~\cite{add-refs}. 
% \gbt{This research also argues that there is high agreement in how people name objects; to do: make coherent.}
% However, this research typically uses stylized drawings are used, and is focused on taxonomic relations (\textit{sparrow}-\textit{bird}).
% \sz{It is thus unclear how findings from these stylized settings generalize to tasks in language \& vision like referring expression generation, where naming is a core aspect. Therefore, in contrast to traditional naming norm studies in Cognitive Science we study object naming in realistic scenes where objects are situated in a natural context! (This comes with additional challenges, like potential object occlusion, background/foreground confusion etc.)}

% Seminal work on prototypes suggests that the prototypicality of the object will determine the level of generality of the object name, i.e.\  a robin can be named \emph{bird}, but a penguin is better referred to as ``\emph{penguin}'' \cite{Rosch1978}.

% Two main findings in the literature:

% \begin{itemize}
% \item very high agreement, most people use the same name for the same object
% \item prototypicality is a factor, context too, but research has only looked at generality/specificity of the name
% \end{itemize}

%The real-world objects that we interact with in our every-day life can be categorized into many thousands and maybe millions of categories. And even a single object can be member of many categories, i.e.\ at different taxonomical levels or in different parts of a taxonomy. For instance, both objects in Figure \ref{fig:cake} are at once instances of \cat{cake}, \cat{cheesecake}, \cat{dessert}, \cat{sweet}, \cat{pastry}, \cat{food} etc. Hence, when speakers name objects, e.g.\ when referring, they have to select a lexical item from a complex network of concepts and competing lexical alternatives.

%To date, research in NLP has surprisingly little to say about object naming, despite the fact that
% there has been a recent explosion of interest in various, and even complex, language \& vision tasks ranging from image captioning \cite{fangetal:2015,devlin:imcaqui,Bernardietal:automatic} to e.g.\ visual dialogue \cite{das2017visual,vries2017guesswhat}. 
%In contrast, closely related areas, such as computer vision and cognitive science, have investigated very related tasks in quite some depth: object recognition systems developed in the area of computer vision  are now able to classify images into thousands of different categories (e.g.\  \newcite{googlenet}).
%Furthermore, work on concepts, following the seminal work by Rosch, suggests that objects are typically conceptualized at a preferred level of specificity called the \textbf{entry-level}. Psycho-linguistic studies have been able to support this theory based on collections of so-called object naming norms. 
%
%This paper aims at addressing the genuinely linguistic questions revolving around the phenomenon of object naming by (i) presenting a collection of high-quality, large-scale naming data,  and (ii) analysis methods for this data and (iii) a first baseline model that accounts for the semantic flexibility of names for objects in real-world images. From computer vision, we borrow the idea of modeling realistic visual objects in realistic scenes (real-world images), but go beyond the simplistic assumption that object names correspond to unambiguous labels in a flat classification scheme (with no conceptual relations between the labels). From psycholinguistics, we borrow the idea of eliciting natural, representative naming data from many subjects, but go beyond using artificial, highly stylized objects.

%%% Local Variables:
%%% mode: latex
%%% TeX-master: "main"
%%% End:


\section{Background}
\label{sec:rel-work}
\subsection{Object Naming as a Linguistic Phenomenon}
\label{subsec:rosch}

The act of naming an object amounts to that of picking out a nominal to be employed to refer to it (e.g., ``the \refexp{dog}'', ``the white \refexp{dog} to the left'').
Since an object is simultaneously a member of multiple categories (e.g., a young beagle belongs to the categories \cat{dog}, \cat{beagle}, \cat{animal}, \cat{puppy}, \cat{pet}, etc.), all the various names that lexicalize these constitute a valid alternative, meaning that the same object can be called by different names \cite{brown1958shall,murphy2004big}.

Seminal work by \newcite{rosch1976basic} inspired a taxonomic view of object naming, in which names exhibit a preferred level of specificity or abstraction called the ``entry-level'' \cite{jolicoeur1984pictures}. 
This typically corresponds to an intermediate level of specificity (basic level,\ e.g., \refexp{bird}, \refexp{car}), as opposed to more generic (super-ordinate,\ e.g., \refexp{animal}, \refexp{vehicle}) or more specific categories (sub-ordinate,\ e.g., \refexp{sparrow}, \refexp{convertible}).
However, less prototypical members of basic level categories tend to be instead identified with sub-ordinate categories (e.g., a penguin is typically called \refexp{penguin} and not \refexp{bird}; \newcite{jolicoeur1984pictures}). 

%This out-of-context preference towards a certain taxonomic level is often referred to as \textbf{lexical availability}. 
While the traditional notion of entry-level categories suggests that objects tend to be named by a \refexp{single} preferred concept, research on pragmatics has found that speakers adopt their naming choices to the context and, hence, are flexible with respect to the chosen level of specificity \cite{olson1970language,rohde2012communicating,graf2016animal}.
%Scenarios where multiple objects (of the same category) are present induce a pressure for generating names which uniquely identify the target \cite{olson1970language}, such that sub-level names can be systematically elicited in these cases \cite{rohde2012communicating,graf2016animal}.
For example, in presence of more than one dog, the name \textsl{dog} is ambiguous and a sub-ordinate category (e.g., \textsl{rottweiler}, \textsl{beagle}) is potentially preferred by speakers. The effect of such distractor objects on the production of referring expressions has been widely examined in the language generation community \cite{krahmer:2012}, though not specifically for object naming. We believe that our new dataset provides an interesting resource for tackling this question.
%, though additional factors such as cost or saliency also come into play \cite{graf2016animal,clark1983common}.

The purely taxonomic view on naming has also been criticized in work on object organization, which found that many objects of our daily lifes are part of multiple category systems at the same time \cite{ross1999food,SHAFTO20111}. 
This \textit{cross-classification} occurs, for instance, with food categories which can be taxonomy-based (e.g.\ \refexp{meat, vegetable}) or script-based (e.g.\  \refexp{breakfast, snack}).
We provide tentative evidence that cross-classification is indeed relevant for naming variation, and that the taxonomic axis is not the most frequent source of variation in our data.

\subsection{Picture Naming in Cognitive Science}

An important experimental paradigm in work on human vision and categorization is picture naming, where subjects have to say or write down the first name that comes to mind when looking at a picture of (typically) a line drawing depicting a prototypical instance of a category \cite{snodgrass,rossion2004revisiting}, see Figure\ \ref{fig:cake}.
Subjects reach very high agreement in this task \cite{rossion2004revisiting}, i.e. for a given object, there is a clear tendency towards a certain name across all speakers.
The resulting naming norms are useful for studying various cognitive processes \cite{humphreys1988cascade}.
Our task is inspired by picture naming, but uses real-world images showing objects in context.

\subsection{Object Recognition in Computer Vision}

In Computer Vision, object recognition is often modeled as a classification task where state-of-the-art systems localize and classify objects into thousands of different categories  \cite{googlenet,ILSVRC15}. 
Current recognition benchmarks use labels and images from the ImageNet \cite{imagenet_cvpr09} ontology, and typically assume a single ground-truth label. 
The construction of ImageNet was set up as a two-stage procedure: (i) images for given categories in the ontology were automatically collected by querying search engines, (ii) crowd-workers then verified whether each candidate image is an instance of the given category.
Other data collection efforts for object labels also used a predefined vocabulary and asked annotators to mark all instances of these categories in a set of images \cite{mscoco,OpenImages}. 
Recently, \newcite{pont2019natural} have argued for annotation of object labels using free form text though here this free vocabulary is then mapped to a set of underlying classes.
Thus, even though object recognition benchmarks do provide images of objects and categories, they generally do not provide what we are interested in in this work, namely natural names of objects.

\subsection{Object Naming in L\&V} 

Previous work in L\&V has collected and used data sets where annotators produced free and natural utterances for a given image. 
Moreover, these data sets typically record utterances that are more complex than a single word, such as image captions \cite{fangetal:2015,devlin:imcaqui,Bernardietal:automatic}, referring expressions \cite{Kazemzadeh2014,mao15,Yu2016}, visual dialogues \cite{das2017visual,vries2017guesswhat} or image paragraphs \cite{krause2017hierarchical}. While object names occur in all of these data sets, they are not necessarily marked up and linked to the corresponding image regions. The overview in Section \ref{sec:survey} will discuss corpora where the grounding of names to regions for objects is given, as in the case of \vgenome \cite{krishna2016visualgenome}, or where it can be easily derived, as in the case of referring expressions.

Our new collection, ManyNames, focusses on object names in isolation and is substantially more controlled than common L\&V data sets. This controlled collection procedure allowed us to elicit many annotations for the same object from different annotators, resulting in a data set that is amenable to studying variation and preferences in naming systematically and on a large scale.

\begin{table*}[htb]
  \centering
  \begin{tabular}{lrrrrr}
    \toprule
    &   RefCOCO/+  &  Flickr30kE &           VG &      VGmn &        MN \\
    \midrule
    \# objects & 50,000 & 243,801 & 3,781,232 & 25,315 & 25,315 \\
    naming vocab size &  5,004 &  10,423 &   105,441 &  1,061 &  7,970 \\
    av. annotations/object &      2.8 &       2.3 &         1.7 &      7.2 &     35.3 \\
    ratio of objects with n types $>$ 1 &      0.7 &       0.3 &         0.02 &      0.05 &      0.9 \\
    av. types/object &      1.9 &       1.4 &         1 &      1.1 &      5.7 \\
    \bottomrule
  \end{tabular}
  \caption{Overview statistics for different data sets containing object naming data. VGmn shows statistics for the subset of \vg that overlaps with our ManyNames dataset.\label{tab:compare}}
\end{table*}


%Work that models which \textbf{word} (as opposed to a category label) a speaker will use to name an object is relatively scarce.
%Natural language generation has intensively investigated referring expressions~\cite{dale:1995,krahmer:2012}; however, this area has focused mostly on the selection of attributes, typically assuming that the name is given.
%\gbt{Add example}
%\newcite{Ordonez:2016} takes up the notion of entry-level categories and transfers an object's predicted label to its name.
%Their model classifies objects into fine-grained categories (\gbt{e.g., ++++++ }), and then predicts a WordNet synset to retrieve the name (e.g., \word{swan}), based on frequencies in a text corpus.
%This work assumes that \gbt{+++++complete please, stating what's different to ours}
 %\newcite{zarriess-schlangen:2017} learn a naming model on referring expressions and real-world images, but focus on combining visual and distributional information. 
%\gbt{I don't understand how this differs from other research and our own.}
%Recent experimental work on reference found that the specificity of a name is dependent on the taxonomic relatedness of other objects in context
%\cite{rohde2012communicating,graf2016animal}. 
%However, this work studies a very limited set of images \gbt{+++++complete please, stating what's different to ours}
%Our work is a first step towards studying naming in real-world, natural reference.
%As there is virtually no existing large-scale resource that provides robust naming data elicited from multiple subjects \textit{and} for instances in real-world images, this paper focuses on naming in isolation, rather than reference where naming interacts with attribute selection.
%
%
%
%This line of research has emphasized the taxonomic organization of categories, following the seminal work on prototypes by \newcite{rosch1976basic} mentioned above.
%These works propose granularity-aware object recognition methods, that incorporate the taxonomic structure underlying object labels in multi-label settings; for the ``young beagle'' example, labels \word{beagle}, \word{dog}, \word{animal} would all be considered valid.
%Instead, other sources of variation like cross-classification have not received attention in L\&V.
%As mentioned above, our results suggest that cross-classification occurs very frequently when naming objects in real-world images.


%%% Local Variables:
%%% mode: latex
%%% TeX-master: "lrec2020naming"
%%% End:


\section{Object Names in Existing L\&V resources}
%\cs{If space allows, I would also put an example image+annotation for RefCOCO and Flickr30kE}
%\gbt{Space doesn't seem to allow :)}
We identified three previously existing resources that can be of use for analysis and modeling of object naming: RefCOCO (and a variant, RefCOCO+), Flickr30k Entities, and Visual Genome. %
Table~\ref{tab:compare} summarizes their main characteristics and compares them to our dataset (last two columns; see Section~\ref{sec:data}).
As the table shows, previous datasets provide between one and three annotations per object, which, we believe, is not enough to assess naming behavior for individual objects and which
motivates our data collection. %, in which we collect 36 names per object.
In the following, we will look at their characteristics in more detail and work out requirements for a dataset that is suitable for a large-scale study of object naming.

\subsection{\refcoco and \refcocop}

Both \refcoco and \refcocop \cite{Yu2016} use the \referit\cite{Kazemzadeh2014} game for collecting referring expressions (RE) for natural objects in real-world images, and are built on top of MS COCO \cite{mscoco}, 
a dataset of images of natural scenes of $91$~common object categories (e.g.,~\cat{dog, pizza, chair}). 
The REs were collected via crowdsourcing in a two-player reference game designed to obtain REs uniquely referring to the target object. 
Specifically, a director and a matcher are presented with an image, and the director produces a RE for an outlined target object. 
The matcher must click on the object she thinks the RE refers to. % (For more details on the datasets see \cite{Yu2016}). 
REs in \refcoco/+ were collected under the constraints that (i) all images contain at least two objects of the same category (80 COCO categories), which results in longer and more complex REs than just the object name, and (ii) in \refcocop the players cannot use location words, urging them to refer to the appearance of objects.
 % \gbt{How come the naming vocabulary size is so large if RefCOCO contains only 80 categories? How did you compute these numbers, are they for names, or whole REs?} \sz{well, 80 categories doesn't mean much as the categories can be very general (like "human") ... and it nicely shows that natural names are much more variable than strict categories}
% Another critical property of the data is that, (iii), not all objects in an image were annotated with REs, may it due to the frequency constraint~(i), or due to the object not being part of the 80 COCO categories.

Table~\ref{tab:compare} shows that the multiple annotations ($2.8$~on average) actually contain a considerable amount of variation in naming (almost~$2$ different names on average per object). 
%However, the small number of annotations per object does not allow for a reliable assessment of speaker agreement. 
However, the small number of annotations per object does not allow to reliably infer object-specific naming preferences or assess speaker agreement. \cs{double-check formulation}

RefCOCO has been used to model and examine the effect of context on referring expression generation in general \cite{Yu2016}, though this work did not look at object names specifically. 
A controlled analysis of the effect of context on choice in naming, as for instance in \cite{graf2016animal}, would require substantial further data annotation as not all objects of an image are annotated with REs and corresponding categories. Hence, so-called distractor objects  \cite{krahmer:2012} and their names cannot be analyzed systematically.
Also, while in \refcoco the elicited names can be assumed to be natural, it is unclear how the additional constraints in \refcocop impact on the naturalness of object naming.
Finally, the underlying set of MS COCO categories is quite small ($80$~categories).
To sum up, \refcoco is suitable for generally modeling referring expressions in context for a restricted set of categories, but less appropriate for analyzing object naming at a large scale.

% \begin{itemize}
% \item[(1)] \textbf{Specific categories}: not available, the $80$~COCO categories tend to be entry-level categories and are not linked to the ImageNet taxonomy (e.g.,~\cat{bird, person, car, bus})
% \item[(2)] \textbf{Exhaustive annotations}: not available, as not all objects were annotated with REs and corresponding categories
% \item[(3)] \textbf{Natural names}: available, though it is unclear how the additional constraints in RefCoco+ impact on the naturalness of object naming
% \end{itemize}

% gbt: I'm commenting out this analysis because it is not really relevant for assessing how useful RefCOCO is to study object naming: what it shows is that names appear at many different levels of a taxonomy, although they tend to be concentrated in middle levels (especially 6-7; in agreement with basic level hypothesis). To me, this is a data result, not a dataset result (clear?). 
% \paragraph{Analysis} We parse REs in \refcoco with the Stanford Dependency Parser and extract the nominal heads. We map these names to their most frequent sense/synset in WordNet.
% We hypothesize that the distance of a name's synset to the root node (\cat{entity}) relates to its specificity.
% We estimate this distance as the minimal path length of all synsets of a word  to the root node.
% Table~\ref{tab:specnames} shows the estimated levels of specificity for object names in the \refcoco data set.
% We observe distances to the root between 2 and 17, meaning that there is a much more fine-grained distinction of levels than the three-way classification adopted in \cite{graf2016animal}.
% Unfortunately, the levels of specificity predicted by WordNet do not seem to reflect linguistic intuitions, e.g.\ \refexp{elephant} is predicted to be more specific than \refexp{panda}.
% At the same time, this overview clearly suggests that object names in \refcoco do not only comprise entry-level categories, but also very general (\refexp{thing}) and very specific names (\refexp{ox}).

% \begin{table*}
% \centering
% \setlength{\tabcolsep}{2pt}
% \begin{small}
% \begin{tabular}{rrl|rrl}
% \toprule
%  spec. &  rel.freq. &                          top 5 names & spec. &  rel.freq. &                          top 5 names \\
% \midrule
%            2 &   $<$ 0.01 &       \tiny                  thing,things & 10 &   0.05 &   elephant,couch,truck,vase,suitcase \\
%            3 &   $<$ 0.01 &    object,group,set,substance,objects & 11 &   $<$ 0.01 &    motorcycle,clock,mom,dad,scissors \\
%            4 &   0.14 &           man,person,piece,head,part & 12 &   $<$ 0.01 &  oven,airplane,suv,taxi,refrigerator  \\
%            5 &   0.10 &       player,glass,baby,front,corner & 13 &   $<$ 0.01 &    laptop,fridge,canoe,orioles,pigeon \\
%            6 &   0.21 &              woman,girl,kid,boy,bowl & 14 &   $<$ 0.01 &   panda,freezer,penguin,rooster,rhino \\
%            7 &   0.25 &            guy,right,chair,lady,bear & 15 &   0.03 &    zebra,giraffe,zebras,giraffes,deer \\
%            8 &   0.11 &           horse,bus,cow,pizza,batter & 16 &  $ <$ 0.01 &       bison,mooses,orang,elks,sambar \\
%            9 &   0.09 &         shirt,car,bike,donut,catcher & 17 &   $<$ 0.01 &           ox,cattle,gnu,mustang,orca \\          
% \bottomrule
% \end{tabular}\caption{Levels of specificity for naming choices in RefCOCO: for each level (distance between name and WordNet root), relative frequency and 5 most frequent names are shown}
% \label{tab:specnames}
% \end{small}
% \label{tab:specnames}
% \end{table*}

\subsection{\flickr}

The \flickr dataset \cite{plummer2015flickr30kentities}
% \footnote{Available at  \url{web.engr.illinois.edu/~bplumme2/Flickr30kEntities}}
augments Flickr30k, a dataset of 30k~images and five sentence-level captions for each of the images, with region-level descriptions extracted from the captions.
Specifically, mentions of the same entities across the five captions of an image are linked to the bounding boxes of the objects they refer to.
% The dataset was designed to advance image description generation and phrase localization in particular \cite{rohrbach2016grounding,plummer2017phrase,yeh2018unsupervised}.

This dataset has three main differences with respect to \refcoco/+: (i) the entity mentions were obtained via an image description task (captioning), as opposed to a referential task; (ii) the images and the production of entity mentions were not subject to any constraints; (iii) a much wider range of categories are covered (cf.\ the number of objects and the vocabulary size in Table~\ref{tab:compare}). %\gbt{Any idea what kind of categories are covered in \flickr?}
Moreover, although no exhaustive annotations of the images are available, the dataset does contain information for the most salient objects in the image, as they are typically mentioned in the captions.
The number of annotations per object, $2.3$\ annotations, is comparable to \refcoco.
This dataset is suitable to analyze object naming in descriptions, for a quite large set of categories (although, again, not enough annotations are available to analyze image-specific naming data).

% \begin{itemize}
% \item[(1)] \textbf{Specific categories}: are not available, object categories tend to be even less specific than those of COCO (e.g.,~\cat{people, animals, bodyparts, clothing}), or are abstract (\cat{other, scene})
% \item[(2)] \textbf{Exhaustive annotations}: are not available
% \item[(3)] \textbf{Natural names}: are available, though object names might not be fully discriminative (as in REs; e.g.,~both animals in the right-most image in Fig.~\ref{fig:graf_genome} are named \refexp{dog})
% \end{itemize}

\subsection{Visual Genome}
\label{sec:vg-survey}

\vgenome (VG, \newcite{krishna2016visualgenome}) is one of the most densely and richly annotated resources currently available in L\&V; here, we focus on aspects immediately relevant to object naming.
%In the following, we will focus on describing aspects immediately relevant to object naming only, but many other annotations are available as well (e.g. questions, paragraphs, etc.)
%\paragraph{Collection and annotation procedure}
\vg aims at providing a full set of descriptions of the scenes which images depict in order to spur complete scene understanding. 
The data collection followed a complex procedure, involving many different rounds of annotation.
The first round of the procedure, and the basic backbone for the further rounds, is a collection of region-based descriptions: workers were asked to describe regions in the image and draw boxes around the corresponding area in the image (for examples, see Figure~\ref{fig:bird}).

In a second, independent round (involving new workers), annotators were asked to process the region descriptions by (i) marking the object names contained in the region description, and (ii) drawing a tight box around the corresponding region. As different region descriptions can potentially mention the same objects, each worker was shown a list of previously marked objects and encouraged to select an existing object rather than annotating a new one.


Some of the main advantages of \vg are its size, with $3.8$~million objects ($108$K images) as opposed to $50$K and $243$K for the other two datasets, and its category coverage, with a vocabulary of object names of $105$K compared to $5$K/$10$K.
Another plus is the fact that it in principle provides exhaustive annotations of objects in an image, often with several region descriptions and possibly object names per object.
This should make it easier to identify factors intervening in naming choices, and to model contextual aspects that may affect them, than in the case of \refcoco.

However, there is a crucial pitfall: As Figure~\ref{fig:bird} shows, there is only a partial linking of objects that are mentioned across different region descriptions; for instance, the first, second, and fourth object ID in the figure actually correspond to the same object.
Moreover, the region for the beak of the object (third object IDs) overlaps with those of the bird. 
%This means that the identity of objects cannot be established based on the annotation, which severely limits the usefulness of the data to analyze naming.
%For instance, even though there is a different name (\word{vulture}) for \word{bird} in Figure~\ref{fig:bird}, the annotation suggests that \word{bird} is the only available name. 
Finally, even though there is a different name (\word{vulture}) for \word{bird} in Figure~\ref{fig:bird}, the annotation suggests that \word{bird} is the only available name. 
Hence, the identity of objects cannot be established based on the annotation, which severely limits the usefulness of the data to analyze naming.
The relatively low number of $1.7$~annotations per objects on average in VG (Table~\ref{tab:compare}) and the very small number of objects that have more than one name associated with it ($2$\%) seem to be an effect of this partial linking problem.
We experimented with filtering and merging bounding boxes based on overlap, but this would introduce substantial noise into the data (e.g.,~truly overlapping objects).

Table~\ref{tab:compare} also shows the statistics for the subset of those VG objects that we selected for ManyNames and, here, we find a considerably higher average of $7$~annotations per object. 
We think that this might be an effect of our category selection procedure explained in Section~\ref{sec:data}. However, interestingly, the portion of objects that have different names associated with them is still extremely small.
Note that in contrast, even though RefCOCO has much less annotations per object, there are many objects with different names ($70$\%).


%\cs{The explanation of intervening factors in object naming in the background may be a bit more detailed, such as to really understand the drawbacks of the existing datasets with respect to distractors.}

\subsection{Discussion}

While some existing resources do provide naming data for objects in context, they do not provide \textit{enough} data to systematically assess how variable or stable object naming really is. The RefCOCO data (and to some extent the Flickr30k data) suggests that for most objects there is more than one available name, but it is unclear which name most speakers would prefer or whether there is such a preferred name at all. 
The VG data, to the contrary, seems to indicate that the vast majority of objects should only be associated with a single name, but it is difficult to estimate to what extent this finding results from problems with annotation (partial linking).
This shows that to be able to analyze object naming in more detail, it is crucial to have naming data from many subjects for the same objects. 
Also, dense annotations of images can be beneficial to analyze the factors affecting naming (e.g.,\ the category or salience of other objects), and how these impact the modeling of natural language in L\&V.
These are the motivations for our dataset, ManyNames, and for building it on top of \vg, as discussed next.

\begin{figure}
\begin{center}
%\fbox{\parbox{6cm}{
%This is a figure with a caption.}}
\includegraphics[scale=0.2]{figures/vulture.png} 
\begin{tabular}{lp{6.2cm}}
object id & linked region descriptions\\
\hline
3595788 & the \textbf{bird} is black in color, nose of the \textbf{bird}, a \textbf{bird} relaxing in stand, small white beak of \textbf{bird}, large black talon of \textbf{bird}, a \textbf{bird} on a green pole, a green bar under \textbf{bird}, black \textbf{bird} on green rail, small black eye of \textbf{bird}\\
2286017 & large black \textbf{vulture} on fence, a vulture on bar\\
%2385747 & small white beak of \textbf{bird}\\
2681429 & a semi \textbf{long beak}\\  
2346210 & a black and gray \textbf{vulture}\\
 \end{tabular}
\caption{Bounding boxes, names and region descriptions for an object in VisualGenome}
\label{fig:bird}
\end{center}
\end{figure}

%%% Local Variables:
%%% mode: latex
%%% TeX-master: "lrec2020naming"
%%% End:


\section{A New Dataset: ManyNames}
\label{sec:data}
% Number of images/objects:        25,596\\
% Number of object names:  450\\
% Number of collection nodes (synsets):    52 \\

We take data from \vgenome \cite[\vg henceforth]{krishna2016visualgenome}, which
% aims to provide a full set of descriptions of the scenes which images depict in order to spur complete scene understanding. 
contains a dense region-based labeling of $108k$~images with, inter alia, objects, attributes and relationships,  %object descriptions, attributes, and relationships, as well as question-answer pairs, 
all linked to WordNet synsets \cite{fellbaum1998wordnet}.
\vg is suitable for our purpose of collecting names for a relatively large amount of instances of common objects in
naturalistic images, as it has images of varying complexity, with close-ups as well as complex images with many objects.
As common in Computer Vision, objects are localized as 
%identified via
 bounding boxes (see red boxes in Figure~\ref{fig:cake}).% 
\footnote{We use image and object interchangeably in the following, since we only selected one target object per image (i.e., each object and image in VG is chosen at most once).}

\subsection{Sampling of Instances}
\label{ssec:sampling}
 % of frequent classes/names in  \vgenome, which, at the same time, have been frequently/commonly studied in the psycholinguistic literature. 
%Criteria: From CV: select images depicting objects with relatively frequent names; From CogSci: select objects which have been frequently studied in cognitive science/psychological norming studies; we chose McRae et al. as basis.
We selected images from seven domains: six based on \newcite{mcrae2005semantic}'s \citeyear{mcrae2005semantic} feature norms, a dataset widely used in Psycholinguistics that consists of common objects of different categories (e.g.,~\textsc{animals}, \textsc{furniture}), and \textsc{person}, because it is a very frequent category in \vg.
% We start from the concepts of McRae et al.'s feature norms (REF), which are common objects of different categories (e.g.,~\textsc{animals}, \textsc{furniture}) and, as such, have a high overlap with standard datasets of object norming studies (REFS).
% We added the \textsc{person} category because it is very frequent category in \vgenome.

Within each domain, we aimed at collecting instances at different levels in a taxonomy to cover a wide range of phenomena, but this is not straightforward because ontological taxonomies do not align well with the lexicon (for instance, \textit{dog} and \textit{cow} are both mammals, but \textit{dog} has many more common subcategories), and most domains are not organized in a clear taxonomy %in the first place 
(e.g.\ \textsc{home}).
% Standard taxonomies do not align well with name variability; for instance, people use more varied names for types of dogs than for types of cows, while both dogs and cows are mammals.
Instead, we defined a set of synsets ($52$\ in total) that we would use to collect our object instances from \vg, balancing variability.
From the set of synsets that match or subsume the concepts in the McRae norms, we kept those that had a high number of \vg object instances of different names.
For example, \vg instances subsumed by McRae's \textsl{dog} were named \textsl{beagle, greyhound, puppy, bulldog}, etc., while McRae's \textsl{duck}, \textsl{goose}, or \textsl{gull} did not have name variants in \vg, so we kept \textsl{dog} and \textsl{bird} (which subsumes \textsl{duck}, \textsl{goose}, or \textsl{gull}) as collection synsets.

We then retrieved all VG images depicting an object whose name matches or is subsumed by words in one of these synsets; we refer to these words as \textit{seeds}, and we had 450 of them.
We did not consider objects with names in plural form, with parts-of-speech other than nouns\footnote{(REF to tagger)}, or that were multi-word expressions (e.g.,~\textsl{pink bird}). 
We further only considered objects whose bounding box covered an area of~$20-90\%$ of the image.
% We based the definition of our set of nodes on the WN (REF) synsets of the McRae concepts (e.g.,~dog, duck, goose, gull), the nominal WordNet hierarchy, and the frequency distribution of the VG object names' synsets.\footnote{TODO: need to be clear from the general description of VG that the frequ. of instances labeled with the synset of the object name is meant.} 
% First, we selected a set of collection node candidates---synsets which match (e.g.,~\textsl{dog, duck, goose, gull}) or subsume (e.g.,~\textsl{mammal, bird}) the McRae synsets\footnote{Specific synset IDs, e.g.,~dog.n.01, are omitted for readability.}. 
% From these candidates we kept those as collection nodes which had a high frequency of VG object instances of different names. For example, VG instances  subsumed by McRae's \textsl{dog} were named \textsl{beagle, greyhound, puppy, bulldog}, etc., while McRae's \textsl{duck, goose}, or \textsl{gull} did not have name variants in VG, so we kept \textsl{dog} and \textsl{bird} as collection nodes.
%\paragraph{Collection of instance candidates}
% Goal of above procedure was the collection of instances of selected object classes---our nodes--- whose VG names correspond to or subsume (are hypernyms of) a McRae concept, and whose object names differ, that is, of which we can expect that people possess different names for them (e.g.,~\textsl{duck, goose, gull} for \textsl{bird}).
% \paragraph{Sampling of instances}
Because of the Zipfian distribution of names, and to balance the collection, we sampled instances depending on the size of the seeds: up to $500$\ instances for seeds with up to $800$\ objects, and up to $1000$\ instances for larger seeds. \textbf{double-check}
This yielded a dataset with $31,093$~instances, which was further pruned during annotation (see Section\ \ref{subsec:elicitation}). 
Table~\ref{tab:overview_dataset1} shows the $7$\ domains together with the top $10$\ \vg names.


\cs{@Carina ToDo: add some examples of synsets + names; maybe in suppl}
% \begin{itemize}
% \item rows: domains (if we go for long paper: then one row per collection node?)
% \item columns:
%   \begin{enumerate}
%   \item \# collection nodes
%   \item collection nodes (list)
%   \item \# unique VG names
%   \item example VG names
%   \item \# unique objects
%   \item \# unique images (? not sure if necessary; maybe only one of unique {objects, images})
%   \end{enumerate}
% \end{itemize}

\subsection{Elicitation Procedure}
\label{ssec:elicitation}
To elicit object names, we set up a crowdsourcing task on Amazon Mechanical Turk (AMT).
In initial pilot studies, we found object identification via bounding boxes to be problematic.
In some cases, the bounding box was not clear; in others, AMT workers named objects that were more salient than the one signaled by the bounding box (for instance, for a box around a jacket, the man wearing it).
We took special care of minimizing this issue, in two ways: Specifying the instructions such that workers pay close attention to what object is being indicated in the box, and pruning the set of images via an initial collection round (9 workers per task, i.e.,\ 9 names/object) in which we allowed workers to indicate whether the object was occluded or the box unclear.
The Appendix~\ref{app:instructions} contains the task instructions and details about the pruning and data procedure.
We eliminated around 5.5K images based on pruning, obtaining the final dataset with 25,596\ images.
We then did 3 more collection rounds, and shuffled the set of images per task between each round. 
Workers could only participate in one round, to avoid workers annotating an instance more than once. 
We obtained a total of 36 names per image (i.e.,\ objects).
As will become clear in the analysis, while object identification remains an issue despite these precautions, most mismatches between \vg annotations  and our data cannot be considered mistakes, and the annotation results have significant implications for the study of object naming in language \& vision (see discussion in Sections~\ref{sec:analysis} and \ref{sec:modeling}).
\gbt{TO DO: revise that this matches the narrative in the rest of the paper.}
Overall $841$\ workers took part in the data elicitation, with a median of  $261$\ instances \mbox{($\textrm{range}=[9,17K]$)} per worker.
\cs{Maybe say something about the rejections, if space permits it.}
%%% Local Variables:
%%% mode: latex
%%% TeX-master: "main"
%%% End:

	
\section{Analysis}
\label{sec:analysis}
\subsection{Agreement}

We compute the following agreement measures:

\begin{itemize}
\item \textbf{\% top}: for each object, we calculate the relative frequency of the most common name, and then average over all objects
\item \textbf{SN}: for each object, we calculate the Snodgrass agreement measure, and then average over all objects \gbt{Note: changing SD to SN cause SD is typically standard deviation}
\item \textbf{=VG}: the proportion of objects where the most frequent name coincides with the name annotated in VisualGenome
\end{itemize}


\begin{table*}
\small
\begin{tabular}{llll|llll|llll}
\toprule
    & \multicolumn{3}{c|}{all synsets} & \multicolumn{4}{c|}{max synset} & \multicolumn{4}{c}{min synset} \\
                         domain & \% top &    SN &   =VG &         id &     \% top &    SN &   =VG &             id &     \% top &    SN &   =VG \\
\midrule
         people &  0.52 &  2.13 &  0.50 &  professional.n.01 &  0.61 &  2.02 &  0.20 &           athlete.n.01 &  0.36 &  2.62 &  0.37 \\
       clothing &  0.64 &  1.58 &  0.70 &      neckwear.n.01 &  0.79 &  0.91 &  0.77 &          footwear.n.01 &  0.47 &  2.55 &  0.40 \\
           home &  0.66 &  1.50 &  0.78 &          tool.n.01 &  0.86 &  0.73 &  0.94 &          crockery.n.01 &  0.52 &  1.92 &  0.40 \\
      buildings &  0.67 &  1.55 &  0.73 &        bridge.n.01 &  0.75 &  1.21 &  0.87 &  place\_of\_worship.n.01 &  0.46 &  2.26 &  0.08 \\
           food &  0.71 &  1.30 &  0.63 &  edible\_fruit.n.01 &  0.80 &  0.89 &  0.79 &         vegetable.n.01 &  0.53 &  1.97 &  0.15 \\
       vehicles &  0.72 &  1.13 &  0.71 &         train.n.01 &  0.93 &  0.42 &  0.99 &          aircraft.n.01 &  0.52 &  1.50 &  0.41 \\
 animals,plants &  0.91 &  0.44 &  0.94 &        feline.n.01 &  0.95 &  0.29 &  0.99 &              fish.n.01 &  0.39 &  2.53 &  0.55 \\
\bottomrule
 all &  0.70 &  1.34 &  0.73            \\

\bottomrule
\end{tabular}
\caption{Agreement in object names for objects of different domains, if applicable, synsets with maximal and minimal agreement (top \%) are shown }
\label{tab:agree}
\end{table*}

Table \ref{tab:agree} shows that, overall, our annotators achieve a fair amount of agreement in the object naming choices. The domain where annotators agree most is the animal domain, which, interestingly, happens to be the domain that has been mostly discussed in the object naming literature. \sz{... much more to say}

Why is naming more flexible in certain domains than in others? \gbt{Hypothesis: expectation: little variation - hypernymy at most, more variation <-> more affordances <-> more varied relationships.}

\subsection{Lexical relations}

In this section, we take a closer look at the lexical variation we observe in our data set. We analyze the data points where participants attributed different names to the same object and extract a set of  pairwise \textbf{naming variants}. These naming variants correspond to pairs of words that can be used interchangeably to name certain objects.
For each object, we extract the set of naming variants $s = \{ (w_{top},w_2), (w_{top},w_3), (w_{top},w_4),... \}$  where $w_{top}$ is the most frequent name annotated for the object and $w_2 ... w_n$ constitute the less frequent alternatives of $w_{top}$.  The  \textbf{type frequency} of a naming variant $(w_{top},w_x)$ corresponds to the number of objects where this variant occurs. The \textbf{token frequency} of $(w_{top},w_x)$ corresponds the count of all annotations where $w_x$ has been used instead of $w_{top}$.
In Table \ref{tab:exvariants}, we show the the naming variants with the highest raw token frequency for each domain. 

The naming variants can be grouped according to their lexical relation, as follows:

\begin{itemize}
\item \textbf{synonymy}: e.g.\ aircraft vs. airplane 
\item \textbf{hyponymy}: e.g.\ man vs. person
\item \textbf{co-hyponymy}: e.g.\ swan vs. goose
\item \textbf{no relation}: e.g.\  desk vs. apple
\end{itemize}


\begin{table}

\begin{tabular}{lll}
\toprule
       relation &  types & tokens \\
\midrule
synonymy &  0.01 &  0.09 \\
co-hyponymy &  0.03 &  0.07 \\
hypernymy &  0.06 &  0.35 \\
not-covered &  0.19 &  0.04 \\
crossclassified &  0.70 &  0.47 \\
\bottomrule
\end{tabular}
% \small
% \begin{tabular}{llll}
% \toprule
%         relation & \% types & \% tokens & av. depth \\
% \midrule
%  co-hyponymy (closure, max depth=10) &  0.889 &  0.551 &       3.479 \\
%     hyponymy (closure, max depth=10) &  0.097 &  0.328 &       2.204 \\
%         synonymy &  0.015 &  0.121 &       1.000 \\
% \bottomrule
% \end{tabular}
\caption{Lexical relations between naming variants according to WordNet, for the set of name pairs where both words can be found in WordNet and stand in a \sz{should we produce this table for the different domains?} \gbt{yes, please. Maybe do rows domains, columns lexical relations (synymymy, hyponymy, co-hyponymy, other, not in wordnet), with subcolumns for types and tokens? And do percentages over rows -- for each domain, how many of the variants we find fall into each of the classes. This way we'll be able to see differences across domains.}}
\label{tab:rel}
\end{table}

\begin{table*}
\small
\begin{tabular}{lp{13cm}}
\toprule
                         category &                                                                                                                                                                                                                    most frequent naming variants \\

\midrule
 people &  woman -- person (3594), man -- person (3546), boy -- child (3243), woman -- girl (2328), girl -- child (1985), woman -- tennis player (1277), man -- player (1273), man -- boy (1214), skateboarder -- skater (1194), man -- t-shirt (1143) \\
 food &  pizza -- food (1883), sandwich -- food (1123), hotdog -- food (540), pizza -- cheese (457), pizza -- plate (430), salad -- food (402), sandwich -- burger (398), hotdog -- sandwich (351), sandwich -- bread (318), cake -- food (286) \\
 home &  couch -- sofa (4090), desk -- table (3448), carpet -- floor (1697), bench -- chair (1401), desk -- keyboard (1380), counter -- table (1201), table -- desk (1135), counter -- countertop (1101), table -- counter (906), rug -- carpet (895) \\
 buildings &  house -- building (1160), building -- house (511), bridge -- train (326), bridge -- overpass (235), house -- window (161), house -- home (123), tent -- canopy (120), building -- castle (101), bridge -- building (98), bridge -- pole (85) \\
 vehicles &  airplane -- plane (11194), plane -- airplane (3829), motorcycle -- bike (2624), airplane -- jet (1319), boat -- ship (1301), truck -- car (1095), car -- vehicle (874), motorcycle -- wheel (861), truck -- vehicle (718), truck -- wheel (716) \\
 clothing &  shirt -- t-shirt (2914), jacket -- coat (2396), jacket -- shirt (1552), jacket -- suit (1168), suit -- jacket (1029), shirt -- jacket (813), shirt -- tie (723), shirt -- man (487), shirt -- dress (462), shirt -- sweater (450) \\
 animals\_plants &  cow -- bull (515), sheep -- goat (486), cow -- animal (445), giraffe -- animal (380), bird -- parrot (349), sheep -- animal (294), sheep -- lamb (282), horse -- animal (269), cat -- animal (237), bird -- seagull (231) \\
\bottomrule
\end{tabular}\caption{Most frequent naming variants for each category}
\label{tab:exvariants}
\end{table*}

Research on object naming following the idea of entry-level categories has, essentially, exclusively looked at names that stand in a hierarchical relation (i.e.\ hyponymy/hypernymy).

We use WordNet to extract lexical relations between the naming variants in our data set.
Unfortunately, this means that we have to exclude a certain portion of the data as either (i) one of the name is not covered in WordNet, (ii) we cannot find a lexical relation between the two names (see below). Also, we had to be relatively permissive with respect to the definition of hyponymy/co-hyponymy. 
For instance, to analyze \textit{giraffe} as a hyponym of \textit{animal} we have to look at the closure of the hyponyms of \textit{animal} with a depth of 8 (in WordNet).
\sz{should we call this co-hyponymy or co-hierarchical relation?}

\sz{include Table that reports counts of the naming variants, coverage in WordNet etc.} \gbt{I think it'd be best to put the out-of-wordnet info in the Lexical relations table -- this way we have everything in one place.}

Table \ref{tab:rel} shows the distribution of lexical relations for those naming variants that we were able to analyze with WordNet.
Both in terms of their types and token frequency, the naming variants that instantiate a (loose) co-hyponymy relation are by far the most frequent.
\sz{discuss in more detail, discuss: to what extent is this an artefact of WordNet?}
This is really interesting: most research on object naming, to date, has focussed on hyponymy/hypernymy, i.e. variation that relates to hierarchical relations between object names.
Our data suggests that co-hierarchical variation is really important too.

\subsection{Beyond synonymy and hypernymy: cross-classification}

\gbt{@Sina, for this section/my analysis, I'd need csvs (or some code that builds dataframes) containing the following (see commented text):}

  % \begin{enumerate}
  % \item Dataframe for images: for each image,
  %   \begin{itemize}
  %   \item top -- top name
  %   \item \% top
  %   \item SN -- snodgrass agreement
  %   \item =VG
  %   \item dictionary or similar with names and counts
  %   \end{itemize}
  % \item Dataframe for name variants: for each variant pair,
  %   \begin{itemize}
  %   \item top name
  %   \item variant
  %   \item frequency of the pair
  %   \item domain of top
  %   \item domain of variant (one of ours or ``OTHER'' if not in any of them?)
  %   \item relationship in wordnet, one of: synonymy, hyponymy, co-hyponymy, other, not in wordnet (agree? the latter means ``one of the elements not in wordnet'')
  %   \end{itemize}
  % \end{enumerate}

Some (interesting, somewhat cherry-picked) word pairs were WordNet does not find any relation (excluded in the above analysis):

\begin{itemize}
\item lettuce -- salad
\item fruit -- food
\item man -- catcher
\item bowl --chili
\item bowl -- diner \gbt{spelling mistake? should be dinner?} 
\item burger -- meat
\item statue -- animal (image shows statue of an animal)
\item bottle -- alcohol
\item donut --desert \gbt{spelling mistake? should be dessert?} 
\item zebra -- stripes
\item oven -- grill
\end{itemize}

\sz{discuss...}


% \subsection{Entry-level names and preference orders....}

% \sz{an interesting example:} In our data set, there are 24 images where \textit{penguin} has been used, so we know that the object is a \textit{penguin}. For 50\% of these images, annotators still prefer \textit{bird} as the most common name. According to the theory of entry-level categories, this should not happen. People should always prefer \textit{penguin} over \textit{bird}. 

% \sz{how can we analyze this quantitatively?}

%%% Local Variables:
%%% mode: latex
%%% TeX-master: "main"
%%% End:


\section{Conclusion}
\label{sec:conc}
% To sum up, we find both substantial consistency in naming (the most frequent name accounts for 75\% of object names, on average) and substantial naming variation (the remaining 25\%). Moreover, there is a very large standard variation in how much agreement there is for objects in images, that only partially depend on the domain.

The question of how people choose names for objects presented visually is relevant for Language and Vision, Computational Linguistics, Computer Vision, Cognitive Science, and Linguistics.
We have surveyed datasets that can be useful to address this question, and proposed a new dataset, ManyNames, that affords new possibilities both for analysis and modeling of object naming.
% gbt: commenting the following out cause it's a bit dangerous to say this without the verification phase.
% , providing a means
% (i) to study how different people would name the same object (image region), and, given a specific name, estimate how likely people are to use it for a given object, which might vary substantially across different instances of a given category, and
% (ii) to obtain a set of possible names for individual objects, as well as available lexical alternatives for specific names, which again might vary strongly across instances and often cannot be retrieved from existing taxonomies like WordNet. 

For Computer Vision and L\&V, our data highlights the fact that bounding boxes are often ambiguous, which can affect model performance on object categorization and naming.
Crucially, evaluations in these tasks assume that object identification is possible based on the bounding box. 
Our data suggests that this is not always the case and provides empirical data to, for instance, assess whether model mistakes are plausible (similar to those of humans, as in the \word{toy/book/bed} case) or really off.
% (as in the case where they label objects as the background, like \word{grass} for \word{dog}).

Moreover, standard evaluations assume that object names (or categories) are unique.
% The naming variation we have found also challenge the common assumption that there exists a single canonical name for individual visual objects (also see \cite{deng2014large,wang2014poodle,peterson2018learning}).
The ability to distinguish incorrect object names from good alternatives is essential for visual object understanding.
Our data provides a first step towards enabling model evaluation on naming variants of an instance, checking, e.g.,\ to what extent the top N predicted names are valid alternatives (\word{dog, animal, pet}) or not (\word{dog, hat, grass}).
However, to fully enable this sort of analysis, a further annotation step is needed, to account for the referential uncertainty of bounding boxes and annotation noise.
We plan to take this step in future work, which will also enable more robust conclusions with respect to naming variation.

% Wang et al.: ". The hypothesis is that by modeling the variation in granularity levels for different concepts, we can gain a more informative insight as to how the output of image annotation systems can relate to how a person describes what he or she perceives in an image, and consequently produce image annotation systems that are more human-centric."
% @related work
%[For example, \newcite{peterson2018learning} train CNN classifiers on objects with multiple labels which stand in a hierarchical relation (e.g., dog, animal) in order to learn better visual representations which capture the hierarchical structure of a taxonomy. \cs{remove or move to related work? also sentence to Ordonez}]
%\footnote{Other work used training data with multiple labels per image to improve image classification performance on images with multiple objects (e.g.,~Wang et al., 2016 REF). \cs{maybe remove, since it is not that relevant?}}
% peterson2018learning: discuss the bias introduced into learned representations by training on data of  single label annotations ("labels cut arbitrarily across natural psychological taxonomies, e.g., dogs are separated into breeds but never categorized as dogs"). 

% However, given the fact that naming variants are often not recoverable by hierarchical relations, a taxonomic hierarchy is only limited in its use to distinguish automatically a truly false prediction (e.g.,\ \textsl{plate}) from a (possibly context-specific) valid alternative (e.g.,\ \textsl{basket}) to the single "ground truth" in a dataset (e.g.,\ \textsl{sandwich}). 
% For the same reason, even fine-grained recognition models (as those trained on ILSVRC) cannot be expected to be able to simply infer the recognition of more general classes.
%since we found that many name alternatives are not hierarchically related to the VG name, there is only limited use of, e.g., a taxonomic hierarchy, to distinguish automatically a "truly" false prediction (e.g.,\ \textsl{plate}) from a (possibly context-specific) valid alternative (e.g.,\ \textsl{basket}) to the single "ground truth" in a dataset (e.g.,\ \textsl{sandwich}). 

Our current data clearly supports the prediction in theoretical research on object naming and categorization that there will in general be a preferred name for a given visually presented object.
It also tentatively suggests that (a) there is also consistent variation in naming, with an average of almost three elicited names per instance; (b) much of this variation cannot be explained by adopting a traditional taxonomy-driven and hierarchical view, which has been dominant in the literature, both in theoretical and in computational approaches; (c) there is high variability in agreement across instances within the same domain.
The latter suggests that there are specific visual characteristics of either the object itself or the visual context in which it appears that trigger variation. With prototypical, idealized pictures of the sort used in traditional studies (see Figure\ \ref{fig:cake}), this observation would not be possible.

We hope that ManyNames triggers more empirical research on object naming, a topic that has been understudied in both computational and theoretical approaches to language.

%%% Local Variables:
%%% mode: latex
%%% TeX-master: "lrec2020naming"
%%% End:


% \section{Acknowledgements}

% Place all acknowledgements (including those concerning research grants and
% funding) in a separate section at the end of the article.

\section{Bibliographical References}
\label{main:ref}

\bibliographystyle{lrec}
\bibliography{naming}


%\section{Language Resource References}
%\label{lr:ref}
%\bibliographystylelanguageresource{lrec}
%\bibliographylanguageresource{lrec2020W-xample}

\end{document}

%%% Local Variables:
%%% mode: latex
%%% TeX-master: t
%%% End:
