% The object naming variation attested in the ManyNames dataset offers new challenges and perspectives, both for practical modeling approaches in Language \& Vision as well as for theoretical work in (Psycho-)linguistics.

The answer to our question, \textit{do objects in real-world images have a canonical name?} is nuanced. 
Our data indeed indicates a tendency for particular instances marked by bounding boxes to elicit a preferred name across subjects, which is in line with existing theoretical research on object naming and categorization. 
However, we show that there is also consistent variation in naming, and even high variability in agreement across instances within the same class or domain. 
The latter suggests that there are specific visual characteristics of either the object itself or the visual context in which it appears that trigger variation. With prototypical pictures (see Figure\ \ref{fig:cake}), as done in traditional studies, this observation would not be possible.
%For theoretical research, w
We moreover find that much of the variation in object naming cannot be explained by adopting a traditional taxonomy-driven and hierarchical view, which has been dominant in the literature.

Our ManyNames dataset provides a means 
(i) to study how different people would name the same object (image region),  
%\cs{"visual awareness" would fit in here}, 
and, given a specific name, estimate its degree of preference, which might vary substantially across different instances of a class, and   
%assess the extent to which different speakers prefer that name, which might vary substantially across different instances of a class, and to 
(ii) to obtain a set of possible names for individual objects, as well as available lexical alternatives for specific names, which again might vary strongly across instances and often cannot be retrieved from existing taxonomies like WordNet. 

As far as research in L\&V and Computer Vision is concerned, our findings are reason to question 
%the appropriateness of the standard, single-label approach in visual object naming and categorization methods. 
 the common assumption that there at least exists a canonical name
%,\ i.e.,\ a  name on which the majority of people would agree 
 for individual visual objects (localized by bounding boxes).
 % cf.\ the high standard deviation in terms of the relative frequency of an instance's top response (column \%top, Table~\ref{tab:agree}), and the average mismatch of $27\%$~between the instances' top response and their VG name (\mbox{top=VG})\cs{a bit redundant now}).
%
The ability to distinguish incorrect object names from name alternatives is essential for visual object understanding, though. 
It would be desirable to explicitly assess models towards their ability to account for naming variants of an instance (e.g.,~in how far are the top N predicted names valid alternatives, such as, \textsl{dog, animal, pet} vs. \textsl{dog, animal, hat}). 
%
The criticism on the use of single "ground truth" labels is not new, but %to our knowledge 
as we outlined in Section\ \ref{sec:relwork}, previous work has focused on the determination of canonical names, or on "granularity-aware" models, where naming variants are hierarchically related. 
%\cs{check that this is in related work: previous work has focused on the determination of canonical names (e.g.,~\newcite{Ordonez:2016}; Mathews et al REF), or on "granularity-aware" models, where naming variants are hierachically related (e.g.,~\newcite{wang2014poodle, peterson2018learning}; Ristin et al., 2015 REF, and the references therein).}
%
% \textit{training} of classifiers with multiple labels to improve image classification model depicting multiple objects (e.g.,~Wang et al., 2016 REF)
% Wang et al.: ". The hypothesis is that by modeling the variation in granularity levels for different concepts, we can gain a more informative insight as to how the output of image annotation systems can relate to how a person describes what he or she perceives in an image, and consequently produce image annotation systems that are more human-centric."
% @related work
%[For example, \newcite{peterson2018learning} train CNN classifiers on objects with multiple labels which stand in a hierarchical relation (e.g., dog, animal) in order to learn better visual representations which capture the hierarchical structure of a taxonomy. \cs{remove or move to related work? also sentence to Ordonez}]
%\footnote{Other work used training data with multiple labels per image to improve image classification performance on images with multiple objects (e.g.,~Wang et al., 2016 REF). \cs{maybe remove, since it is not that relevant?}}
%
%
% peterson2018learning: discuss the bias introduced into learned representations by training on data of  single label annotations ("labels cut arbitrarily across natural psychological taxonomies, e.g., dogs are separated into breeds but never categorized as dogs"). 
%
%
However, given the fact that naming variants are often not recoverable by hierarchical relations, a taxonomic hierarchy is only limited in its use to distinguish automatically a truly false prediction (e.g.,\ \textsl{plate}) from a (possibly context-specific) valid alternative (e.g.,\ \textsl{basket}) to the single "ground truth" in a dataset (e.g.,\ \textsl{sandwich}). 
For the same reason, even fine-grained recognition models (as those trained on ILSVRC) cannot be expected to be able to simply infer the recognition of more general classes.
%since we found that many name alternatives are not hierarchically related to the VG name, there is only limited use of, e.g., a taxonomic hierarchy, to distinguish automatically a "truly" false prediction (e.g.,\ \textsl{plate}) from a (possibly context-specific) valid alternative (e.g.,\ \textsl{basket}) to the single "ground truth" in a dataset (e.g.,\ \textsl{sandwich}). 
%\cs{also distributional similarity? Remember that we looked into this for the checkpoint and did not find a clear similarity threshold/pattern} 
%

%\cs{?: Say something about type disagreements on class-level, after having written the results discussion.}
%

%\cs{Ah, we still should have a comparison model with above argumentation :/.}
%
%As a use-case study, in Section\ \ref{sec:modeling}, we will treat ManyNames as evaluation data and assess whether an existing SOTA model can account for the naming variants of individual objects. 

%%% Local Variables:
%%% mode: latex
%%% TeX-master: "lrec2020naming"
%%% End:
