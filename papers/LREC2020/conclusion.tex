% To sum up, we find both substantial consistency in naming (the most frequent name accounts for 75\% of object names, on average) and substantial naming variation (the remaining 25\%). Moreover, there is a very large standard variation in how much agreement there is for objects in images, that only partially depend on the domain.

The question of how people choose names for objects presented visually is relevant for Language and Vision, Computational Linguistics, Computer Vision, Cognitive Science, and Linguistics.
We have surveyed datasets that can be useful to address this question, and proposed a new dataset, ManyNames, that affords new possibilities both for analysis and modeling of object naming.
% gbt: commenting the following out cause it's a bit dangerous to say this without the verification phase.
% , providing a means
% (i) to study how different people would name the same object (image region), and, given a specific name, estimate how likely people are to use it for a given object, which might vary substantially across different instances of a given category, and
% (ii) to obtain a set of possible names for individual objects, as well as available lexical alternatives for specific names, which again might vary strongly across instances and often cannot be retrieved from existing taxonomies like WordNet. 

For Computer Vision and L\&V, our data highlights the fact that bounding boxes are often ambiguous, which can affect model performance on object categorization and naming.
Crucially, evaluations in these tasks assume that object identification is possible based on the bounding box. 
Our data suggests that this is not always the case and provides empirical data to for instance asses whether model mistakes are plausible (similar to those of humans, as in the \word{toy/book/bed} case) or really off.
% (as in the case where they label objects as the background, like \word{grass} for \word{dog}).

Moreover, standard evaluations assume that object names (or categories) are unique.
% The naming variation we have found also challenge the common assumption that there exists a single canonical name for individual visual objects (also see \cite{deng2014large,wang2014poodle,peterson2018learning}).
The ability to distinguish incorrect object names from good alternatives is essential for visual object understanding.
Our data provides a first step towards enabling model evaluation on naming variants of an instance, checking for instance to what extent the top N predicted are names valid alternatives (\textsl{dog, animal, pet}) or not (\textsl{dog, hat, grass}).
However, to fully enable this sort of analysis, a further annotation step is needed, to account for the referential uncertainty of bounding boxes and annotation noise.
We plan to take this step in future work, which will also enable more robust conclusions with respect to naming variation.

% Wang et al.: ". The hypothesis is that by modeling the variation in granularity levels for different concepts, we can gain a more informative insight as to how the output of image annotation systems can relate to how a person describes what he or she perceives in an image, and consequently produce image annotation systems that are more human-centric."
% @related work
%[For example, \newcite{peterson2018learning} train CNN classifiers on objects with multiple labels which stand in a hierarchical relation (e.g., dog, animal) in order to learn better visual representations which capture the hierarchical structure of a taxonomy. \cs{remove or move to related work? also sentence to Ordonez}]
%\footnote{Other work used training data with multiple labels per image to improve image classification performance on images with multiple objects (e.g.,~Wang et al., 2016 REF). \cs{maybe remove, since it is not that relevant?}}
% peterson2018learning: discuss the bias introduced into learned representations by training on data of  single label annotations ("labels cut arbitrarily across natural psychological taxonomies, e.g., dogs are separated into breeds but never categorized as dogs"). 

% However, given the fact that naming variants are often not recoverable by hierarchical relations, a taxonomic hierarchy is only limited in its use to distinguish automatically a truly false prediction (e.g.,\ \textsl{plate}) from a (possibly context-specific) valid alternative (e.g.,\ \textsl{basket}) to the single "ground truth" in a dataset (e.g.,\ \textsl{sandwich}). 
% For the same reason, even fine-grained recognition models (as those trained on ILSVRC) cannot be expected to be able to simply infer the recognition of more general classes.
%since we found that many name alternatives are not hierarchically related to the VG name, there is only limited use of, e.g., a taxonomic hierarchy, to distinguish automatically a "truly" false prediction (e.g.,\ \textsl{plate}) from a (possibly context-specific) valid alternative (e.g.,\ \textsl{basket}) to the single "ground truth" in a dataset (e.g.,\ \textsl{sandwich}). 

Our current data clearly supports the prediction in theoretical research on object naming and categorization that there will in general be a preferred name for a given visually presented object.
It also tentatively suggests that (a) there is also consistent variation in naming, with an average of almost three elicited names per instance; (b) much of this variation cannot be explained by adopting a traditional taxonomy-driven and hierarchical view, which has been dominant in the literature, both in theoretical and in computational approaches; (c) there is high variability in agreement across instances within the same domain.
The latter suggests that there are specific visual characteristics of either the object itself or the visual context in which it appears that trigger variation. With prototypical, idealized pictures of the sort used in traditional studies (see Figure\ \ref{fig:cake}), this observation would not be possible.

We hope that ManyNames triggers more empirical research on object naming, a topic that has been understudied in both computational and theoretical approaches to language.

%%% Local Variables:
%%% mode: latex
%%% TeX-master: "lrec2020naming"
%%% End:
