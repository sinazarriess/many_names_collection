\subsection{Object Naming as a Linguistic Phenomenon}
\label{subsec:rosch}

The act of naming an object amounts to that of picking out a nominal to be employed to refer to it (e.g., ``the \refexp{dog}'', ``the white \refexp{dog} to the left'').
Since an object is simultaneously a member of multiple categories (e.g., a young beagle belongs to the categories \cat{dog}, \cat{beagle}, \cat{animal}, \cat{puppy}, \cat{pet}, etc.), all the various names that lexicalize these constitute a valid alternative, meaning that the same object can be called by different names \cite{brown1958shall,murphy2004big}.

Seminal work by \newcite{rosch1976basic} inspired a taxonomic view of object naming, in which names exhibit a preferred level of specificity or abstraction called the ``entry-level'' \cite{jolicoeur1984pictures}. 
This typically corresponds to an intermediate level of specificity (basic level,\ e.g., \refexp{bird}, \refexp{car}), as opposed to more generic (super-ordinate,\ e.g., \refexp{animal}, \refexp{vehicle}) or more specific categories (sub-ordinate,\ e.g., \refexp{sparrow}, \refexp{convertible}).
However, less prototypical members of basic level categories tend to be instead identified with sub-ordinate categories (e.g., a penguin is typically called \refexp{penguin} and not \refexp{bird}; \newcite{jolicoeur1984pictures}). 

%This out-of-context preference towards a certain taxonomic level is often referred to as \textbf{lexical availability}. 
While the traditional notion of entry-level categories suggests that objects tend to be named by a \refexp{single} preferred concept, research on pragmatics has found that speakers adopt their naming choices to the context and, hence, are flexible with respect to the chosen level of specificity \cite{olson1970language,rohde2012communicating,graf2016animal}.
%Scenarios where multiple objects (of the same category) are present induce a pressure for generating names which uniquely identify the target \cite{olson1970language}, such that sub-level names can be systematically elicited in these cases \cite{rohde2012communicating,graf2016animal}.
For example, in presence of more than one dog, the name \textsl{dog} is ambiguous and a sub-ordinate category (e.g., \textsl{rottweiler}, \textsl{beagle}) is potentially preferred by speakers. The effect of such distractor objects on the production of referring expressions has been widely examined in the language generation community \cite{krahmer:2012}, though not specifically for object naming. We believe that our new dataset provides an interesting resource for tackling this question.
%, though additional factors such as cost or saliency also come into play \cite{graf2016animal,clark1983common}.

The purely taxonomic view on naming has also been criticized in work on object organization, which found that many objects of our daily lifes are part of multiple category systems at the same time \cite{ross1999food,SHAFTO20111}. 
This \textit{cross-classification} occurs, for instance, with food categories which can be taxonomy-based (e.g.\ \refexp{meat, vegetable}) or script-based (e.g.\  \refexp{breakfast, snack}).
We provide tentative evidence that cross-classification is indeed relevant for naming variation, and that the taxonomic axis is not the most frequent source of variation in our data.

\subsection{Picture Naming in Cognitive Science}

An important experimental paradigm in work on human vision and categorization is picture naming, where subjects have to say or write down the first name that comes to mind when looking at a picture of (typically) a line drawing depicting a prototypical instance of a category \cite{snodgrass,rossion2004revisiting}, see Figure\ \ref{fig:cake}.
Subjects reach very high agreement in this task \cite{rossion2004revisiting}, i.e. for a given object, there is a clear tendency towards a certain name across all speakers.
The resulting naming norms are useful for studying various cognitive processes \cite{humphreys1988cascade}.
Our task is inspired by picture naming, but uses real-world images showing objects in context.

\subsection{Object Recognition in Computer Vision}

In Computer Vision, object recognition is often modeled as a classification task where state-of-the-art systems localize and classify objects into thousands of different categories  \cite{googlenet,ILSVRC15}. 
Current recognition benchmarks use labels and images from the ImageNet \cite{imagenet_cvpr09} ontology, and typically assume a single ground-truth label. 
The construction of ImageNet was set up as a two-stage procedure: (i) images for given categories in the ontology were automatically collected by querying search engines, (ii) crowd-workers then verified whether each candidate image is an instance of the given category.
Other data collection efforts for object labels also used a predefined vocabulary and asked annotators to mark all instances of these categories in a set of images \cite{mscoco,OpenImages}. 
Recently, \newcite{pont2019natural} have argued for annotation of object labels using free form text though here this free vocabulary is then mapped to a set of underlying classes.
Thus, even though object recognition benchmarks do provide images of objects and categories, they generally do not provide what we are interested in in this work, namely natural names of objects.

\subsection{Object Naming in L\&V} 

Previous work in L\&V has collected and used data sets where annotators produced free and natural utterances for a given image. 
Moreover, these data sets typically record utterances that are more complex than a single word, such as image captions \cite{fangetal:2015,devlin:imcaqui,Bernardietal:automatic}, referring expressions \cite{Kazemzadeh2014,mao15,Yu2016}, visual dialogues \cite{das2017visual,vries2017guesswhat} or image paragraphs \cite{krause2017hierarchical}. While object names occur in all of these data sets, they are not necessarily marked up and linked to the corresponding image regions. The overview in Section \ref{sec:survey} will discuss corpora where the grounding of names to regions for objects is given, as in the case of \vgenome \cite{krishna2016visualgenome}, or where it can be easily derived, as in the case of referring expressions.

Our new collection, ManyNames, focusses on object names in isolation and is substantially more controlled than common L\&V data sets. This controlled collection procedure allowed us to elicit many annotations for the same object from different annotators, resulting in a data set that is amenable to studying variation and preferences in naming systematically and on a large scale.

\begin{table*}[htb]
  \centering
  \begin{tabular}{lrrrrr}
    \toprule
    &   RefCOCO/+  &  Flickr30kE &           VG &      VGmn &        MN \\
    \midrule
    \# objects & 50,000 & 243,801 & 3,781,232 & 25,315 & 25,315 \\
    naming vocab size &  5,004 &  10,423 &   105,441 &  1,061 &  7,970 \\
    av. annotations/object &      2.8 &       2.3 &         1.7 &      7.2 &     35.3 \\
    ratio of objects with n types $>$ 1 &      0.7 &       0.3 &         0.02 &      0.05 &      0.9 \\
    av. types/object &      1.9 &       1.4 &         1 &      1.1 &      5.7 \\
    \bottomrule
  \end{tabular}
  \caption{Overview statistics for different data sets containing object naming data. VGmn shows statistics for the subset of \vg that overlaps with our ManyNames dataset.\label{tab:compare}}
\end{table*}


%Work that models which \textbf{word} (as opposed to a category label) a speaker will use to name an object is relatively scarce.
%Natural language generation has intensively investigated referring expressions~\cite{dale:1995,krahmer:2012}; however, this area has focused mostly on the selection of attributes, typically assuming that the name is given.
%\gbt{Add example}
%\newcite{Ordonez:2016} takes up the notion of entry-level categories and transfers an object's predicted label to its name.
%Their model classifies objects into fine-grained categories (\gbt{e.g., ++++++ }), and then predicts a WordNet synset to retrieve the name (e.g., \word{swan}), based on frequencies in a text corpus.
%This work assumes that \gbt{+++++complete please, stating what's different to ours}
 %\newcite{zarriess-schlangen:2017} learn a naming model on referring expressions and real-world images, but focus on combining visual and distributional information. 
%\gbt{I don't understand how this differs from other research and our own.}
%Recent experimental work on reference found that the specificity of a name is dependent on the taxonomic relatedness of other objects in context
%\cite{rohde2012communicating,graf2016animal}. 
%However, this work studies a very limited set of images \gbt{+++++complete please, stating what's different to ours}
%Our work is a first step towards studying naming in real-world, natural reference.
%As there is virtually no existing large-scale resource that provides robust naming data elicited from multiple subjects \textit{and} for instances in real-world images, this paper focuses on naming in isolation, rather than reference where naming interacts with attribute selection.
%
%
%
%This line of research has emphasized the taxonomic organization of categories, following the seminal work on prototypes by \newcite{rosch1976basic} mentioned above.
%These works propose granularity-aware object recognition methods, that incorporate the taxonomic structure underlying object labels in multi-label settings; for the ``young beagle'' example, labels \word{beagle}, \word{dog}, \word{animal} would all be considered valid.
%Instead, other sources of variation like cross-classification have not received attention in L\&V.
%As mentioned above, our results suggest that cross-classification occurs very frequently when naming objects in real-world images.


%%% Local Variables:
%%% mode: latex
%%% TeX-master: "lrec2020naming"
%%% End:
