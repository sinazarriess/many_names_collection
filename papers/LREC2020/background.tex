\subsection{Object Naming as a Linguistic Phenomenon}

The act of naming an object amounts to that of picking out a nominal to be employed to refer to it (e.g., ``the \refexp{dog}'', ``the white \refexp{dog} to the left'').
Since an object is simultaneously a member of multiple categories (e.g., a young \refexp{beagle} is at once a \cat{dog}, a \cat{beagle}, an \cat{animal}, a \cat{puppy} etc.), all the various names that lexicalize these constitute a valid alternative, meaning that the same object can be named with more or less \textbf{specific names} \cite{brown1958shall,murphy2004big}. 
Seminal work on concepts by Rosch suggests that object names typically exhibit a preferred level of specificity called the \textbf{entry-level}. This typically corresponds to an intermediate level of specificity, i.e., \textbf{basic level} (e.g, \refexp{bird}, \refexp{car}) \cite{rosch1976basic}, as opposed to more generic (i.e., \textbf{super-level}; e.g., \refexp{animal}, \refexp{vehicle}) or specific categories (i.e., \textbf{sub-level}; e.g., \refexp{sparrow}, \refexp{convertible}). However, less prototypical members of basic-level categories tend to be instead identified with sub-level categories (e.g., a \cat{penguin} is typically called a \refexp{penguin} and not a \refexp{bird}) \cite{jolicoeur1984pictures}. 
%This out-of-context preference towards a certain taxonomic level is often referred to as \textbf{lexical availability}. 
While the traditional notion of entry-level categories suggests that objects tend to be named by a \refexp{single} preferred concept, research on pragmatics has found that speakers are flexible in  
%with respect to the chose level of specificity. 
their choice of the level of specificity. 
Scenarios where multiple objects (of the same category) are present induce a pressure for generating names which uniquely identify the target \cite{olson1970language}, such that sub-level names can be systematically elicited in these cases %\cite{rohde2012communicating} \cite{graf2016animal}. 
\cite{rohde2012communicating}\cite{graf2016animal}.
For example, in presence of more than one dog, the name \textsl{dog} is ambiguous and a sub-level category (e.g., \textsl{rottweiler}, \textsl{beagle}) is more informative and potentially preferred by speakers, though additional factors such as cost or saliency also come into play \cite{graf2016animal}\cite{clark1983common}.

\subsection{Visual Object Recognition}

This line of research studies and models object representations in the human visual system, (cf.\ \newcite{regan2000human,rossion2004revisiting}). 
An important experimental paradigm here is picture naming, where subjects have to say or write down the first name that comes to mind when looking at a picture of (typically) a line drawing depicting a prototypical instance of a category \cite{snodgrass}, see Figure\ \ref{fig:cake}.
Subjects reach very high agreement in this task \cite{rossion2004revisiting}, and the resulting naming norms are useful for studying various cognitive processes \cite{humphreys1988cascade}.
Our task is inspired by picture naming, but uses real-world images with objects highlighted in them.
Recognition of instances (as opposed to categories) in images has also been the focus of Computer Vision, where state-of-the-art systems are now able to predict thousands of different categories, (e.g.\ \newcite{googlenet}). 
Current recognition benchmarks use labels (and images) from the ImageNet \cite{imagenet_cvpr09} taxonomy, but typically implement a single ground-truth label approach. 
%, but typically implement multi-label classifiers where relations between labels are not considered \cite{ILSVRC15}. 
%\cs{do you mean multi-label classifiers in single ground-truth label settings?}
In L\&V, deep object recognition systems are widely used for feature extraction, whereas the object labeling itself can often not be used directly. For instance, many labels in the ILSVRC  challenge \cite{ILSVRC15} correspond to very specific breeds of animals, whereas other common categories  for,\ e.g.,\ people are missing.

\subsection{Hierarchical Object Categorization} 
%goes beyond isolated object categories.
% but looks at the principles underlying the organization of object categories. 
... This line of research has emphasized the taxonomic organization of categories, e.g.\ seminal work on prototypes by \newcite{rosch1976basic},  and found that humans tend to conceptualize objects at a basic or medium level of abstraction.
Granularity-aware object recognition methods have incorporated the taxonomic structure underlying object labels in multi-label settings \cite{deng2014large,wang2014poodle,peterson2018learning}, as an account to the criticism on the use of single-labels. 
%While this work goes beyond the simplistic modeling assumption that categories are just unrelated labels, 
%but still aim to predict a single canonical category. % (that does have relations to other categories). 
The purely taxonomic view has been criticized in more recent work on concept organization, which found that many objects of our daily lifes are part of multiple category systems at the same time \cite{ross1999food,SHAFTO20111}. 
This \textit{cross-classification} occurs, for instance, with food categories which can be taxonomy-based (e.g.\  \refexp{meat, vegetable}) or script-based (e.g.\  \refexp{breakfast, snack}).
To the best of our knowledge, this phenomenon has not received any attention in L\&V.
Our results, however, suggest that cross-classification occurs very frequently when naming objects in real-world images.


\subsection{Object Naming in L\&V} 

... work that models which \textit{word} (i.e.\ not label) a speaker will use to name an object is relatively scarce.
Though names are prominent in referring expressions, investigated a lot in natural language generation \cite{dale:1995}, this area has focused mostly on the selection of attributes % rather than on determining the referent's name 
\cite{krahmer:2012}. 
\newcite{Ordonez:2016} takes up the notion of entry-level categories \cite{rosch1976basic} and transfers an object's predicted fine-grained label to its name using text corpus statistics.
% extend object recognition to naming, taking up the notion of entry-level categories \cite{rosch1976basic}.
%Their model predicts classifies objects into fine-grained categories and then predicts a WordNet synset for retrieving the name, based on frequencies in a text corpus. 
 \newcite{zarriess-schlangen:2017} learn a naming model on referring expressions and real-world images, but focus on combining visual and distributional information. 
 Recent experimental work on reference found that the specificity of a name is dependent on the taxonomic relatedness of other objects in context
\cite{rohde2012communicating,graf2016animal}. Our work is a first step towards studying naming in real-world, natural reference.
But as there is virtually no existing large-scale resource that provides robust naming data elicited from multiple subjects \textit{and} for instances in real-world images, this paper focuses on naming in isolation, rather than reference where naming interacts with attribute selection.

%%% Local Variables:
%%% mode: latex
%%% TeX-master: "lrec2020naming"
%%% End:
