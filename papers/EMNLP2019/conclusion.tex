The object naming variation attested in the ManyNames dataset offers new challenges and perspectives, both for practical modeling approaches in Language \& Vision as well as for theoretical work in (Psycho-)linguistics.
For modeling, it provides data on possible names for objects that constitutes a more robust testing ground.
In future work, we want to use the ManyNames data to fine-tune existing models such that their behavior is more human-like, and check if models trained on the ManyNames data perform better than models trained on single labels. \gbt{@Carina, pls check if this is what you meant}

For theoretical work, it provides name variation data with natural images and thus at the \textit{instance} level, as opposed to the category level that has been the focus up to now in theoretical work, due to the use of highly stylized drawings.
This affords tremendous opportunities to study factors affecting object naming that have not been studied to date, such as prototypicality of the instance itself, or the effect of the visual context.
However, it also comes with its own challenges, the main one being the issue of object identification, that we have found to interfere with the study of true naming variation.
Bounding boxes are not always clear enough pointers to objects, and at the same time there is no clear line between true mistakes (\textit{batter-helmet}), referential disagreements (\textit{bed-sleeping bag} when the two overlap in the image), and valid alternatives (\textit{carpet-floor} can be a metonymy).
This of course is not an issue when working with the kind of images that traditional studies in Psycholinguistics have used, and future work will need to address it.
\gbt{Do you think this is too negative as a conclusion? I feel like it's the elephant in the room -- if we don't say it here, we should say it somewhere prominently.}

%%% Local Variables:
%%% mode: latex
%%% TeX-master: "main"
%%% End:
