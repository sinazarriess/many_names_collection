The object naming variation attested in the ManyNames dataset offers new challenges and perspectives, both for practical modeling approaches in Language \& Vision as well as for theoretical work in (Psycho-)linguistics.
For modeling, it provides data on possible names for objects that constitutes a more robust testing ground.
In future work, we want to use ManyNames to fine-tune and examine  existing models in a multi-label setting towards a behavior that is more human-like. 

For theoretical work, it provides name variation data with natural images and thus at the \textit{instance} level, as opposed to the category level that has been the focus up to now in theoretical work, due to the use of highly stylized drawings.
This affords tremendous opportunities to study factors affecting object naming that have not been studied to date, such as the prototypicality of the instance itself, or the effect of the visual context.

%However, it also comes with its own challenges, the main one being the issue of object identification, that we have found to interfere with the study of true naming variation. 
The use of naturalistic images in place of stylized drawings also comes with new challenges, which interfere with the study of naming variation: the factors inherently underlying variation cannot be easily separated from those introduced by issues such as object identification.
Bounding boxes are not always clear enough pointers to objects, and at the same time there is no clear line between true mistakes (\textit{batter-helmet}), referential disagreements (\textit{bed-sleeping bag}, when the two visually overlap in the image), and valid alternatives (\textit{carpet-floor} can be a metonymy).
Future work will need to address these challenges. 

%%% Local Variables:
%%% mode: latex
%%% TeX-master: "main"
%%% End:
