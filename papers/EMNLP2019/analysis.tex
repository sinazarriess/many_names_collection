\subsection{Agreement}

We compute the following agreement measures:

\begin{itemize}
\item \textbf{\% top}: for each object, we calculate the relative frequency of the most common name, and then average over all objects
\item \textbf{SN}: for each object, we calculate the Snodgrass agreement measure, and then average over all objects \gbt{Note: changing SD to SN cause SD is typically standard deviation}
\item \textbf{=VG}: the proportion of objects where the most frequent name coincides with the name annotated in VisualGenome
\end{itemize}


\begin{table*}
\small
\begin{tabular}{llll|llll|llll}
\toprule
    & \multicolumn{3}{c|}{all synsets} & \multicolumn{4}{c|}{max synset} & \multicolumn{4}{c}{min synset} \\
                         domain & \% top &    SN &   =VG &         id &     \% top &    SN &   =VG &             id &     \% top &    SN &   =VG \\
\midrule
         people &  0.52 &  2.13 &  0.50 &  professional.n.01 &  0.61 &  2.02 &  0.20 &           athlete.n.01 &  0.36 &  2.62 &  0.37 \\
       clothing &  0.64 &  1.58 &  0.70 &      neckwear.n.01 &  0.79 &  0.91 &  0.77 &          footwear.n.01 &  0.47 &  2.55 &  0.40 \\
           home &  0.66 &  1.50 &  0.78 &          tool.n.01 &  0.86 &  0.73 &  0.94 &          crockery.n.01 &  0.52 &  1.92 &  0.40 \\
      buildings &  0.67 &  1.55 &  0.73 &        bridge.n.01 &  0.75 &  1.21 &  0.87 &  place\_of\_worship.n.01 &  0.46 &  2.26 &  0.08 \\
           food &  0.71 &  1.30 &  0.63 &  edible\_fruit.n.01 &  0.80 &  0.89 &  0.79 &         vegetable.n.01 &  0.53 &  1.97 &  0.15 \\
       vehicles &  0.72 &  1.13 &  0.71 &         train.n.01 &  0.93 &  0.42 &  0.99 &          aircraft.n.01 &  0.52 &  1.50 &  0.41 \\
 animals,plants &  0.91 &  0.44 &  0.94 &        feline.n.01 &  0.95 &  0.29 &  0.99 &              fish.n.01 &  0.39 &  2.53 &  0.55 \\
\bottomrule
 all &  0.70 &  1.34 &  0.73            \\

\bottomrule
\end{tabular}
\caption{Agreement in object names for objects of different domains, if applicable, synsets with maximal and minimal agreement (top \%) are shown }
\label{tab:agree}
\end{table*}

Table \ref{tab:agree} shows that, overall, our annotators achieve a fair amount of agreement in the object naming choices. The domain where annotators agree most is the animal domain, which, interestingly, happens to be the domain that has been mostly discussed in the object naming literature. \sz{... much more to say}

Why is naming more flexible in certain domains than in others? \gbt{Hypothesis: expectation: little variation - hypernymy at most, more variation <-> more affordances <-> more varied relationships.}

\subsection{Lexical relations}

In this section, we take a closer look at the lexical variation we observe in our data set. We analyze the data points where participants attributed different names to the same object and extract a set of  pairwise \textbf{naming variants}. These naming variants correspond to pairs of words that can be used interchangeably to name certain objects.
For each object, we extract the set of naming variants $s = \{ (w_{top},w_2), (w_{top},w_3), (w_{top},w_4),... \}$  where $w_{top}$ is the most frequent name annotated for the object and $w_2 ... w_n$ constitute the less frequent alternatives of $w_{top}$.  The  \textbf{type frequency} of a naming variant $(w_{top},w_x)$ corresponds to the number of objects where this variant occurs. The \textbf{token frequency} of $(w_{top},w_x)$ corresponds the count of all annotations where $w_x$ has been used instead of $w_{top}$.
In Table \ref{tab:exvariants}, we show the the naming variants with the highest raw token frequency for each domain. 

The naming variants can be grouped according to their lexical relation, as follows:

\begin{itemize}
\item \textbf{synonymy}: e.g.\ aircraft vs. airplane 
\item \textbf{hyponymy}: e.g.\ man vs. person
\item \textbf{co-hyponymy}: e.g.\ swan vs. goose
\item \textbf{no relation}: e.g.\  desk vs. apple
\end{itemize}


\begin{table}

\begin{tabular}{lll}
\toprule
       relation &  types & tokens \\
\midrule
synonymy &  0.01 &  0.09 \\
co-hyponymy &  0.03 &  0.07 \\
hypernymy &  0.06 &  0.35 \\
not-covered &  0.19 &  0.04 \\
crossclassified &  0.70 &  0.47 \\
\bottomrule
\end{tabular}
% \small
% \begin{tabular}{llll}
% \toprule
%         relation & \% types & \% tokens & av. depth \\
% \midrule
%  co-hyponymy (closure, max depth=10) &  0.889 &  0.551 &       3.479 \\
%     hyponymy (closure, max depth=10) &  0.097 &  0.328 &       2.204 \\
%         synonymy &  0.015 &  0.121 &       1.000 \\
% \bottomrule
% \end{tabular}
\caption{Lexical relations between naming variants according to WordNet, for the set of name pairs where both words can be found in WordNet and stand in a \sz{should we produce this table for the different domains?} \gbt{yes, please. Maybe do rows domains, columns lexical relations (synymymy, hyponymy, co-hyponymy, other, not in wordnet), with subcolumns for types and tokens? And do percentages over rows -- for each domain, how many of the variants we find fall into each of the classes. This way we'll be able to see differences across domains.}}
\label{tab:rel}
\end{table}

\begin{table*}
\small
\begin{tabular}{lp{13cm}}
\toprule
                         category &                                                                                                                                                                                                                    most frequent naming variants \\

\midrule
 people &  woman -- person (3594), man -- person (3546), boy -- child (3243), woman -- girl (2328), girl -- child (1985), woman -- tennis player (1277), man -- player (1273), man -- boy (1214), skateboarder -- skater (1194), man -- t-shirt (1143) \\
 food &  pizza -- food (1883), sandwich -- food (1123), hotdog -- food (540), pizza -- cheese (457), pizza -- plate (430), salad -- food (402), sandwich -- burger (398), hotdog -- sandwich (351), sandwich -- bread (318), cake -- food (286) \\
 home &  couch -- sofa (4090), desk -- table (3448), carpet -- floor (1697), bench -- chair (1401), desk -- keyboard (1380), counter -- table (1201), table -- desk (1135), counter -- countertop (1101), table -- counter (906), rug -- carpet (895) \\
 buildings &  house -- building (1160), building -- house (511), bridge -- train (326), bridge -- overpass (235), house -- window (161), house -- home (123), tent -- canopy (120), building -- castle (101), bridge -- building (98), bridge -- pole (85) \\
 vehicles &  airplane -- plane (11194), plane -- airplane (3829), motorcycle -- bike (2624), airplane -- jet (1319), boat -- ship (1301), truck -- car (1095), car -- vehicle (874), motorcycle -- wheel (861), truck -- vehicle (718), truck -- wheel (716) \\
 clothing &  shirt -- t-shirt (2914), jacket -- coat (2396), jacket -- shirt (1552), jacket -- suit (1168), suit -- jacket (1029), shirt -- jacket (813), shirt -- tie (723), shirt -- man (487), shirt -- dress (462), shirt -- sweater (450) \\
 animals\_plants &  cow -- bull (515), sheep -- goat (486), cow -- animal (445), giraffe -- animal (380), bird -- parrot (349), sheep -- animal (294), sheep -- lamb (282), horse -- animal (269), cat -- animal (237), bird -- seagull (231) \\
\bottomrule
\end{tabular}\caption{Most frequent naming variants for each category}
\label{tab:exvariants}
\end{table*}

Research on object naming following the idea of entry-level categories has, essentially, exclusively looked at names that stand in a hierarchical relation (i.e.\ hyponymy/hypernymy).

We use WordNet to extract lexical relations between the naming variants in our data set.
Unfortunately, this means that we have to exclude a certain portion of the data as either (i) one of the name is not covered in WordNet, (ii) we cannot find a lexical relation between the two names (see below). Also, we had to be relatively permissive with respect to the definition of hyponymy/co-hyponymy. 
For instance, to analyze \textit{giraffe} as a hyponym of \textit{animal} we have to look at the closure of the hyponyms of \textit{animal} with a depth of 8 (in WordNet).
\sz{should we call this co-hyponymy or co-hierarchical relation?}

\sz{include Table that reports counts of the naming variants, coverage in WordNet etc.} \gbt{I think it'd be best to put the out-of-wordnet info in the Lexical relations table -- this way we have everything in one place.}

Table \ref{tab:rel} shows the distribution of lexical relations for those naming variants that we were able to analyze with WordNet.
Both in terms of their types and token frequency, the naming variants that instantiate a (loose) co-hyponymy relation are by far the most frequent.
\sz{discuss in more detail, discuss: to what extent is this an artefact of WordNet?}
This is really interesting: most research on object naming, to date, has focussed on hyponymy/hypernymy, i.e. variation that relates to hierarchical relations between object names.
Our data suggests that co-hierarchical variation is really important too.

\subsection{Beyond synonymy and hypernymy: cross-classification}

\gbt{@Sina, for this section/my analysis, I'd need csvs (or some code that builds dataframes) containing the following (see commented text):}

  % \begin{enumerate}
  % \item Dataframe for images: for each image,
  %   \begin{itemize}
  %   \item top -- top name
  %   \item \% top
  %   \item SN -- snodgrass agreement
  %   \item =VG
  %   \item dictionary or similar with names and counts
  %   \end{itemize}
  % \item Dataframe for name variants: for each variant pair,
  %   \begin{itemize}
  %   \item top name
  %   \item variant
  %   \item frequency of the pair
  %   \item domain of top
  %   \item domain of variant (one of ours or ``OTHER'' if not in any of them?)
  %   \item relationship in wordnet, one of: synonymy, hyponymy, co-hyponymy, other, not in wordnet (agree? the latter means ``one of the elements not in wordnet'')
  %   \end{itemize}
  % \end{enumerate}

Some (interesting, somewhat cherry-picked) word pairs were WordNet does not find any relation (excluded in the above analysis):

\begin{itemize}
\item lettuce -- salad
\item fruit -- food
\item man -- catcher
\item bowl --chili
\item bowl -- diner \gbt{spelling mistake? should be dinner?} 
\item burger -- meat
\item statue -- animal (image shows statue of an animal)
\item bottle -- alcohol
\item donut --desert \gbt{spelling mistake? should be dessert?} 
\item zebra -- stripes
\item oven -- grill
\end{itemize}

\sz{discuss...}


% \subsection{Entry-level names and preference orders....}

% \sz{an interesting example:} In our data set, there are 24 images where \textit{penguin} has been used, so we know that the object is a \textit{penguin}. For 50\% of these images, annotators still prefer \textit{bird} as the most common name. According to the theory of entry-level categories, this should not happen. People should always prefer \textit{penguin} over \textit{bird}. 

% \sz{how can we analyze this quantitatively?}

%%% Local Variables:
%%% mode: latex
%%% TeX-master: "main"
%%% End:
