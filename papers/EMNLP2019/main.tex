%
% File emnlp2019.tex
%
%% Based on the style files for ACL 2019, which were
%% Based on the style files for EMNLP 2018, which were
%% Based on the style files for ACL 2018, which were
%% Based on the style files for ACL-2015, with some improvements
%%  taken from the NAACL-2016 style
%% Based on the style files for ACL-2014, which were, in turn,
%% based on ACL-2013, ACL-2012, ACL-2011, ACL-2010, ACL-IJCNLP-2009,
%% EACL-2009, IJCNLP-2008...
%% Based on the style files for EACL 2006 by 
%%e.agirre@ehu.es or Sergi.Balari@uab.es
%% and that of ACL 08 by Joakim Nivre and Noah Smith

\documentclass[11pt,a4paper]{article}
\usepackage[hyperref]{emnlp-ijcnlp-2019}
\usepackage{times}
\usepackage{latexsym}
\usepackage{graphicx}  %%% for including graphics
\usepackage{booktabs}
\usepackage{xspace}

\usepackage{url}

%\aclfinalcopy % Uncomment this line for the final submission

%\setlength\titlebox{5cm}
% You can expand the titlebox if you need extra space
% to show all the authors. Please do not make the titlebox
% smaller than 5cm (the original size); we will check this
% in the camera-ready version and ask you to change it back.

\newcommand\BibTeX{B{\sc ib}\TeX}
\newcommand\confname{EMNLP-IJCNLP 2019}
\newcommand\conforg{SIGDAT}
\newcommand{\refexp}[1]{\textsl{#1}}
\newcommand{\word}[1]{\textsl{#1}}
\newcommand{\cat}[1]{\textsc{#1}}
\newcommand{\vgenome}{VisualGenome\xspace}
\newcommand{\ra}{$\rightarrow$}

\newcommand{\sz}[1]{\textcolor{blue}{\emph{//sz: #1//}}}
\newcommand{\gbt}[1]{\textcolor{orange}{\emph{//g: #1//}}}
\newcommand{\cs}[1]{\textcolor{green}{\emph{//cs: #1//}}}

\title{Do Objects in Real-World Images Have a Canonical Name?} 

\author{First Author \\
  Affiliation / Address line 1 \\
  Affiliation / Address line 2 \\
  Affiliation / Address line 3 \\
  {\tt email@domain} \\\And
  Second Author \\
  Affiliation / Address line 1 \\
  Affiliation / Address line 2 \\
  Affiliation / Address line 3 \\
  {\tt email@domain} \\}

\date{}

\begin{document}
\maketitle

\begin{abstract}
Research in language \& vision commonly acknowledges the fact that speakers can interpret and describe visual scenes in many different ways. For instance, frameworks in image captioning  typically elicit (during collection) or train and test on (during modeling) multiple valid alternatives. Individual visual objects and their names, however, are often collected and modeled in simple annotation set-ups where annotators or systems identify objects via bounding boxes and label them with a single ``correct'' name. A prime example here is VisualGenome that features complex and varied annotations on the level of scenes and regions, but provides canonicalized objects linked to a single name. In this paper, we put the assumption that objects in complex visual scenes bear a canonical name to an empirical test. Based on pre-annotated object bounding boxes in VisualGenome, we elicit multiple names from 36 subjects per object and investigate the extent to which there is variation in the names chosen by different people for the same object. Our analysis reveals that non-canonicalized object naming data shows a lot of interesting linguistic variation as speakers name objects on different levels of specificity (cow-animal) or verbalize different aspect of the same object (bowl-salad). At the same time, we find that object naming prompted via bounding boxes is subject to a certain amount of noise as speakers have problems re-identifying the object that was annotated by the original bounding box. We investigate whether a state-of-the-art model of object labeling implicitly encodes similar variation in object naming and discuss implications for research in language \& vision.
\end{abstract}

\section{Introduction}
\input{intro2}

\section{Related Work}
\label{sec:relwork}

Object naming is closely related to processes of visual object recognition investigated in Computer Vision or Cognitive Science.
This Section will discuss how naming relates to questions addressed in these neighbouring areas and how it has been and should be studied as a linguistic phenomenon in Language \& Vision.



\paragraph{Work on Visual Object Recognition} studies and models object representations in the human visual system, cf.\ \cite{regan2000human,rossion2004revisiting}. 
An important paradigm for experimental research here is picture naming, where human subjects are asked to say or write down the first name that comes to mind when looking at a given picture, typically a normed, black-and-white line drawing depicting a prototypical instance of a category on a white background \cite{snodgrass}, see Figure \ref{fig:picture_naming}.
Subjects have been shown to reach very high agreement on this task, e.g.\ \cite{rossion2004revisiting}, and the resulting naming norm data is useful for studying various cognitive processes \cite{humphreys1988cascade}.
Our annotation task is inspired by these picture naming studies, but uses real-world images with particular objects highlighted in them.
Recognition of instances (as opposed to categories) in images has also been the focus of computer vision, where state-of-the-art systems are now able to classify of real-world objects into thousands of different categories, e.g.\  \newcite{googlenet}. Object recognition benchmarks use object labels (and images) from the ImageNet \cite{imagenet_cvpr09} taxonomy, but typically frame the task as a multi-label classification problem where taxonomic relations between labels are not considered \cite{ILSVRC15}. 
In L\&V,  deep object recognition systems are widely used for feature extraction, whereas the object labeling itself can often not be used directly. For instance, many labels in the ILSVRC15 challenge correspond to very specific breeds of animals, whereas other common categories  for e.g. people are missing.


\paragraph{Work on Hierarchical Object Categorization} is not only interested in visual object representations but also in principles underlying the organization of these object categories. 
For instance, seminal work on prototypes by Rosch \cite{rosch1976basic} has emphasized the hierarchical organization of categories and found that humans tend to conceptualize objects at a basic or medium level of abstraction in this taxonomy. 
%This suggests that object names typically exhibit a preferred level of specificity, which \citet{jolicoeur1984pictures} called the \textbf{entry-level}, e.g, \refexp{bird}, \refexp{car}), as opposed to more generic (e.g., \refexp{animal}, \refexp{vehicle}) or specific categories (e.g., \refexp{sparrow}, \refexp{convertible}).Less prototypical members of basic-level categories have been found to be identified with sub-level categories (e.g., a \cat{penguin} is typically called a \refexp{penguin} and not a \refexp{bird}) \cite{jolicoeur1984pictures}. 
Hierarchical or "granularity-aware" approaches to visual object recognition have aimed at exploiting the rich structure underlying object labels \cite{Deng:2012,frome2013,deng2014large,wang2014poodle,peterson2018learning}. 
While this work goes beyond the simplistic modeling assumption that categories are just unrelated labels, the main objective here is still to predict a single correct or canonical category (that does have relations to other categories). 
This view has been criticized by more recent work on concept organization, which found that many objects that we frequently interact with in our daily lifes are part of multiple taxonomies at the same time \cite{ross1999food,SHAFTO20111}. A prominent example for this \textit{cross-classification} are food categories which can be taxonomy-based (e.g.\  \refexp{meat, vegetable}) or script-based (e.g.\  \refexp{breakfast, snack}).
To the best of our knowledge, this phenomenon has not received any attention in recent corpus-based work on L\&V.
Our results, however, suggest that cross-classification occurs very frequently when naming objects in real-world images.
 
%Nevertheless, the way the treat object recognition is conceptually very simple (if not to say, naive):  standard object classification schemes are inherently ``flat'', and treat object labels as mutually exclusive \cite{deng2014large}, ignoring all kinds of linguistic relations between these labels and ignoring the fact that an object can easily be an instance of several categories.\cs{I would make this statement stronger and argue that \textbf{object recognition is merely a labeling of objects  with human interpretable symbols}, and that a system would probably fail if it had to decide whether an object labeled as, e.g.\ \refexp{fig} may also be labeled as \refexp{food}.} \gbt{ok}


%have  frameworks previous work has focused on the determination of canonical names (e.g.,~\newcite{Ordonez:2016}; Mathews et al REF), or on "granularity-aware" models, where naming variants are hierachically related (e.g.,~\newcite{wang2014poodle, peterson2018learning}; 
%Ristin et al., 2015 REF, and the references therein). 

% \textit{training} of classifiers with multiple labels to improve image classification model depicting multiple objects (e.g.,~Wang et al., 2016 REF)
% Wang et al.: ". The hypothesis is that by modeling the variation in granularity levels for different concepts, we can gain a more informative insight as to how the output of image annotation systems can relate to how a person describes what he or she perceives in an image, and consequently produce image annotation systems that are more human-centric."
%[For example, \newcite{peterson2018learning} train CNN classifiers on objects with multiple labels which stand in a hierarchical relation (e.g., dog, animal) in order to learn better visual representations which capture the hierarchical structure of a taxonomy. \cs{remove or move to related work? also sentence to Ordonez}

%This out-of-context preference towards a certain taxonomic level is often referred to as \textbf{lexical availability}. 
%While the traditional notion of entry-level categories suggests that objects tend to be named by a \refexp{single} preferred concept, research on pragmatics has found that speakers are flexible in  
%with respect to the chose level of specificity. 
%their choice of the level of specificity. 



\paragraph{Work on Object Naming} that explicitly tries to model which \textit{word} (i.e. not a label or category) a speaker is going to use to name to an object is relatively scarce.
Object names figure prominently in referring expressions that have been investigated a lot in natural language generation \cite{dale:1995,krahmer:2012}, but this area has, to date, focussed mostly on the selection of attributes rather than on determining the referent's name \cite{krahmer:2012,Kazemzadeh2014}. 
\newcite{Ordonez:2016} frame object naming as an extension of object recognition, taking up the notion of basic-level or entry-level categories from \cite{rosch1976basic}.
Their model first classifies objects into very fine-grained categories and then decides which level in the taxonomy (i.e.\ WordNet) is appropriate for retrieving the actual name.
This latter decision, however, is not based on natural referring expression data, but derived from frequencies in a text corpus. 
 \newcite{zarriess-schlangen:2017} learn a model of object naming on a corpus of referring expressions for objects in real-world images, but focus on combining visual and distributional information and on zero-shot learning. 
 Recent experimental work on reference found that the specificity of an object name is dependent on the taxonomic relatedness of other objects present in the visual context
\cite{rohde2012communicating,graf2016animal}. Our work can be seen as a first step towards studying naming in real-world, natural reference.
But as there is virtually no existing large-scale resource that provides robust naming data elicited from multiple subjects \textit{and} for instances in real-world images, this paper focusses on naming in isolation, rather than reference where naming interacts with other decisions like attribute selection.


%Scenarios where multiple objects (of the same category) are present induce a pressure for generating names which uniquely identify the target \cite{olson1970language}, 

% \paragraph{Existing resources and their shortcomings}
% Moreover, existing resources in L\&V hardly provide any consistent taxonomic information on objects and their categories, e.g. object labels are typically quite general as in Flickr30k \cite[e.g.,~\cat{people, animals, bodyparts, clothing}]{plummer2015flickr30kentities} or taxonomically heterogeneous as in MS COCO \cite[e.g.,~\cat{people, baseball glove, bird}]{mscoco}.


%%% Local Variables:
%%% mode: latex
%%% TeX-master: "main"
%%% End:


\section{Data collection}
\label{sec:data}
% Number of images/objects:        25,596\\
% Number of object names:  450\\
% Number of collection nodes (synsets):    52 \\

We take data from \vgenome \cite[\vg henceforth]{krishna2016visualgenome}, which
% aims to provide a full set of descriptions of the scenes which images depict in order to spur complete scene understanding. 
contains a dense region-based labeling of $108k$~images with, inter alia, objects, attributes and relationships,  %object descriptions, attributes, and relationships, as well as question-answer pairs, 
all linked to WordNet synsets \cite{fellbaum1998wordnet}.
\vg is suitable for our purpose of collecting names for a relatively large amount of instances of common objects in
naturalistic images, as it has images of varying complexity, with close-ups as well as complex images with many objects.
As common in Computer Vision, objects are localized as 
%identified via
 bounding boxes (see red boxes in Figure~\ref{fig:cake}).% 
\footnote{We use image and object interchangeably in the following, since we only selected one target object per image (i.e., each object and image in VG is chosen at most once).}

\subsection{Sampling of Instances}
\label{ssec:sampling}
 % of frequent classes/names in  \vgenome, which, at the same time, have been frequently/commonly studied in the psycholinguistic literature. 
%Criteria: From CV: select images depicting objects with relatively frequent names; From CogSci: select objects which have been frequently studied in cognitive science/psychological norming studies; we chose McRae et al. as basis.
We selected images from seven domains: six based on \newcite{mcrae2005semantic}'s \citeyear{mcrae2005semantic} feature norms, a dataset widely used in Psycholinguistics that consists of common objects of different categories (e.g.,~\textsc{animals}, \textsc{furniture}), and \textsc{person}, because it is a very frequent category in \vg.
% We start from the concepts of McRae et al.'s feature norms (REF), which are common objects of different categories (e.g.,~\textsc{animals}, \textsc{furniture}) and, as such, have a high overlap with standard datasets of object norming studies (REFS).
% We added the \textsc{person} category because it is very frequent category in \vgenome.

Within each domain, we aimed at collecting instances at different levels in a taxonomy to cover a wide range of phenomena, but this is not straightforward because ontological taxonomies do not align well with the lexicon (for instance, \textit{dog} and \textit{cow} are both mammals, but \textit{dog} has many more common subcategories), and most domains are not organized in a clear taxonomy %in the first place 
(e.g.\ \textsc{home}).
% Standard taxonomies do not align well with name variability; for instance, people use more varied names for types of dogs than for types of cows, while both dogs and cows are mammals.
Instead, we defined a set of synsets ($52$\ in total) that we would use to collect our object instances from \vg, balancing variability.
From the set of synsets that match or subsume the concepts in the McRae norms, we kept those that had a high number of \vg object instances of different names.
For example, \vg instances subsumed by McRae's \textsl{dog} were named \textsl{beagle, greyhound, puppy, bulldog}, etc., while McRae's \textsl{duck}, \textsl{goose}, or \textsl{gull} did not have name variants in \vg, so we kept \textsl{dog} and \textsl{bird} (which subsumes \textsl{duck}, \textsl{goose}, or \textsl{gull}) as collection synsets.

We then retrieved all VG images depicting an object whose name matches or is subsumed by words in one of these synsets; we refer to these words as \textit{seeds}, and we had 450 of them.
We did not consider objects with names in plural form, with parts-of-speech other than nouns\footnote{(REF to tagger)}, or that were multi-word expressions (e.g.,~\textsl{pink bird}). 
We further only considered objects whose bounding box covered an area of~$20-90\%$ of the image.
% We based the definition of our set of nodes on the WN (REF) synsets of the McRae concepts (e.g.,~dog, duck, goose, gull), the nominal WordNet hierarchy, and the frequency distribution of the VG object names' synsets.\footnote{TODO: need to be clear from the general description of VG that the frequ. of instances labeled with the synset of the object name is meant.} 
% First, we selected a set of collection node candidates---synsets which match (e.g.,~\textsl{dog, duck, goose, gull}) or subsume (e.g.,~\textsl{mammal, bird}) the McRae synsets\footnote{Specific synset IDs, e.g.,~dog.n.01, are omitted for readability.}. 
% From these candidates we kept those as collection nodes which had a high frequency of VG object instances of different names. For example, VG instances  subsumed by McRae's \textsl{dog} were named \textsl{beagle, greyhound, puppy, bulldog}, etc., while McRae's \textsl{duck, goose}, or \textsl{gull} did not have name variants in VG, so we kept \textsl{dog} and \textsl{bird} as collection nodes.
%\paragraph{Collection of instance candidates}
% Goal of above procedure was the collection of instances of selected object classes---our nodes--- whose VG names correspond to or subsume (are hypernyms of) a McRae concept, and whose object names differ, that is, of which we can expect that people possess different names for them (e.g.,~\textsl{duck, goose, gull} for \textsl{bird}).
% \paragraph{Sampling of instances}
Because of the Zipfian distribution of names, and to balance the collection, we sampled instances depending on the size of the seeds: up to $500$\ instances for seeds with up to $800$\ objects, and up to $1000$\ instances for larger seeds. \textbf{double-check}
This yielded a dataset with $31,093$~instances, which was further pruned during annotation (see Section\ \ref{subsec:elicitation}). 
Table~\ref{tab:overview_dataset1} shows the $7$\ domains together with the top $10$\ \vg names.


\cs{@Carina ToDo: add some examples of synsets + names; maybe in suppl}
% \begin{itemize}
% \item rows: domains (if we go for long paper: then one row per collection node?)
% \item columns:
%   \begin{enumerate}
%   \item \# collection nodes
%   \item collection nodes (list)
%   \item \# unique VG names
%   \item example VG names
%   \item \# unique objects
%   \item \# unique images (? not sure if necessary; maybe only one of unique {objects, images})
%   \end{enumerate}
% \end{itemize}

\subsection{Elicitation Procedure}
\label{ssec:elicitation}
To elicit object names, we set up a crowdsourcing task on Amazon Mechanical Turk (AMT).
In initial pilot studies, we found object identification via bounding boxes to be problematic.
In some cases, the bounding box was not clear; in others, AMT workers named objects that were more salient than the one signaled by the bounding box (for instance, for a box around a jacket, the man wearing it).
We took special care of minimizing this issue, in two ways: Specifying the instructions such that workers pay close attention to what object is being indicated in the box, and pruning the set of images via an initial collection round (9 workers per task, i.e.,\ 9 names/object) in which we allowed workers to indicate whether the object was occluded or the box unclear.
The Appendix~\ref{app:instructions} contains the task instructions and details about the pruning and data procedure.
We eliminated around 5.5K images based on pruning, obtaining the final dataset with 25,596\ images.
We then did 3 more collection rounds, and shuffled the set of images per task between each round. 
Workers could only participate in one round, to avoid workers annotating an instance more than once. 
We obtained a total of 36 names per image (i.e.,\ objects).
As will become clear in the analysis, while object identification remains an issue despite these precautions, most mismatches between \vg annotations  and our data cannot be considered mistakes, and the annotation results have significant implications for the study of object naming in language \& vision (see discussion in Sections~\ref{sec:analysis} and \ref{sec:modeling}).
\gbt{TO DO: revise that this matches the narrative in the rest of the paper.}
Overall $841$\ workers took part in the data elicitation, with a median of  $261$\ instances \mbox{($\textrm{range}=[9,17K]$)} per worker.
\cs{Maybe say something about the rejections, if space permits it.}
%%% Local Variables:
%%% mode: latex
%%% TeX-master: "main"
%%% End:


\section{Analysis}
\label{sec:analysis}
\subsection{Agreement}

We compute the following agreement measures:

\begin{itemize}
\item \textbf{\% top}: for each object, we calculate the relative frequency of the most common name, and then average over all objects
\item \textbf{SN}: for each object, we calculate the Snodgrass agreement measure, and then average over all objects \gbt{Note: changing SD to SN cause SD is typically standard deviation}
\item \textbf{=VG}: the proportion of objects where the most frequent name coincides with the name annotated in VisualGenome
\end{itemize}


\begin{table*}
\small
\begin{tabular}{llll|llll|llll}
\toprule
    & \multicolumn{3}{c|}{all synsets} & \multicolumn{4}{c|}{max synset} & \multicolumn{4}{c}{min synset} \\
                         domain & \% top &    SN &   =VG &         id &     \% top &    SN &   =VG &             id &     \% top &    SN &   =VG \\
\midrule
         people &  0.52 &  2.13 &  0.50 &  professional.n.01 &  0.61 &  2.02 &  0.20 &           athlete.n.01 &  0.36 &  2.62 &  0.37 \\
       clothing &  0.64 &  1.58 &  0.70 &      neckwear.n.01 &  0.79 &  0.91 &  0.77 &          footwear.n.01 &  0.47 &  2.55 &  0.40 \\
           home &  0.66 &  1.50 &  0.78 &          tool.n.01 &  0.86 &  0.73 &  0.94 &          crockery.n.01 &  0.52 &  1.92 &  0.40 \\
      buildings &  0.67 &  1.55 &  0.73 &        bridge.n.01 &  0.75 &  1.21 &  0.87 &  place\_of\_worship.n.01 &  0.46 &  2.26 &  0.08 \\
           food &  0.71 &  1.30 &  0.63 &  edible\_fruit.n.01 &  0.80 &  0.89 &  0.79 &         vegetable.n.01 &  0.53 &  1.97 &  0.15 \\
       vehicles &  0.72 &  1.13 &  0.71 &         train.n.01 &  0.93 &  0.42 &  0.99 &          aircraft.n.01 &  0.52 &  1.50 &  0.41 \\
 animals,plants &  0.91 &  0.44 &  0.94 &        feline.n.01 &  0.95 &  0.29 &  0.99 &              fish.n.01 &  0.39 &  2.53 &  0.55 \\
\bottomrule
 all &  0.70 &  1.34 &  0.73            \\

\bottomrule
\end{tabular}
\caption{Agreement in object names for objects of different domains, if applicable, synsets with maximal and minimal agreement (top \%) are shown }
\label{tab:agree}
\end{table*}

Table \ref{tab:agree} shows that, overall, our annotators achieve a fair amount of agreement in the object naming choices. The domain where annotators agree most is the animal domain, which, interestingly, happens to be the domain that has been mostly discussed in the object naming literature. \sz{... much more to say}

Why is naming more flexible in certain domains than in others? \gbt{Hypothesis: expectation: little variation - hypernymy at most, more variation <-> more affordances <-> more varied relationships.}

\subsection{Lexical relations}

In this section, we take a closer look at the lexical variation we observe in our data set. We analyze the data points where participants attributed different names to the same object and extract a set of  pairwise \textbf{naming variants}. These naming variants correspond to pairs of words that can be used interchangeably to name certain objects.
For each object, we extract the set of naming variants $s = \{ (w_{top},w_2), (w_{top},w_3), (w_{top},w_4),... \}$  where $w_{top}$ is the most frequent name annotated for the object and $w_2 ... w_n$ constitute the less frequent alternatives of $w_{top}$.  The  \textbf{type frequency} of a naming variant $(w_{top},w_x)$ corresponds to the number of objects where this variant occurs. The \textbf{token frequency} of $(w_{top},w_x)$ corresponds the count of all annotations where $w_x$ has been used instead of $w_{top}$.
In Table \ref{tab:exvariants}, we show the the naming variants with the highest raw token frequency for each domain. 

The naming variants can be grouped according to their lexical relation, as follows:

\begin{itemize}
\item \textbf{synonymy}: e.g.\ aircraft vs. airplane 
\item \textbf{hyponymy}: e.g.\ man vs. person
\item \textbf{co-hyponymy}: e.g.\ swan vs. goose
\item \textbf{no relation}: e.g.\  desk vs. apple
\end{itemize}


\begin{table}

\begin{tabular}{lll}
\toprule
       relation &  types & tokens \\
\midrule
synonymy &  0.01 &  0.09 \\
co-hyponymy &  0.03 &  0.07 \\
hypernymy &  0.06 &  0.35 \\
not-covered &  0.19 &  0.04 \\
crossclassified &  0.70 &  0.47 \\
\bottomrule
\end{tabular}
% \small
% \begin{tabular}{llll}
% \toprule
%         relation & \% types & \% tokens & av. depth \\
% \midrule
%  co-hyponymy (closure, max depth=10) &  0.889 &  0.551 &       3.479 \\
%     hyponymy (closure, max depth=10) &  0.097 &  0.328 &       2.204 \\
%         synonymy &  0.015 &  0.121 &       1.000 \\
% \bottomrule
% \end{tabular}
\caption{Lexical relations between naming variants according to WordNet, for the set of name pairs where both words can be found in WordNet and stand in a \sz{should we produce this table for the different domains?} \gbt{yes, please. Maybe do rows domains, columns lexical relations (synymymy, hyponymy, co-hyponymy, other, not in wordnet), with subcolumns for types and tokens? And do percentages over rows -- for each domain, how many of the variants we find fall into each of the classes. This way we'll be able to see differences across domains.}}
\label{tab:rel}
\end{table}

\begin{table*}
\small
\begin{tabular}{lp{13cm}}
\toprule
                         category &                                                                                                                                                                                                                    most frequent naming variants \\

\midrule
 people &  woman -- person (3594), man -- person (3546), boy -- child (3243), woman -- girl (2328), girl -- child (1985), woman -- tennis player (1277), man -- player (1273), man -- boy (1214), skateboarder -- skater (1194), man -- t-shirt (1143) \\
 food &  pizza -- food (1883), sandwich -- food (1123), hotdog -- food (540), pizza -- cheese (457), pizza -- plate (430), salad -- food (402), sandwich -- burger (398), hotdog -- sandwich (351), sandwich -- bread (318), cake -- food (286) \\
 home &  couch -- sofa (4090), desk -- table (3448), carpet -- floor (1697), bench -- chair (1401), desk -- keyboard (1380), counter -- table (1201), table -- desk (1135), counter -- countertop (1101), table -- counter (906), rug -- carpet (895) \\
 buildings &  house -- building (1160), building -- house (511), bridge -- train (326), bridge -- overpass (235), house -- window (161), house -- home (123), tent -- canopy (120), building -- castle (101), bridge -- building (98), bridge -- pole (85) \\
 vehicles &  airplane -- plane (11194), plane -- airplane (3829), motorcycle -- bike (2624), airplane -- jet (1319), boat -- ship (1301), truck -- car (1095), car -- vehicle (874), motorcycle -- wheel (861), truck -- vehicle (718), truck -- wheel (716) \\
 clothing &  shirt -- t-shirt (2914), jacket -- coat (2396), jacket -- shirt (1552), jacket -- suit (1168), suit -- jacket (1029), shirt -- jacket (813), shirt -- tie (723), shirt -- man (487), shirt -- dress (462), shirt -- sweater (450) \\
 animals\_plants &  cow -- bull (515), sheep -- goat (486), cow -- animal (445), giraffe -- animal (380), bird -- parrot (349), sheep -- animal (294), sheep -- lamb (282), horse -- animal (269), cat -- animal (237), bird -- seagull (231) \\
\bottomrule
\end{tabular}\caption{Most frequent naming variants for each category}
\label{tab:exvariants}
\end{table*}

Research on object naming following the idea of entry-level categories has, essentially, exclusively looked at names that stand in a hierarchical relation (i.e.\ hyponymy/hypernymy).

We use WordNet to extract lexical relations between the naming variants in our data set.
Unfortunately, this means that we have to exclude a certain portion of the data as either (i) one of the name is not covered in WordNet, (ii) we cannot find a lexical relation between the two names (see below). Also, we had to be relatively permissive with respect to the definition of hyponymy/co-hyponymy. 
For instance, to analyze \textit{giraffe} as a hyponym of \textit{animal} we have to look at the closure of the hyponyms of \textit{animal} with a depth of 8 (in WordNet).
\sz{should we call this co-hyponymy or co-hierarchical relation?}

\sz{include Table that reports counts of the naming variants, coverage in WordNet etc.} \gbt{I think it'd be best to put the out-of-wordnet info in the Lexical relations table -- this way we have everything in one place.}

Table \ref{tab:rel} shows the distribution of lexical relations for those naming variants that we were able to analyze with WordNet.
Both in terms of their types and token frequency, the naming variants that instantiate a (loose) co-hyponymy relation are by far the most frequent.
\sz{discuss in more detail, discuss: to what extent is this an artefact of WordNet?}
This is really interesting: most research on object naming, to date, has focussed on hyponymy/hypernymy, i.e. variation that relates to hierarchical relations between object names.
Our data suggests that co-hierarchical variation is really important too.

\subsection{Beyond synonymy and hypernymy: cross-classification}

\gbt{@Sina, for this section/my analysis, I'd need csvs (or some code that builds dataframes) containing the following (see commented text):}

  % \begin{enumerate}
  % \item Dataframe for images: for each image,
  %   \begin{itemize}
  %   \item top -- top name
  %   \item \% top
  %   \item SN -- snodgrass agreement
  %   \item =VG
  %   \item dictionary or similar with names and counts
  %   \end{itemize}
  % \item Dataframe for name variants: for each variant pair,
  %   \begin{itemize}
  %   \item top name
  %   \item variant
  %   \item frequency of the pair
  %   \item domain of top
  %   \item domain of variant (one of ours or ``OTHER'' if not in any of them?)
  %   \item relationship in wordnet, one of: synonymy, hyponymy, co-hyponymy, other, not in wordnet (agree? the latter means ``one of the elements not in wordnet'')
  %   \end{itemize}
  % \end{enumerate}

Some (interesting, somewhat cherry-picked) word pairs were WordNet does not find any relation (excluded in the above analysis):

\begin{itemize}
\item lettuce -- salad
\item fruit -- food
\item man -- catcher
\item bowl --chili
\item bowl -- diner \gbt{spelling mistake? should be dinner?} 
\item burger -- meat
\item statue -- animal (image shows statue of an animal)
\item bottle -- alcohol
\item donut --desert \gbt{spelling mistake? should be dessert?} 
\item zebra -- stripes
\item oven -- grill
\end{itemize}

\sz{discuss...}


% \subsection{Entry-level names and preference orders....}

% \sz{an interesting example:} In our data set, there are 24 images where \textit{penguin} has been used, so we know that the object is a \textit{penguin}. For 50\% of these images, annotators still prefer \textit{bird} as the most common name. According to the theory of entry-level categories, this should not happen. People should always prefer \textit{penguin} over \textit{bird}. 

% \sz{how can we analyze this quantitatively?}

%%% Local Variables:
%%% mode: latex
%%% TeX-master: "main"
%%% End:


\section{Modeling}
\label{sec:modeling}

\section{Conclusions}
\label{sec:conc}

\bibliographystyle{acl_natbib}
\bibliography{naming}

\appendix
\section{Instructions for AMT Experiment}
\label{app:instructions}

\begin{figure*}
	\centering
	\includegraphics[width=2\columnwidth]{figures/round0.png}
	\caption{Instructions for AMT annotators for round~$0$.}
\end{figure*}

\begin{figure*}
	\centering
	\includegraphics[width=2\columnwidth]{figures/round1+_p1.png}
	\includegraphics[width=2\columnwidth]{figures/round1+_p2.png}
	\caption{Instructions for AMT annotators for rounds~$1$ to~$3$.}
\end{figure*}

\end{document}


%%% Local Variables:
%%% mode: latex
%%% TeX-master: t
%%% End:
