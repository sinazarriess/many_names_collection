
Object naming is closely related to processes of visual object recognition investigated in Computer Vision or Cognitive Science.
This Section will discuss how naming relates to questions addressed in these neighbouring areas and how it has been and should be studied as a linguistic phenomenon in Language \& Vision.



\paragraph{Work on Visual Object Recognition} studies and models object representations in the human visual system, cf.\ \cite{regan2000human,rossion2004revisiting}. 
An important paradigm for experimental research here is picture naming, where human subjects are asked to say or write down the first name that comes to mind when looking at a given picture, typically a normed, black-and-white line drawing depicting a prototypical instance of a category on a white background \cite{snodgrass}, see Figure \ref{fig:picture_naming}.
Subjects have been shown to reach very high agreement on this task, e.g.\ \cite{rossion2004revisiting}, and the resulting naming norm data is useful for studying various cognitive processes \cite{humphreys1988cascade}.
Our annotation task is inspired by these picture naming studies, but uses real-world images with particular objects highlighted in them.
Recognition of instances (as opposed to categories) in images has also been the focus of computer vision, where state-of-the-art systems are now able to classify of real-world objects into thousands of different categories, e.g.\  \newcite{googlenet}. Object recognition benchmarks use object labels (and images) from the ImageNet \cite{imagenet_cvpr09} taxonomy, but typically frame the task as a multi-label classification problem where taxonomic relations between labels are not considered \cite{ILSVRC15}. 
In L\&V,  deep object recognition systems are widely used for feature extraction, whereas the object labeling itself can often not be used directly. For instance, many labels in the ILSVRC15 challenge correspond to very specific breeds of animals, whereas other common categories  for e.g. people are missing.


\paragraph{Work on Hierarchical Object Categorization} is not only interested in visual object representations but also in principles underlying the organization of these object categories. 
For instance, seminal work on prototypes by Rosch \cite{rosch1976basic} has emphasized the hierarchical organization of categories and found that humans tend to conceptualize objects at a basic or medium level of abstraction in this taxonomy. 
%This suggests that object names typically exhibit a preferred level of specificity, which \citet{jolicoeur1984pictures} called the \textbf{entry-level}, e.g, \refexp{bird}, \refexp{car}), as opposed to more generic (e.g., \refexp{animal}, \refexp{vehicle}) or specific categories (e.g., \refexp{sparrow}, \refexp{convertible}).Less prototypical members of basic-level categories have been found to be identified with sub-level categories (e.g., a \cat{penguin} is typically called a \refexp{penguin} and not a \refexp{bird}) \cite{jolicoeur1984pictures}. 
Hierarchical or "granularity-aware" approaches to visual object recognition have aimed at exploiting the rich structure underlying object labels \cite{Deng:2012,frome2013,deng2014large,wang2014poodle,peterson2018learning}. 
While this work goes beyond the simplistic modeling assumption that categories are just unrelated labels, the main objective here is still to predict a single correct or canonical category (that does have relations to other categories). 
This view has been criticized by more recent work on concept organization, which found that many objects that we frequently interact with in our daily lifes are part of multiple taxonomies at the same time \cite{ross1999food,SHAFTO20111}. A prominent example for this \textit{cross-classification} are food categories which can be taxonomy-based (e.g.\  \refexp{meat, vegetable}) or script-based (e.g.\  \refexp{breakfast, snack}).
To the best of our knowledge, this phenomenon has not received any attention in recent corpus-based work on L\&V.
Our results, however, suggest that cross-classification occurs very frequently when naming objects in real-world images.
 
%Nevertheless, the way the treat object recognition is conceptually very simple (if not to say, naive):  standard object classification schemes are inherently ``flat'', and treat object labels as mutually exclusive \cite{deng2014large}, ignoring all kinds of linguistic relations between these labels and ignoring the fact that an object can easily be an instance of several categories.\cs{I would make this statement stronger and argue that \textbf{object recognition is merely a labeling of objects  with human interpretable symbols}, and that a system would probably fail if it had to decide whether an object labeled as, e.g.\ \refexp{fig} may also be labeled as \refexp{food}.} \gbt{ok}


%have  frameworks previous work has focused on the determination of canonical names (e.g.,~\newcite{Ordonez:2016}; Mathews et al REF), or on "granularity-aware" models, where naming variants are hierachically related (e.g.,~\newcite{wang2014poodle, peterson2018learning}; 
%Ristin et al., 2015 REF, and the references therein). 

% \textit{training} of classifiers with multiple labels to improve image classification model depicting multiple objects (e.g.,~Wang et al., 2016 REF)
% Wang et al.: ". The hypothesis is that by modeling the variation in granularity levels for different concepts, we can gain a more informative insight as to how the output of image annotation systems can relate to how a person describes what he or she perceives in an image, and consequently produce image annotation systems that are more human-centric."
%[For example, \newcite{peterson2018learning} train CNN classifiers on objects with multiple labels which stand in a hierarchical relation (e.g., dog, animal) in order to learn better visual representations which capture the hierarchical structure of a taxonomy. \cs{remove or move to related work? also sentence to Ordonez}

%This out-of-context preference towards a certain taxonomic level is often referred to as \textbf{lexical availability}. 
%While the traditional notion of entry-level categories suggests that objects tend to be named by a \refexp{single} preferred concept, research on pragmatics has found that speakers are flexible in  
%with respect to the chose level of specificity. 
%their choice of the level of specificity. 



\paragraph{Work on Object Naming} that explicitly tries to model which \textit{word} (i.e. not a label or category) a speaker is going to use to name to an object is relatively scarce.
Object names figure prominently in referring expressions that have been investigated a lot in natural language generation \cite{dale:1995,krahmer:2012}, but this area has, to date, focussed mostly on the selection of attributes rather than on determining the referent's name \cite{krahmer:2012,Kazemzadeh2014}. 
\newcite{Ordonez:2016} frame object naming as an extension of object recognition, taking up the notion of basic-level or entry-level categories from \cite{rosch1976basic}.
Their model first classifies objects into very fine-grained categories and then decides which level in the taxonomy (i.e.\ WordNet) is appropriate for retrieving the actual name.
This latter decision, however, is not based on natural referring expression data, but derived from frequencies in a text corpus. 
 \newcite{zarriess-schlangen:2017} learn a model of object naming on a corpus of referring expressions for objects in real-world images, but focus on combining visual and distributional information and on zero-shot learning. 
 Recent experimental work on reference found that the specificity of an object name is dependent on the taxonomic relatedness of other objects present in the visual context
\cite{rohde2012communicating,graf2016animal}. Our work can be seen as a first step towards studying naming in real-world, natural reference.
But as there is virtually no existing large-scale resource that provides robust naming data elicited from multiple subjects \textit{and} for instances in real-world images, this paper focusses on naming in isolation, rather than reference where naming interacts with other decisions like attribute selection.


%Scenarios where multiple objects (of the same category) are present induce a pressure for generating names which uniquely identify the target \cite{olson1970language}, 

% \paragraph{Existing resources and their shortcomings}
% Moreover, existing resources in L\&V hardly provide any consistent taxonomic information on objects and their categories, e.g. object labels are typically quite general as in Flickr30k \cite[e.g.,~\cat{people, animals, bodyparts, clothing}]{plummer2015flickr30kentities} or taxonomically heterogeneous as in MS COCO \cite[e.g.,~\cat{people, baseball glove, bird}]{mscoco}.


%%% Local Variables:
%%% mode: latex
%%% TeX-master: "main"
%%% End:
