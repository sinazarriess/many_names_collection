
[Sectioning to be revised]

As we outline in the following, our work on natural object naming is of relevance to research in psychology, \langvis and computer vision. 

\paragraph{(Entry-Level) Object Names}

cognitive science/psychology:\\
- levels of specificity\\
- basic-level vs. entry-level\\ 
- object naming studies

Language + Vision:\\
Ordonez et al., Zarriess \& Schlangen, ... \\
- Our work: empirical notion of entry-level

\paragraph{}
In psychology and cognitive science, the study of (entry-level) naming is well-established (CITE). 
Studies have usually used prototypical line drawings of members of different \textit{basic-level} categories (e.g., a duck, a robin, a penguin, etc. are all members of \cat{birds}) 
\cite{jolicoeur1984pictures}, and examined the influence of an object's typicality with respect to its category on the natural choice of its name: highly typical objects (e.g.,\ a robin) have been found to be preferably named by their basic-level category (\name{bird}), while for atypical members a subordinate category (e.g.,\ \name{penguin}) is preferred. 
As such, entry-level names have been considered being an attribute of concrete \textit{concepts} (e.g.,\ penguin). \cs{Need to check Jolicoeur and their experiments; say sth about shared visual features}
% prototypical visual features common to all its instances (e.g.,\ apples are round, pears are ... (CHECK)). 



\paragraph{Visual Object Recognition in Computer Vision}
- Object detection, using pre-trained features representations (see next point) (localize object and predict its \textit{label}; trained with, e.g.\ more coarse-grained ImageNet labels or VG names)\\
- Pre-trained feature representations, trained on image classification with 1000 ImageNet fine-grained labels (predict \textit{label} for most salient object in image)\\
- Our work: can object detectors account for natural human object naming with respect to predicting the entry-level name? 