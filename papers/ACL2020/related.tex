

As we outline in the following, our work on natural object naming is of relevance to research in psychology, \langvis and computer vision. 

\paragraph{(Entry-Level) Object Names}

%cognitive science/psychology:\\
%- levels of specificity\\
%- basic-level vs. entry-level\\ 
%- object naming studies
%
%Language + Vision:\\
%Ordonez et al., Zarriess \& Schlangen, ... \\
%- Our work: empirical notion of entry-level

In psychology and cognitive science, studies of picture naming have found that humans identify objects at a preferred level of abstraction, the so-called basic-level \cite{rosch1976basic,jolicoeur1984pictures}. 
The typicality of the object with respect to this basic-level category is important for determining the preferred name, i.e.\ the entry-level name: highly typical objects (e.g.,\ a robin) are simply named by their basic-level category (\name{bird}), while for atypical objects  (e.g.,\ \name{penguin}) the more specific name corresponds to the entry-level category. 
This strictly taxonomic view suggests that entry-level names generally hold for all instances a category (e.g.,\ penguin). 
\newcite{Ordonez:2016} take up this view and learn to transfer an object's category detected by an object recognition system to its natural name, using external resources like corpus statistics. 
In contrast, our work exploits actual naming data for instances of real-world objects and tests object naming model on ManyNames, a corpus with robust name annotations from many different speakers.
This dataset is comparable to picture naming data, though we focus on real-world objects in images instead of prototypical line drawings as in e.g.\ \cite{rossion2004revisiting}, 
 \newcite{zarriess-schlangen:2017} learn a naming model from names produced in referring expressions to real-world objects, but they do not have access to name annotations from (many) different speakers and adopt very simply evaluation measures (like accuracy). 
 This work explores and adopts more nuanced evaluation metrics and aims to model robust speaker preferences with respect to natural naming.
 \sz{maybe we should not mention the Graf paper here ... we do not look at objects in context}
 
%members of different \textit{basic-level} categories (e.g., a duck, a robin, a penguin, etc. are all members of \cat{birds}) 
% \newcite{rohde2012communicating} and \newcite{graf2016animal} investigate naming in context and find that the specificity of a name is dependent on the taxonomic relatedness of other objects in context.

 \cs{Need to check Jolicoeur and their experiments; say sth about shared visual features}
% prototypical visual features common to all its instances (e.g.,\ apples are round, pears are ... (CHECK)). 



\paragraph{Visual Object Recognition in Computer Vision}
- Object detection, using pre-trained features representations (see next point) (localize object and predict its \textit{label}; trained with, e.g.\ more coarse-grained ImageNet labels or VG names)\\
- Pre-trained feature representations, trained on image classification with 1000 ImageNet fine-grained labels (predict \textit{label} for most salient object in image)\\
- Our work: can object detectors account for natural human object naming with respect to predicting the entry-level name? 