
\cs{need to make clear the diff btw objects meaning the concept/class and meaning an instance}
\cs{need to discuss problematic of standard evaluation procedure of CV models}

When people talk, they choose particular names for objects, such as \textit{bird} or \textit{duck} for the images in Figure~\ref{fig:duck}.
The task of \textbf{object naming} has been studied in Psycholinguistics~\cite{refs}, \gbt{Which term do you prefer, Psycholing, or CogSci? I don't care. Use same term throughout.} but has not received much attention in Computational Linguistics; we seek to remedy this situation by providing data and modeling results on this phenomenon, and in particular on \textbf{entry-level names}, defined in psycholinguistic research as the preferred name for an object~\cite{rosch1976basic,Rosch1978,jolicoeur1984pictures}: For the left object in the figure, according to our data, the entry-level name is \textit{bird}, and for the right object it is instead \textit{duck}.
%We seek to shed light into how people naturally name objects in real images, and in 

Our contributions are two-fold.
First, we provide data on object naming with real images on a large scale.
We extend \mn (under submission), a dataset that provides 36 names for each of 25K objects from \vg~\cite{+++}, with information on adequacy and reference of all collected names, which affords rich analysis possibilities.
Second, we provide an extensive analysis of the performance of object classification models on the task of predicting the entry-level names in our dataset.

\begin{figure}[htb]
	\centering
	\small
	\begin{tabular}{p{3cm}p{3cm}}
		\centering
		\includegraphics[scale=0.15]{images/2327551_2960743_seed_ambiguous.png} &
		\includegraphics[scale=0.15]{images/2358126_805887_singleton_obj.png}\\
		bird\ (27),  duck\ (8) & duck (33), bird (3)\\
	\end{tabular}
	
	\caption{Different naming preferences for different instances of the same category \textsc{duck} in \mn.\label{fig:duck}}
\end{figure}
As for the former, note that Psycholinguistic research on entry-level names has focused on categories, as opposed to instances:
When a subcategory is atypical, such as penguins not being very typical birds, this affects their entry-level name (it is common to call a sparrow, but not a penguin, \textit{bird}).
Accordingly, datasets used for object naming in Psycholinguistics use idealized drawings that correspond more to categories than to instances~\cite{+++++}.
However, when they speak, humans in general name specific \textit{instances} of objects ---actually, object instances in particular situations and points of view.
Our data contain object instances, corresponding more closely to the kind of visual objects that humans typically encounter, and so can potentially shed light on the instance-level factors that affect naming.
%Indeed, Psycholinguistic research has typically ignored
% The example shows two instances of the category \textsc{duck}, and when people were asked to name the highlighted object, most ($27$) people called the instance on the left \name{bird}, while \name{duck} was strongly preferred for the right instance. 
%It seems plausible that there are instance-specific factors affecting naming behavior in general.
%, and entry-level names in particular, and our data suggest that this is the case
% gbt: maybe put back a "crucially" somewhere: "Crucially, though, as we will empirically show, human object naming is \textit{instance}-dependent: It depends on the characteristics "
In particular, we validate the notion of entry-level name at the instance level:
For almost 90\% of the 25K objects analyzed, humans showed a preference for a given name (frequency of that name $>50\%$ of the valid responses).

As for the second contribution, \gbt{to be expanded; proposed structure follows; probably here only a summary and a fuller explanation later?}
% allows for an in-depth analysis of naming data as well as of naming models on the task of object naming.
\begin{itemize}
\item resources developed in CV and \lv can be useful to study object naming. 
\item However, they provide only one (or a few) name per object \ra no guarantee that it is the entry-level name. (We provide more, to have a robust estimate of entry-level names (as well as data on other possible names). \gbt{actually, maybe move this to the previous part? \ra differences with Psycholing, with CV resources})
\item object naming has often been conflated with object classification in CV. \gbt{Discuss differences; mention the swan - whatever name it has in wordnet example of ordonez et al?}
\item a big problem with this approach is that it follows a single-label evaluation. With our data, we show that often model responses are adequate even if they do not correspond to the entry-level name.
\item we also show that object classification models ``spontaneously'' learn to provide entry-level names for objects
\item and that it is possible to use \mn to successfully fine-tune existing object detection and classification models to predict entry-level names for objects.
\end{itemize}

\gbt{Below is the old text. I factored out the entry-level aspects in the previous paragraph. It should be removed from this second part.}

Objects, being members of many categories, can be called by many names (e.g.,\ a duck can be called \name{duck, bird, animal} etc.). 
In \langvis research (\lv), the choice of a particular name for an object is a ubiquitous problem---it underlies virtually all tasks that model how speakers use language to refer to objects in the world, such as image description, visual question answering, referring expression generation, etc. 
%
\lv methods are generally based on object detection or image classification models (or the visual features extracted from them) that were pre-trained towards predicting the single correct label of objects. 
The set of labels are often determined rather arbitrarily---it may contain very specific (e.g.,\ \name{goblet, gyromitra}) as well as "basic-level" labels (e.g., \name{bus}; cf.\ the label inventory of the ILSVRC challenges; \citealt{ILSVRC15}). 

This pre-training strategy has its justification in that computer vision models can learn rich, discriminative visual feature representations which capture fine-grained differences in object appearances (e.g.,\ sharp--pointed vs. slightly pointed). 
It is, however, different from predicting the natural name of an object, 
because, as has been found in numerous studies in psychology, humans have a preference towards a particular name (\textit{entry-level name}, e.g.,\ \name{duck}) when naturally calling an object  \cite{rosch1976basic,Rosch1978,jolicoeur1984pictures}. 
Entry-level names have hereby usually been considered to be an attribute of concrete \textit{concepts} (e.g.,\ penguin, duck), where the choice of a concept's entry-level name depends on factors such as its typicality with respect to its basic-level category (bird). \cs{Need to check Jolicoeur and their experiments} 
Crucially, though, as we will empirically show, human object naming is \textit{instance}-dependent---contextual visual features may humans have prefer different names for object instances of the same concrete concept, and have them even disagree in their choice for the same instance (see also \citealt{graf2016animal}). 
For example, Figure~\ref{fig:duck} shows two instances of the concept duck, and when people were asked to name the highlighted object, most ($27$) people called the instance on the left \name{bird}, while \name{duck} was strongly preferred for the right instance. 

Despite its relevance in \lv, the challenge of instance-based natural naming has been overseen, and most \lv datasets not necessarily provide enough information to make progress on the problem of modeling natural object naming (they only provide very few names). 
Research that particularly focuses on the prediction of entry-level names is scarce, and existing work has adopted the view that (i) entry-level names arise on the conceptual level, and (ii) developed specialized methods \cite{Ordonez:2016}. 
%To our knowledge, work that gives systematic insights into the ability of standard object recognition models to account for natural object naming, and in how far simple \cs{straightfoward?} re-training or transfer learning can increase this ability, is lacking. 
%

In this paper, we seek an understanding of the notion of entry-level names of instances of real-world objects in images, and to give systematic insights into the problem of retraining or fine-tuning object detection models (and features in transfer learning) such that they capture a natural vocabulary and account for linguistic preferences in naming. %, i.e.,\ the name that humans naturally prefer to call an object.  
Specifically, and in contrast to previous works, we 
(i) take on an empirical notion of entry-level names, and define it on the instance-level, i.e., the name that humans prefer to use when calling a particular object in a real-world image.   
(ii) We examine in how far standard models in computer vision, namely object detection and image classification models, do learn entry-level naming by being trained on images labeled with rather \textit{\arbitrary} (as opposed to entry-level) object names. \cs{add sth based on results - how to fine-tune or whether it's possible to fine-tune; how readily available they are to be used for transfer learning, the standard method in \lv}

We use \mn, a new dataset of real-world images, built on top of \vg, which provides an excellent resource for our study, since it was annotated with a large number\ $(36)$ of object names by means of a crowdsourcing study.

Our contributions are: 
(i) We present our extension of the \mn dataset that augments all collected object names with verification information, which allows an in-depth analysis of naming models on the task of object naming. 
(ii) Theoretical:
(a) We show that entry-level names should be derived on the instance-level as opposed to the XXX level \cs{(TO SHOW: entry-level names are different for the same "class" @Sina or @Matthijs?) (contrast to psychological studies)}
(b) We need many name annotations to derive the entry-level name (contrast to few annotations in L+V and computer vision; \cs{TO SHOW: entry-level name of an instance varies depending on the number of annotations/instance)}
(iii) Technical:
(a) Object detectors trained on "\arbitrary" natural object names do learn entry-level names to some extend (as expected: they learn the shared features leading to entry-level \cs{-- can we show that mistakes are rather class-based? i.e., tendency towards a certain name for each class?)}
(b) To predict entry-level names without annotating a huge amount of training examples, it is possible to fine-tune (iii) on \mn. We show that \cs{(say something about mistakes not made anymore). }


%%% Local Variables:
%%% mode: latex
%%% TeX-master: "acl2020_main"
%%% End:
