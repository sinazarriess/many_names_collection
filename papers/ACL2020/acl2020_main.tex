%
% File acl2020.tex
%
%% Based on the style files for ACL 2020, which were
%% Based on the style files for ACL 2018, NAACL 2018/19, which were
%% Based on the style files for ACL-2015, with some improvements
%%  taken from the NAACL-2016 style
%% Based on the style files for ACL-2014, which were, in turn,
%% based on ACL-2013, ACL-2012, ACL-2011, ACL-2010, ACL-IJCNLP-2009,
%% EACL-2009, IJCNLP-2008...
%% Based on the style files for EACL 2006 by 
%%e.agirre@ehu.es or Sergi.Balari@uab.es
%% and that of ACL 08 by Joakim Nivre and Noah Smith

\documentclass[11pt,a4paper]{article}
\usepackage[hyperref]{acl2020}

\usepackage{amsmath}
\usepackage{amsfonts}
\usepackage{amssymb}
\usepackage{booktabs}
\usepackage{color, colortbl}
\usepackage{enumitem}
\usepackage{graphicx}
\usepackage{hyperref}
\usepackage[latin1]{inputenc}
\usepackage{latexsym}
\usepackage{multirow}
\usepackage{times}
\usepackage{url}
\usepackage{xcolor}
\usepackage{xspace}

\renewcommand{\UrlFont}{\ttfamily\small}

%\usepackage{microtype}

%\aclfinalcopy % Uncomment this line for the final submission
%\def\aclpaperid{***} %  Enter the acl Paper ID here

%\setlength\titlebox{5cm}
% You can expand the titlebox if you need extra space
% to show all the authors. Please do not make the titlebox
% smaller than 5cm (the original size); we will check this
% in the camera-ready version and ask you to change it back.

\newcommand{\sz}[1]{\textcolor{blue}{\emph{//sz: #1//}}}
\newcommand{\gbt}[1]{\textcolor{orange}{\emph{//gbt: #1//}}}
\newcommand{\cs}[1]{\textcolor{green!60!black}{\emph{//cs: #1//}}}
\newcommand{\mw}[1]{\textcolor{orange!60!black}{\emph{//mw: #1//}}}

% To streamline frequently used terms:
\newcommand{\langvis}{language\ \&\ vision\xspace}
\newcommand{\lv}{L\&V\xspace}
\newcommand{\mn}{ManyNames\xspace}
\newcommand{\vg}{VisualGenome\xspace}

\newcommand{\cat}[1]{\textsc{#1}}
\newcommand{\name}[1]{\textsl{#1}}

\title{How to Name a Real-World Object, and how not to}

\author{First Author \\
  Affiliation / Address line 1 \\
  Affiliation / Address line 2 \\
  Affiliation / Address line 3 \\
  \texttt{email@domain} \\\And
  Second Author \\
  Affiliation / Address line 1 \\
  Affiliation / Address line 2 \\
  Affiliation / Address line 3 \\
  \texttt{email@domain} \\}

\date{}

\begin{document}
\maketitle
\begin{abstract}

\end{abstract}

\section{Introduction}
\label{sec:intro}
\begin{itemize}
\item We're interested in modeling human naming behavior [motivate]
\item names are sometimes used as labels for object classification in CV
\item usually: single-label evaluation scheme
  \begin{enumerate}
  \item if a single-label evaluation scheme is used, the target name should be the one that is most naturally used for the object. [Here we need a bit of background saying that it has been shown in Psycholing that people tend to agree in the name they give to objects.] To estimate this ``natural name'' (need better term), we need a reasonable sample.
  \item however, a single-label evaluation scheme assumes object names are unique (cf.\ evaluation; any other response is counted as incorrect). But alternative names are in fact possible.
  \end{enumerate}
\item (thanks to ManyNames,) we train models on the task of providing natural names; existing models, fine-tuned models 
\item and we analyze the consequences of evaluating them on a single-label scheme vs.\ one that allows for naming variants
\item (TBD: and we compare their errors/behavior to the human errors/behavior)
\end{itemize}

%%% Local Variables:
%%% mode: latex
%%% TeX-master: "acl2020_main"
%%% End:


\section{Related Work}
\label{sec:related}

[Sectioning to be revised]

As we outline in the following, our work on natural object naming is of relevance to research in psychology, \langvis and computer vision. 

\paragraph{(Entry-Level) Object Names}

cognitive science/psychology:\\
- levels of specificity\\
- basic-level vs. entry-level\\ 
- object naming studies

Language + Vision:\\
Ordonez et al., Zarriess \& Schlangen, ... \\
- Our work: empirical notion of entry-level

\paragraph{}
In psychology and cognitive science, the study of (entry-level) naming is well-established (CITE). 
Studies have usually used prototypical line drawings of members of different \textit{basic-level} categories (e.g., a duck, a robin, a penguin, etc. are all members of \cat{birds}) 
\cite{jolicoeur1984pictures}, and examined the influence of an object's typicality with respect to its category on the natural choice of its name: highly typical objects (e.g.,\ a robin) have been found to be preferably named by their basic-level category (\name{bird}), while for atypical members a subordinate category (e.g.,\ \name{penguin}) is preferred. 
As such, entry-level names have been considered being an attribute of concrete \textit{concepts} (e.g.,\ penguin). \cs{Need to check Jolicoeur and their experiments; say sth about shared visual features}
% prototypical visual features common to all its instances (e.g.,\ apples are round, pears are ... (CHECK)). 



\paragraph{Visual Object Recognition in Computer Vision}
- Object detection, using pre-trained features representations (see next point) (localize object and predict its \textit{label}; trained with, e.g.\ more coarse-grained ImageNet labels or VG names)\\
- Pre-trained feature representations, trained on image classification with 1000 ImageNet fine-grained labels (predict \textit{label} for most salient object in image)\\
- Our work: can object detectors account for natural human object naming with respect to predicting the entry-level name? 

\section{ManyNames: A Source for Entry-Level Names}
\label{sec:manynames}


\cs{@gbt @mw}
\subsection{Overview of ManyNames}
\label{sect:mn_overview}

brief summary of lrec paper, shortcoming, what we added

\subsection{Verification of Annotations}
\label{sect:mn_verification}

\mw{Below can be shortened no doubt; but could also be moved to appendix. Plz. let me know what you consider the most appropriate (I'm an ACL noobie).}

Although the initial data gathering phase involved some very basic quality control (detecting repeated names, empty text fields, typos), we had no rigorous way of ascertaining that our participants were providing adequate names for the intended object.
Given our high number of annotators in phase 1 we decided to simply discard names that were entered by only one annotator.
Moreover, for images where all 36 annotators entered the same name we decided to take this name to be adequate. 
That still leaves 19427 images, with on average 4 names each, to be properly verified, which we did as follows.

We conducted a second round of annotation where we asked people whether each name for a given image was adequate and, if not, what type of inadequacy it was.
Moreover, we asked them to indicate which names were likely intended for the same real-world object.
The task interface is shown in figure~\ref{fig:verification-interface} in the appendix.
Adequacy was rated on a 3-point scale from ``perfectly adequate'' to ``there may be a (slight) inadequacy'' to ``totally inadequate'', represented by appropriate emotes.
We provided (and clarified by means of explanation and visual examples) the following definition: \textit{a name is ``adequate'' if there is an object in the image, whose visible parts are tightly circumscribed by the red bounding box, that one could reasonably call by that name}.
When the participant selected slightly or totally inadequate, four additional icons would appear to indicate the type of inadequacy: ``linguistic mistake'', ``named object not tightly in bounding box'', ``named object mistaken for something it's not'', and ``something else''.
We provided detailed instructions and examples at the top, as well as informative hover texts on all the buttons.

Because the quality of the data we gather in this phase will be essential for gaining insight into the ManyNames dataset, as well as for evaluating computational models (section~\ref{sec:experiments}), we conducted rigorous quality control.
Every task included around 15\% automatically generated quality control items of various kinds (names for objects in a different image, names for other objects in the same image, WordNet synonyms of given names, inserted typos, and more).
When trying to submit their results, participants would get a warning if their accuracy on these items was below 90\%, and the advice to stop doing these tasks if it was below 80\%, with the option to double-check their responses.
They would get a bonus of \$0.15 if all control items were answered correctly (happened about half the time).
After every round we blocked annotators with low ($<$90\%) average accuracy, removed their results from the dataset and republished the relevant tasks to ensure consistent coverage of our data.

We implemented this verification task on Amazon Mechanical Turk (\url{https://mturk.com}). 
Basic numbers and results are reported in Table~\ref{tab:verification-numbers}.
\begin{table}[t]
	\centering
	\small
	\begin{tabular}{|ll|ll|}
		\hline
		\multicolumn{2}{|c|}{\textbf{Task setup:}} & \multicolumn{2}{c|}{\textbf{Results:}} \\ \hline
		Images: & 19427 &
			Annotators: & 255 \\
		Img-name pairs: & 69356 &
			Annotators/task: & $\geq$ 3 \\
		Tasks: & 3052 &
			Adeq. mean: & \hspace{-3em}.80 (std .093)\\
		Images/task: & 6-7 &			
			Agreement (adeq.) & 88\% \\ 
		Names/task: & 20-30 &
			Agr. (inadeq. type): & 86\% \\
		\multicolumn{2}{|l|}{Task reward: \ \$.50 (+.15)} & 
			Agr. (same-obj): & 94\% \\
		\hline
	\end{tabular}
	\caption{Verification task overview.}
	\label{tab:verification-numbers}
\end{table}
Average name adequacy, encoded on a scale $[0,1]$, is 0.80 with small standard deviation (0.093). 
Inter-annotator agreement is high also by other measures: the proportion of annotators agreeing on ``adequate'' is 88\%, agreeing on inadequacy type is 86\%, and agreeing on whether two names were intended for the same object is 94\%.
This level is agreement is high given the complexity of the task (e.g., pictures sometimes unclear), especially bearing in mind that we essentially asked our participants to do some kind of mind-reading (i.e., which object could someone who entered an inadequate name plausibly have had in mind?).
Figure~\ref{fig:verification-piechart} shows the proportion of `adequate' judgments and the types of inadequacies our annotators reported.
\begin{figure}[t]
	\centering
	\hspace*{.2\columnwidth}\includegraphics[width=.7\columnwidth]{images/verification_piechart.pdf}
	\caption{Verification results (counting individual annotations). Lighter shade within a hue indicates slight/possible error of that type; darker shade severe error.}
	\label{fig:verification-piechart}
\end{figure}
\mw{Possibly a pie-chart counting actual names and their aggregated (+thresholded) scores would be more informative than the current pie-chart, which counts original (i.e., non-aggregated) verification annotations.}

We aggregate our name adequacy judgments (representing adequate as 1, slightly inadequate as 0.5 and totally inadequate as 0) by taking the mean, and inadequacy type judgments by taking the majority (also counting ``none'' if there was no inadequacy).
As for the judgments of whether two names were intended for the same object, we use two different types of aggregations, for different purposes.
For evaluating computational object naming models, we want to know primarily whether a name generated by a model could have been intended to name the same object as the entry-level name (i.e., the most frequent name).
This we compute by a simple majority rule: true if name1 and name2 were judged to be for the same object by a majority of annotators, and false otherwise.
However, we are also interested in more general statistics, e.g., the mean number of names per object, and the relative frequencies of names for a given object.
For this, we need to compute clusters of names (the pairwise majority rule does not result in a proper clustering): as distance between names we use the proportion of annotators saying they do not name the same object, and we perform agglomerative, complete-linkage clustering with a threshold of .5 (i.e., all pairs of names in a cluster must name the same object according to a majority of annotators).
A possible disadvantage of this (hard) clustering method is that every name can only be in one cluster, i.e., name only one object.
It entails for instance that \emph{food} cannot be clustered both with \emph{pizza} and with the \emph{cheese} on top, even though both are definitely instances of food.
But this is not as problematic as it may seem, because we do not care primarily about such taxonomic relationships (which could be gotten from, e.g., WordNet) -- rather, we care about (and asked our annotators) which single object a person who entered \emph{food} was likely intending to name: presumably the pizza as a whole rather than the cheese on top.
Moreover, although (hard) clustering may occasionally lead us to \emph{under}estimate the number of names for an object, the alternative, \emph{soft} clustering (where each name could potentially be in multiple clusters) would risk \emph{over}estimating the number of names per object, as well as the number of named objects in an image, both of which would be more harmful to our aims.

% We relied on 255 mostly recurring workers, who did a total of 9491 published tasks.
% Each task would present the worker with 6 or 7 images, for a total of between 20 and 30 names.
% We offered a reward of \$0.50 with an additional bonus of \$0.15 if all control items were % answered correctly as extra incentive (which happened about half the time).
% Our approach was valued by the workers for the interesting task, natural interface and good reward.


\subsection{Analysis}
\label{sect:mn_analysis}

\cs{+ @sz?}\\
\cs{Here goes what has not been discussed in LREC paper, because it focusses on entry-level names. }
\cs{Results aimed to show that we need \underline{multiple} annotations per object \underline{instance} to get the entry-level name}

\paragraph{Definition of ``Entry-level Name''} \sz{has this been mentioned before?}
\mw{I'd definitely put this in the introduction, and leave only a reminder here.}
In the following, we consider a name annotated for an object as an entry-level name if it corresponds to the most frequent name given a set of name annotations for an object.
\mw{We could also consider defining entry-level name as `the most frequent adequate name'? Or perhaps as `the most frequent name \emph{if it is adequate}, none otherwise'? Depending on what makes the narrative easier.}

\mw{Also make sure to define the notion of `entry-level object', or `canonical object' perhaps? i.e., the object/cluster of the entry-level name.}

\paragraph{Entry-level Names in ManyNames:} As shown in Table \ref{tab:stat-entry-level}, ManyNames provides a large vocabulary of names (7970) for 25K objects. Note that this vocabulary is much larger than what state-of-the-art object detectors are typically trained on. The vocabulary of entry-level names is drastically smaller and contains only 442 types. This small set, however, covers at least 50\% of the objects in VisualGenome, i.e. almost 2M objects in VisualGenome are mentioned in a region descriptions with one of these 442 entry-level names.

\begin{table}[t]
	\centering
	\small
	\begin{tabular}{p{5cm}l}
		\toprule		
\# objects in MN & 25,315\\
total vocab  &  7,970\\
entry-level vocab & 442\\
\# objects with entry-level name in VG & $\sim$ 2M (50\%)\\
\midrule
av. agreement all names & 34.9\%\\
av. agreement entry-level names & 75.2\%\\
av. agreement entry-level cluster & 42.3\%\\
\midrule
av. adequacy all names & 0.81\\
av. adequacy entry-level names & 0.97 \\
av. adequacy entry-level cluster & 0.94 \\
\bottomrule	
	\end{tabular}
	\caption{Basic statistics for entry-level names in MN}
	\label{tab:stat-entry-level}
\end{table}

\paragraph{Entry-level Names are Instance-Dependent:} In a traditional, taxonomic view of object naming, entry-level names can be determined solely based on the object's class, meaning that all instances of a class have the same entry-level name. The MN data provides evidence that this is not the case, even though we do not have access to ground-truth class annotations in \vg (we only have names). Instead, we observe that there are many pairs of entry-level names that do not have a fixed preference ordering across instances, as illustrated in Figure \ref{fig:duck} where the rather non-prototypical instance of a \textit{duck} is named \textit{bird} by most annotators. We find that there are 879 pairs of entry-level names that show a different preference ordering depending on the instance (e.g. \textit{boy-player}, \textit{sandwich-food}, \textit{building-church}, ...).


\sz{does this solve this? Statistics wrt entry-level of object class != entry-level of its instances to (i.e.,\ entry-levels cannot be derived on class-level)} \gbt{But, what's an object class in our data? We can't really use collection synsets, can we? Those are not really object classes?} \gbt{UPDATE: We change focus from class to ``entry-level names are instance-specific''. Matthijs is on it.}


\sz{somewhere we should mention these tendencies on adequacy and agreement for (non-)entry-level names shown in Table \ref{tab:stat-entry-level}}


\paragraph{Identifying entry-level names requires many annotators}
ManyNames was collected in four rounds of 9 annotators each.
On average, the most frequent names differed between rounds for 20\% of objects (std.dev. 2.4\%p), suggesting that entry-level names cannot be reliably identified from too few annotators.
Figure~\ref{fig:entry-level-name-stability} shows this in a more general way.
\mw{@gbt: I first did it the way you suggested, i.e., by sampling, but I found the resulting plot (not shown now) too hard to interpret, even a bit misleading. I've kept this text + plot in the TeX source further below, commented out, in case you want to have a look. But I think the following is better:}
\begin{figure}[t]
	\includegraphics[width=\columnwidth]{images/stability_analytic.png}
	\caption{Proportion of objects (vertical) for which the entry-level name can be confidently (95\%) identified after gathering N names (horizontal). \mw{increase plot font size}}
	\label{fig:entry-level-name-stability}
\end{figure}
For each object we computed, on the basis of the distribution of names in ManyNames for that object, how many names should have been gathered for the majority name among them to be the entry-level name (as defined on the basis of 36 names) with 95\% probability (a reasonable confidence requirement).
The blue line shows that, even after gathering 10 names, the majority name is this likely to be the entry-level name for only 68\% of objects (put differently: for the remaining 32\% of objects, there's a greater than 5\% chance of getting a different majority name).
Gathering 20 names increases this to 78\%.
The red line shows the same tendency but by taking only names from the entry-level cluster into account (as identified by our Verification results):
by discarding names for other objects, the entry-level name for the intended object can be confidently identified slightly more quickly (but not by much, since names for the wrong object tend to be a minority anyway).
Both lines level level out well below 100\%, reflecting that for 15\%-20\% \cs{Can you compute that individually for each domain?} of objects even gathering 36 names is not enough for the entry-level name to come out as majority with 95\% probability: differences in frequencies between the contenders are simply too small.
\mw{@cs, @sz, @gbt: A closely related (but different) question this immediately brings to mind (of a critical reviewer?) is: how sure are we, for a given object, given the particular names we gathered, that the most frequent name in our sample (what we \emph{call} the entry-level name) is indeed the most frequent name in the population (what actually \emph{is} the entry-level name, right?)? Should we address this question earlier? I think the stats will look very similar (though they're not the same): that for 15-20\% of objects we really aren't sure.}

%%%%%% BY SAMPLING, FOR GEMMA:
% For each object that has an adequate entry-level name (adequacy $<.5$ according to our Verification results), we randomly sampled (including duplicates) a list of 36 names using the proportions in ManyNames as sampling probabilities, and computed the lowest number of collected names (i.e., the lowest index in the list) after which the entry-level name would reliably remain the most frequent one.
% We did this 30 times for each object.
% The blue line in the figure shows that, after collecting 10 names, the entry-level name is reliably identified for on average around 85\% of objects; collecting 20 names increases this to 90\%.\footnote{
%	But even when gathering 36 names, for around 6\% of objects this can still result in a different entry-level name. This reflects the existence of objects with no clear entry-level name, e.g., where two (almost) equally frequent names are both candidates.
%	\mw{Can we quantify this?}
%}
%The red line shows the same tendency but by taking only names from the entry-level cluster into account (as identified by our Verification results):
%by discarding names for other objects, the entry-level name for the intended object can be identified slightly more quickly (but not by much, since names for the wrong object tend to be a minority anyway).
%\begin{figure}[t]
%	\includegraphics[width=\columnwidth]{images/stability-all-05-05-spellchecked.png}
%	\caption{}
%	\label{fig:entry-level-name-stability}
%\end{figure}

\paragraph{Of Verification Annotations}

\cs{@gbt @mw}



\begin{itemize}
	\item ...
	\item ...
	\item For the instances where VG!=topMN: Percentage of instances where the VG name is among the responses of an \textit{alternative object} (as given by clustering).
	\item Comparison of VG and MN, demonstrating that MN is a more reliable source for entry-level names. 
\end{itemize}


%%% Local Variables:
%%% mode: latex
%%% TeX-master: "acl2020_main"
%%% End:


\section{Experiments}
\label{sec:experiments}
% experiments.tex

\cs{todo: discuss differences between domains}

Our goal is to experimentally obtain insights into the usefulness of computer vision object labeling models, that are commonly used for \lv methods, to account for natural human object naming. 
Specifically, our research questions are
\begin{enumerate}
	\item Do object detection models, despite being trained on \arbitrary  
	names, account for human object naming?\cs{repetitive}
	\item Can image features, that were pre-trained using object detection or image classification model, be fine-tuned on \mn towards this naming task through simple transfer learning? Figure~\ref{fig:exp_confusions} \\
	\cs{move to models: -- can we learn entry-level names by fine-tuning object detection models on \mn?\\
		-- Can we directly learn entry-level names by fine-tuning image classification models on \mn, i.e.,\ the pre-trained features to initialize object detection models?}
\end{enumerate}
TO BE COVERED BY THE SUBSECTIONS BELOW:
\begin{enumerate}
	\item Do models prefer the entry-level name of an object over an \cs{arbitrary$\rightarrow$?} name? Table~\ref{tab:exp_VGvsMN}
	\item In cases of failure, do they predict alternative, but less preferred  names instead? Tables~\ref{tab:exp_overview_results}, \ref{tab:exp_alternatives}
	\item What phenomena of human behavior on the object naming task of \mn do the models mirror? That is, which model mistakes or confusions are due to the artificial setup of the task itself, and less a failure of the model? (Recall that \mn serves as a \textit{proxy} for natural human object naming, since annotators were asked to name a highlighted object in an image isolated from its situative context; Section~\ref{sec:manynames}) Table~\ref{tab:exp_details_wrong} (and \ref{tab:exp_errors_agreement},  \ref{tab:exp_alternatives})
	
	%That is, do they make similar mistakes (wrong object--maybe more salient?, related name, e.g.,~co-hyponyms)?
	\item \cs{Not examined at the moment -- MRR and Jaccard may be re-added for this: Are valid name alternatives among the top-N predictions?}
	%\item do we get different evaluation results when testing existing object classification models on preferred responses from name distributions, as compared to naming responses collected from 1-3 workers (e.g., VisualGenome)? 
\end{enumerate}
We use the \mn dataset in two ways, as analytical testbed, and for fine-tuning pre-trained image features towards entry-level name prediction.


\subsection{Experimental Setup}
\label{sect:exp_setup}

\paragraph{Data}

\paragraph{Models}
We test two object detection models that are trained with a different vocabulary:

\begin{itemize}
	\item FRCNN--VG1600: we used the code and model of the bottom-up attention model (\textit{Bottom-up} henceforth)  \citeauthor{anderson2018updown}'s \citeyear{anderson2018updown} \footnote{\url{https://github.com/peteanderson80/bottom-up-attention}} The model is based on the object detection method Faster R-CNN \cite{fasterrcnn2015}. The authors initialized it with ResNet-101 \cite{he2016deep} features (pretrained on ImageNet) to then train it on \vg.\footnote{Bottom-up is trained with an additional output over attributes, but we only use the object class prediction layer.} 
	
	\item FRCNN--MN442: we retrain Faster R-CNN \cite{fasterrcnn2015} on a smaller vocab. We use \citeauthor{anderson2018updown}'s to preprocess the data set, but restrict the object classes to the 442 entry-level names found in ManyNames. For training, we use a PyTorch reimplementation of Faster R-CNN\footnote{\url{https://github.com/jwyang/faster-rcnn.pytorch}} and initialized it with a pretrained ResNet-101 provided in this implementation.\footnote{ResNet-101 was pretrained with Caffe.}
\end{itemize}

We test a range of object classification models via transfer learning, that are finetuned with different vocabulary/training data:

\begin{itemize}
	\item ...
\end{itemize}

\paragraph{Measures}


\subsection{Results}
\label{sect:exp_results}

\paragraph{Entry-Level Name Prediction}

\begin{table}[t]
	\centering
	\small
	\begin{tabular}{l@{~}|@{~}c@{~}|@{~}r@{~}r@{~}|@{~}r@{~}r@{~}}
		\toprule
		&  & \multicolumn{2}{c}{All} 
		& \multicolumn{2}{c}{VG$\neq$MN}\\
		&  & \multicolumn{2}{c}{($1072$)} 
		& \multicolumn{2}{c}{($223$)}\\	
		Model--Vocab	 
		&  GTtrain &  =VG & =MN 
		&  =VG & =MN \\ 
		\midrule
		FRCNN--MN442 & VG &  65.4 &      71.2 &   20.0 &      48.4  \\
		FRCNN--VG1600 & VG &    67.3 &      74.5 &    19.1 &      52.9  \\
		\midrule \midrule
		\multicolumn{6}{c}{Classifiers: Fine-tuning pre-trained features on MN}\\
		Features--Vocab &   \\
		\midrule 
		FRCNN--VG1600--VGMN & MN &    70.8 &      80.6 &    13.8 &      60.4  \\ 
		\midrule
		ResNet101--VGMN & MN  & 60.9 &  68.6 &  13.8 & 50.2  \\
		
		ResNet101--MN442 & MN &            61.7 &              69.6 &                        13.8 &              50.7 \\
		ResNet101--VGMN\_4ep  &   VG &  62.4 &              62.9 &              28.4 &              31.1  \\
		ResNet101--VGMN\_8ep & VG &            63.9 &              62.6 &          34.2 &              28.0 \\
		ResNet101--MN442 & VG  &            63.7 &              63.7 &                  30.7 &              31.1  \\
		\bottomrule
	\end{tabular}
	\caption{Target vocabulary in test data: MN442. Vocab denotes the dataset from which the target vocabulary for training was induced (the numbers give the size of the vocabulary). GTtrain denotes the dataset from which the ground truth labels are obtained during \textit{training}. Note that we considered all name responses in MN, including those with $\text{count}(name)<2$\label{tab:exp_VGvsMN}. }
\end{table}

\paragraph{Alternative Name Prediction}
\begin{figure}
	\centering
	\includegraphics[scale=.2]{images/2323938.jpg}
	\includegraphics[scale=.2]{images/2322259.jpg}
	\includegraphics[scale=.2]{images/2371657.jpg}
	
	\caption{TODO: examples resnet mistakes (trained on vg\_manynames, tested on manynames-442)\label{fig:mistakes}}
\end{figure}

\cs{TODO::}
Categorization of "errors" (see Figure~\ref{fig:mistakes}):
\begin{enumerate}
	\item Clear mistake \\
	e.g.,\ rice vs. bread
	\item Alternative name\\
	e.g.,\ building vs. house
	\item Alternative object \cs{(other cluster from verif data)}
	\item Synonym\\
	e.g.,\ plane vs. airplane
	\item Semantically related\\
	e.g.,\  motorcycle vs. scooter
\end{enumerate}

\begin{table*}[t]
	\centering
	\small
	\begin{tabular}{l|l|r@{~}r@{~}r@{~}||r@{~}r@{~}r@{~}}
		\toprule
		& & \multicolumn{3}{c}{All Test Images ($\#$)} 
		& \multicolumn{3}{c}{VG$\neq$MN Images ($\#$)}\\
		\toprule
		Model--Vocab	& GTtrain  
		&  hit &  correct &  incorrect &  hit &  correct &  wrong \\
		\midrule
		FRCNN--VG1600 & VG           &         74.8 &                  13.9 &                    11.3 &         54.3 &                  30.0 &                    15.7 \\
		FRCNN--MN442 & VG &         71.1 &                  13.9 &                    15.0 &         48.4 &                  28.7 &                    22.9 \\
		\midrule \midrule
		FRCNN--VG1600--VGMN & MN %&         0.81 &                  0.09 &                    0.11 &         0.60 &                  0.23 &                    0.17 \\
		&         80.7 &                   9.2 &                    10.1 &         60.1 &                  23.8 &                    16.1 \\
		\midrule
		ResNet101--MN442 & MN %& 0.70 &                  0.09 &                    0.21 &         0.51 &                  0.22 &                    0.27 \\
		&         69.7 &                  10.3 &                    20.1 &         51.1 &                  23.3 &                    25.6 \\
		ResNet101--VGMN & MN% &         0.69 &                  0.09 &                    0.22 &         0.51 &                  0.23 &                    0.26 \\	
		&         68.7 &                  10.5 &                    20.8 &         50.7 &                  24.2 &                    25.1 \\
		ResNet101--VGMN & VG %&         0.63 &                  0.11 &                    0.26 &         0.31 &                  0.28 &                    0.40 \\
		&         62.8 &                  12.6 &                    24.6 &         31.4 &                  30.0 &                    38.6 \\
		ResNet101--MN442 & VG %&         0.64 &                  0.11 &                    0.25 &         0.32 &                  0.29 &                    0.39 \\
		&         63.8 &                  12.6 &                    23.6 &         32.3 &                  30.0 &                    37.7 \\
		\bottomrule
	\end{tabular}
	\caption{Break-down of the results (in \%): Categorization of a predicted name\ $\hat{n}$ into either a \textit{hit} (exact match with entry-level name, cf. standard evaluation), \textit{correct} (less preferred name, synonym, hypernym/hyponym), or \textit{wrong} (wrong object, $\text{count}(\hat{n})<2$, co-hyponym, clear mistake). \label{tab:exp_overview_results}}
\end{table*}

\begin{table*}[t]
	\centering
	\small
	\begin{tabular}{lrrr||rrr}
		\toprule
		&  \multicolumn{3}{c||}{MN agreement $>$ 0.9} & \multicolumn{3}{c}{MN agreement $\leq$ 0.9}\\
		model &  hit &  correct &  incorrect &  hit &  correct &  incorrect \\
		\midrule
		FRCNN--VG1600\_VG &      94.8 &           1.8 &             3.4 &     63.6 &         24.5 &           12.0 \\
		FRCNN--MN442\_VG &      89.6 &           1.6 &             8.8 &     60.7 &         23.9 &           15.4 \\
		\midrule \midrule
		FRCNN--VG1600--VGMN\_MN &      94.5 &           1.3 &             4.2 &     72.9 &         16.5 &           10.6 \\
		\midrule 
		ResNet101--VGMN\_VG &      88.3 &           1.8 &             9.9 &     48.5 &         23.6 &           27.8 \\
		ResNet101--VGMN\_MN &      89.6 &           1.6 &             8.8 &     57.0 &         19.4 &           23.6 \\
		ResNet101--MN442\_MN &      90.1 &           1.3 &             8.6 &     58.2 &         19.5 &           22.3 \\
		\bottomrule
		
	\end{tabular}
	\caption{Break-down of the results (in \%) according to the agreement level of the MN name: Categorization of a predicted name\ $\hat{n}$ into either a \textit{hit}, \textit{correct} (less preferred name, synonym, hypernym/hyponym), or \textit{wrong} \label{tab:exp_errors_agreement}}
\end{table*}

\begin{table*}[t]
	\centering
	\small
	\begin{tabular}{llr@{~}|r@{~}r@{~}r@{~}r@{~}r@{~}||r@{~}|r@{~}r@{~}r@{~}r@{~}r@{~}}
		\toprule
		& & \multicolumn{6}{c}{All Test Images ($\#$)} 
		& \multicolumn{6}{c}{VG$\neq$MN Images ($\#$)}\\
		\toprule
		& &  same &  syn. &  syn. &  hyper. &  hypo. &  hyper. &  same &  syn. &  syn. &  hyper. &  hypo. &  hyper. \\
		& 	&  cluster &  & cluster & & & cluster 
		& cluster  &  & cluster & & & cluster \\
		\midrule
		FRCNN--VG1600 & VG     %        &                  0.95 &              0.0 &                0.01 &              0.01 &             0.03 &                 0.0 &                  0.96 &              0.0 &                 0.0 &              0.01 &             0.03 &                 0.0 \\
		&                  94.6 &              0.0 &                 0.7 &               3.4 &              1.3 &                  0.0 &                  95.5 &              0.0 &                 0.0 &               3.0 &              1.5 &                  0.0 \\
		FRCNN--MN442 & VG %&                   0.97 &             0.01 &                0.02 &               0.0 &              0.0 &                 0.0 &                  0.97 &              0.0 &                0.03 &               0.0 &              0.0 &                 0.0 \\
		&                  93.3 &              1.3 &                 2.0 &               1.3 &              2.0 &                  0.0 &                  93.8 &              0.0 &                 3.1 &               1.6 &              1.6 &                  0.0 \\
		\midrule \midrule
		FRCNN--VG1600--VGMN & MN %&                   1.0 &              0.0 &                 0.0 &               0.0 &              0.0 &                 0.0 &                   1.0 &              0.0 &                 0.0 &               0.0 &              0.0 &                 0.0 \\
		&                  94.9 &              0.0 &                 0.0 &               2.0 &              3.0 &                  0.0 &                  96.2 &              0.0 &                 0.0 &               1.9 &              1.9 &                  0.0 \\
		\midrule
		ResNet101--VGMN & MN %&                  0.97 &              0.0 &                0.03 &               0.0 &              0.0 &                 0.0 &                  0.98 &              0.0 &                0.02 &               0.0 &              0.0 &                 0.0 \\
		&                  83.9 &              0.0 &                 2.7 &               6.2 &              5.4 &                  1.8 &                  92.6 &              0.0 &                 1.9 &               1.9 &              3.7 &                  0.0 \\
		ResNet101--MN442 & MN %& 0.97 &              0.0 &                0.03 &               0.0 &              0.0 &                 0.0 &                  0.98 &              0.0 &                0.02 &               0.0 &              0.0 &                 0.0 \\
		&                  86.4 &              0.0 &                 2.7 &               5.5 &              3.6 &                  1.8 &                  92.3 &              0.0 &                 1.9 &               1.9 &              3.8 &                  0.0 \\
		ResNet101--VGMN & VG %&                   0.98 &              0.0 &                0.02 &               0.0 &              0.0 &                 0.0 &                  0.98 &              0.0 &                0.02 &               0.0 &              0.0 &                 0.0 \\
		&                  87.4 &              0.0 &                 2.2 &               1.5 &              8.1 &                  0.7 &                  92.5 &              0.0 &                 1.5 &               1.5 &              4.5 &                  0.0 \\
		ResNet101--MN442 & VG% &                  0.98 &              0.0 &                0.02 &               0.0 &              0.0 &                 0.0 &                  0.97 &              0.0 &                0.03 &               0.0 &              0.0 &                 0.0 \\		
		&                  88.9 &              0.0 &                 2.2 &               2.2 &              5.9 &                  0.7 &                  92.5 &              0.0 &                 3.0 &               1.5 &              3.0 &                  0.0 \\
		\bottomrule
	\end{tabular}
	
	\caption{Break-down of the results for the \textit{correct} name predictions. Proportions (in \%) of the \textit{correct} categories to all correctly classified instances.  \textit{hyponym}: $\hat{n}$ is a hyponym of the entry-level name. \textit{hypernym\_cl}: $\hat{n}$ is a hypernym of any of the valid names (cluster). \textit{Synonym} and \textit{synonym\_cl} are analogous. \label{tab:exp_alternatives}}
\end{table*}

\paragraph{Incorrect Predictions}

\begin{table*}[t]
	\centering
	\small
	\begin{tabular}{ll|r@{~}|r@{~}r@{~}r@{~}r@{~}|r@{~}r@{~}||r@{~}|r@{~}r@{~}r@{~}r@{~}|r@{~}r@{~}}
		\toprule
		&& \multicolumn{7}{c}{All Test Images ($\#$)} 
		& \multicolumn{7}{c}{VG$\neq$MN Images ($\#$)}\\
		\toprule
		Model--Vocab & GTtrain  
		&  co- &  \multicolumn{4}{c}{other object}  &  error &  low 
		&  co- &  \multicolumn{4}{c}{other object}  &  error &  low \\
		& & hypo. & (vis. &  ling. &  box &  other)   & & count 
		&  hypo. & (vis. &  ling. &  box &  other) &   & count     \\
		
		\midrule
		FRCNN--VG1600 & VG     &                 13.2 &             1.7 &                 0.0 &                  17.4 &            6.6 &           39.7 &             21.5 &                  5.7 &             5.7 &                 0.0 &                  17.1 &           14.3 &           42.9 &             14.3 \\
		FRCNN--MN442 & VG       &                 15.5 &             0.6 &                 0.0 &                  15.5 &            6.2 &           49.1 &             13.0 &                  5.9 &             2.0 &                 0.0 &                  13.7 &           11.8 &           54.9 &             11.8 \\
		\midrule \midrule
		FRCNN--VG1600--VGMN & MN &                 30.6 &             0.9 &                 0.0 &                  13.0 &            5.6 &           32.4 &             17.6 &                 16.7 &             2.8 &                 0.0 &                  16.7 &           11.1 &           41.7 &             11.1 \\
		\midrule
		ResNet101--VGMN & MN	&                 34.1 &             0.0 &                 0.0 &                  13.0 &            1.3 &           39.5 &             12.1 &                 19.6 &             0.0 &                 0.0 &                  12.5 &            5.4 &           51.8 &             10.7 \\
		ResNet101--MN442 & MN  &                 33.0 &             0.0 &                 0.0 &                  14.0 &            1.9 &           37.7 &             13.5 &                 21.1 &             0.0 &                 0.0 &                  15.8 &            7.0 &           43.9 &             12.3 \\
		ResNet101--VGMN & VG  &                 37.6 &             0.0 &                 0.0 &                   6.1 &            2.3 &           41.1 &             12.9 &                 24.4 &             0.0 &                 0.0 &                   3.5 &            5.8 &           45.3 &             20.9 \\
		ResNet101--MN442 & VG &                 34.0 &             0.0 &                 0.0 &                   7.5 &            2.8 &           41.9 &             13.8 &                 22.6 &             0.0 &                 0.0 &                   6.0 &            8.3 &           39.3 &             23.8 \\
		\bottomrule
	\end{tabular}
	\caption{Break-down of the results for the \textit{wrong} name predictions. Proportions (in \%) of the corresponding categories to all wrongly classified instances.  \label{tab:exp_details_wrong}}
\end{table*}

\paragraph{Human Object Naming through Transfer-Learning} Figure \ref{fig:exp_confusions} visualizes the change in predictions for  some of our retrained and finetuned models with respect to the object detection baseline (bottom-up-1600). For each object, where the new model predicts a different name than the baseline model, we look at the hit-error categories of the original and new predicted name. Observations:
\begin{itemize}
	\item Retraining faster-rcnn on less names does not lead to better calibration of entry-level names: more original hits change to same-cluster predictions than the other way round. (top left matrix)
	\item Finetuning the original faster-rcnn on many names recalibrates many decision from same-cluster names to the correct hit. Interestingly, also clear errors are calibrated to perfect hits, whereas hardly any error is changed to same-cluster or related.  (top right matrix)
	\item A similar tendency can be found for the ResNet models: finetuning ResNet on MN changes many predictions from same-cluster to hits. This is less the case when ResNet is finetuned on VG annotations.
	\item Interestingly, both ResNet models change hits into related names (hypernyms, hyponyms, cp-hyponyms) -- why does this happen?
\end{itemize}

\begin{figure*}[t]
	\includegraphics[scale=.5]{images/matrix_FRCNN--VG1600_VG_FRCNN--MN442_VG.jpg}
	\includegraphics[scale=.5]{images/matrix_FRCNN--VG1600_VG_FRCNN--VG1600--VGMN_MN.jpg}
	\includegraphics[scale=.5]{images/matrix_FRCNN--VG1600_VG_ResNet101--VGMN_VG.jpg}
	\includegraphics[scale=.5]{images/matrix_FRCNN--VG1600_VG_ResNet101--MN442_MN.jpg}
	
	\caption{Confusion-matrix-style visualization showing error categories of predictions that changed from object detection with FRCNN-VG1600 to naming with FRCNN-VG1600-VGMN (finetuned) \label{fig:exp_confusions}}
\end{figure*}


\subsection{Discussion}
\label{sect:exp_discussion}

\iffalse
\cs{under construction}
\begin{itemize}
	\item Object detection models in CV have different goal than L+V object recognition models (see related work--former is on labels, latter on predicting natural language). 
	However, the former, pre-trained towards labels, are the backbone / used as feature extractors for the latter.
	\item ... 
	\item As we explained, there is a high variation of object names, and objects may be named by multiple alternative names (\cs{vocab size MN vs. vocab size VG for same object set}). 
	Yet, humans usually have a clearly preferred name for individual object instances (and the set of those is relatively small)--humans agree on a particular name for an instance (entry-level name).
	\item Hence, to model human object naming, datasets with many/dozens of annotations for the same instance are required. 
	\item We argue that such datasets are more reliable in that they provide empirically derived preferred names (entry-level names), and richer--the provide valid name alternatives. 
	\item But since their elicitation is expensive and time-consuming, it is not realistic to create training datasets of dozens of object names for pre-training features with CV models, which need a huge amount of training data. 
\end{itemize}
\fi

\iffalse
We propose to use datasets of object names for evaluating models on the task of object naming (depending on results: also valuable for comparing object detection/image classification models).
Specifically, we analyse object detection [and classification] models on the task of human object naming, using the \mn dataset as test data:\\
Can object detection models, trained on arbitrary names, account for human object naming?
\begin{itemize}
	\item Do they predict the entry-level name?
	\item Are valid name alternatives among the top-N predictions?
	\item Do models make similar mistakes as humans when being faced with the task of naming highlighted objects in images [i.e.,\ the artificial setup of object naming for data annotation]?\\
	-- naming an alternative object (maybe more salient)\\
	-- predicting a semantically related name
	%do we get different evaluation results when testing existing object classification models on preferred responses from name distributions, as compared to naming responses collected from 1-3 workers (e.g., VisualGenome)? 
	\item Can we use \mn as fine-tuning dataset? (Here: only Vanilla model)\\
	-- can we learn entry-level names by fine-tuning object detection models on \mn?\\
	-- Can we directly learn entry-level names by fine-tuning image classification models on \mn, i.e.,\ the pre-trained features to initialize object detection models?
\end{itemize}

\subsection{Experimental Setup}
\label{sect:exp_setup}

\paragraph{Data}

\paragraph{Models}

We test two object detection models that are trained with a different vocabulary:

\begin{itemize}
\item FRCNN--VG1600: we used the code and model of the bottom-up attention model (\textit{Bottom-up} henceforth)  \citeauthor{anderson2018updown}'s \citeyear{anderson2018updown} \footnote{\url{https://github.com/peteanderson80/bottom-up-attention}} The model is based on the object detection method Faster R-CNN \cite{fasterrcnn2015}. The authors initialized it with ResNet-101 \cite{he2016deep} features (pretrained on ImageNet) to then train it on \vg.\footnote{Bottom-up is trained with an additional output over attributes, but we only use the object class prediction layer.} 

\item FRCNN--MN442: we retrain Faster R-CNN \cite{fasterrcnn2015} on a smaller vocab. We use \citeauthor{anderson2018updown}'s to preprocess the data set, but restrict the object classes to the 442 entry-level names found in ManyNames. For training, we use a PyTorch reimplementation of Faster R-CNN\footnote{\url{https://github.com/jwyang/faster-rcnn.pytorch}} and initialized it with a pretrained ResNet-101 provided in this implementation.\footnote{ResNet-101 was pretrained with Caffe.}
\end{itemize}

We test a range of object classification models via transfer learning, that are finetuned with different vocabulary/training data:

\begin{itemize}
\item ...
\end{itemize}

\paragraph{Measures}


\subsection{[TBC] Predicting Entry-Level Names}
\label{sect:exp_entry}
Question: Can object detection models trained on a set of "arbitrarily" chosen object names account for entry-level object names? 

Note that the source dataset for defining the vocabulary (\textsl{Vocab}, i.e.,\ the overall set of considered names, i.e., the softmax layer's dimensions), and the source dataset which provides the ground truth names for the individual objects (\textsl{GTtrain}) may differ.  
\begin{table*}[t]
	\centering
	\small
	\begin{tabular}{l|c|r@{~}r@{~}r@{~}r@{~}r@{~}r@{~}r|@{~}r@{~}r@{~}r@{~}r@{~}r@{~}r@{~}r@{~}}
		\toprule
		&   & \multicolumn{6}{c}{All Test Images ($\#$)} 
		& \multicolumn{6}{c}{VG$\neq$MN Images ($\#$)}\\	
		Model--Vocab	 
		&  GTtrain &  =VG & =MN & $\in$MN  & KL & J & MRR & AvgMRR 
		&  =VG & =MN & $\in$MN  & KL & J & MRR & AvgMRR\\ 
		\midrule
		FRCNN--MN442 & VG &            65.4 &              71.2 &                85.6 &         1.0 &             69.8 &          0.8 &             0.7 &            20.0 &              48.4 &                78.7 &         1.4 &             60.4 &          0.7 &             0.5 \\
		FRCNN--VG2500 & VG \\
		FRCNN--VG1600 & VG &            67.3 &              74.5 &                89.2 &         0.6 &             74.3 &          0.8 &             0.7 &            19.1 &              52.9 &                86.2 &         0.8 &             69.4 &          0.7 &             0.6 \\
		\midrule \midrule
		& \multicolumn{12}{c}{Classifiers: Fine-tuning pre-trained image features on \mn}\\
		Features--Vocab & GTtrain  \\
		\midrule 
		FRCNN--VG1600--VGMN & MN &            70.8 &              80.6 &                90.1 &         4.7 &             62.0 &          0.8 &             0.6 &            13.8 &              60.4 &                85.8 &         4.6 &             47.3 &          0.7 &             0.5 \\ 
		FRCNN--VG1600--MN442 &  MN \\
		\midrule
		ResNet101--VGMN & MN  &            60.9 &              68.6 &                77.9 &         4.9 &             56.4 &          0.8 &             0.6 &            13.8 &              50.2 &                73.3 &         4.7 &             42.9 &          0.6 &             0.4 \\
		
		ResNet101--MN442 & MN &            61.7 &              69.6 &                78.9 &         4.9 &             57.4 &          0.8 &             0.6 &            13.8 &              50.7 &                73.8 &         4.6 &             45.1 &          0.6 &             0.5 \\
		ResNet101--VGMN\_4ep  &   VG &  62.4 &              62.9 &                75.0 &         5.0 &             53.7 &          0.7 &             0.6 &            28.4 &              31.1 &                64.9 &         4.8 &             39.1 &          0.4 &             0.4 \\
		ResNet101--VGMN\_8ep & VG &            63.9 &              62.6 &                75.6 &         5.0 &             53.9 &          0.7 &             0.6 &            34.2 &              28.0 &                66.2 &         4.9 &             39.7 &          0.4 &             0.4 \\
		ResNet101--MN442 & VG  &            63.7 &              63.7 &                76.6 &         5.0 &             55.4 &          0.7 &             0.6 &            30.7 &              31.1 &                67.1 &         4.8 &             41.0 &          0.4 &             0.4 \\
		\bottomrule
	\end{tabular}
	\caption{Target vocabulary in test data: MN442. Vocab denotes the dataset from which the target vocabulary for training was induced (the numbers give the size of the vocabulary). GTtrain denotes the dataset from which the ground truth labels are obtained during \textit{training}. MRR is mean reciprocal rank; J is Jaccard score. Note that we considered all name responses in MN, including those with $\text{count}(name)<2$\label{tab:entrylevels}. }
\end{table*}


\begin{table*}[t]
	\centering
	\small
	\begin{tabular}{l@{~}|rrrr}
		\toprule
		... \\
		\midrule
		Domain	 & ... \\ 
		\midrule
		All           \\
		home           \\
		food           \\
		buildings      \\
		vehicles       \\
		animals\_plants \\
		clothing       \\
		people         \\
		\bottomrule
	\end{tabular}
	\caption{RESULTS FOR SELECTED MODELS \label{tab:domains_bestmodel}}
\end{table*}

\subsection{[TBC] Humans vs. Models: Which Mistakes do they Make?}
\label{sect:exp_analysis}

\paragraph{Categorization of Errors}
see Figure\ ref{fig:mistakes}
\begin{enumerate}
	\item Clear mistake \\
	e.g.,\ rice vs. bread
	\item Alternative name\\
	e.g.,\ building vs. house
	\item Alternative object \cs{(other cluster from verif data)}
	\item Synonym\\
	e.g.,\ plane vs. airplane
	\item Semantically related\\
	e.g.,\  motorcycle vs. scooter
\end{enumerate}

\begin{figure}
	\centering
	\includegraphics[scale=.2]{images/2323938.jpg}
	\includegraphics[scale=.2]{images/2322259.jpg}
	\includegraphics[scale=.2]{images/2371657.jpg}
	
	\caption{resnet mistakes (trained on vg\_manynames, tested on manynames-442)\label{fig:matrices}}
\end{figure}

\subsection{Results}
\paragraph{Overview}
Table\ \ref{tab:exp_overview_results}: \cs{Shows the overview and what MN adds without going into detail: Standard evaluation -- hit; We can additionally distinguish between correct (name alternative, see caption of Table) and clear mistake.}

\paragraph{How do predictions change?} Figure \ref{fig:matrices} visualizes the change in predictions for  some of our retrained and finetuned models with respect to the object detection baseline (bottom-up-1600). For each object, where the new model predicts a different name than the baseline model, we look at the hit-error categories of the original and new predicted name. Observations:
\begin{itemize}
\item Retraining faster-rcnn on less names does not lead to better calibration of entry-level names: more original hits change to same-cluster predictions than the other way round. (top left matrix)
\item Finetuning the original faster-rcnn on many names recalibrates many decision from same-cluster names to the correct hit. Interestingly, also clear errors are calibrated to perfect hits, whereas hardly any error is changed to same-cluster or related.  (top right matrix)
\item A similar tendency can be found for the ResNet models: finetuning ResNet on MN changes many predictions from same-cluster to hits. This is less the case when ResNet is finetuned on VG annotations.
\item Interestingly, both ResNet models change hits into related names (hypernyms, hyponyms, cp-hyponyms) -- why does this happen?
\end{itemize}


\paragraph{Correct predictions}
Table\ \ref{tab:exp_details_correct}: \cs{Gives detailed results to the categories of "correct name" (but not hit): same cluster, WordNet synonym, WordNet hypernym/hyponym}\\
To look into in detail (example images): 
\begin{itemize}
	\item Compare FRCNN vs. FRCNN-finetuned (row block 1 vs. row block 2) with respect to the synonym categories (i.e., predicted name is in a synonym/synonyms\_cluster-relation to the target object) vs. same\_cluster (i.e., predicted name is in response set). 
\end{itemize}

\paragraph{Wrong predictions}

\begin{figure*}[t]
\includegraphics[scale=.5]{images/matrix_FRCNN--VG1600_VG_FRCNN--MN442_VG.jpg}
\includegraphics[scale=.5]{images/matrix_FRCNN--VG1600_VG_FRCNN--VG1600--VGMN_MN.jpg}
\includegraphics[scale=.5]{images/matrix_FRCNN--VG1600_VG_ResNet101--VGMN_VG.jpg}
\includegraphics[scale=.5]{images/matrix_FRCNN--VG1600_VG_ResNet101--MN442_MN.jpg}

\caption{Confusion-matrix-style visualization showing error categories of predictions that changed from object detection with FRCNN-VG1600 to naming with FRCNN-VG1600-VGMN (finetuned)}
\end{figure*}



Table\ \ref{tab:exp_details_wrong}: \cs{Gives detailed results to the categories of predicted name\ $\hat{n}$ is incorrect: WordNet co-hyponyms,  other object (inadequacy types: visual, linguistic, bounding box, other (types other+None), $\text{count}(\hat{n})<2$, error (just wrong+unkn(not found in WordNet))}

\begin{table*}[t]
	\centering
	\small
	\begin{tabular}{l|l|r@{~}r@{~}r@{~}||r@{~}r@{~}r@{~}}
		\toprule
		& & \multicolumn{3}{c}{All Test Images ($\#$)} 
		& \multicolumn{3}{c}{VG$\neq$MN Images ($\#$)}\\
	\toprule
	Model--Vocab	& GTtrain  
	&  hit &  correct &  incorrect &  hit &  correct &  wrong \\
	\midrule
	FRCNN--VG1600 & VG           &         74.8 &                  13.9 &                    11.3 &         54.3 &                  30.0 &                    15.7 \\
	FRCNN--MN442 & VG &         71.1 &                  13.9 &                    15.0 &         48.4 &                  28.7 &                    22.9 \\
	\midrule \midrule
	FRCNN--VG1600--VGMN & MN %&         0.81 &                  0.09 &                    0.11 &         0.60 &                  0.23 &                    0.17 \\
	 &         80.7 &                   9.2 &                    10.1 &         60.1 &                  23.8 &                    16.1 \\
	\midrule
	ResNet101--MN442 & MN %& 0.70 &                  0.09 &                    0.21 &         0.51 &                  0.22 &                    0.27 \\
	 &         69.7 &                  10.3 &                    20.1 &         51.1 &                  23.3 &                    25.6 \\
	ResNet101--VGMN & MN% &         0.69 &                  0.09 &                    0.22 &         0.51 &                  0.23 &                    0.26 \\	
	 &         68.7 &                  10.5 &                    20.8 &         50.7 &                  24.2 &                    25.1 \\
	ResNet101--VGMN & VG %&         0.63 &                  0.11 &                    0.26 &         0.31 &                  0.28 &                    0.40 \\
	 &         62.8 &                  12.6 &                    24.6 &         31.4 &                  30.0 &                    38.6 \\
	ResNet101--MN442 & VG %&         0.64 &                  0.11 &                    0.25 &         0.32 &                  0.29 &                    0.39 \\
	 &         63.8 &                  12.6 &                    23.6 &         32.3 &                  30.0 &                    37.7 \\
	\bottomrule
\end{tabular}
\caption{Break-down of the results (in \%): Categorization of a predicted name\ $\hat{n}$ into either a \textit{hit} (exact match with entry-level name, cf. standard evaluation), \textit{correct} (less preferred name, synonym, hypernym/hyponym), or \textit{wrong} (wrong object, $\text{count}(\hat{n})<2$, co-hyponym, clear mistake). \label{tab:exp_overview_results}}
\end{table*}

\begin{table*}[t]
\centering
	\small
\begin{tabular}{lrrr||rrr}
\toprule
&  \multicolumn{3}{c||}{MN agreement $>$ 0.9} & \multicolumn{3}{c}{MN agreement $\leq$ 0.9}\\
                  model &  hit &  correct &  incorrect &  hit &  correct &  incorrect \\
\midrule
       FRCNN--VG1600\_VG &      94.8 &           1.8 &             3.4 &     63.6 &         24.5 &           12.0 \\
        FRCNN--MN442\_VG &      89.6 &           1.6 &             8.8 &     60.7 &         23.9 &           15.4 \\
        		\midrule \midrule
 FRCNN--VG1600--VGMN\_MN &      94.5 &           1.3 &             4.2 &     72.9 &         16.5 &           10.6 \\
 		\midrule 
     ResNet101--VGMN\_VG &      88.3 &           1.8 &             9.9 &     48.5 &         23.6 &           27.8 \\
     ResNet101--VGMN\_MN &      89.6 &           1.6 &             8.8 &     57.0 &         19.4 &           23.6 \\
    ResNet101--MN442\_MN &      90.1 &           1.3 &             8.6 &     58.2 &         19.5 &           22.3 \\
\bottomrule

\end{tabular}
\caption{Break-down of the results (in \%) according to the agreement level of the MN name: Categorization of a predicted name\ $\hat{n}$ into either a \textit{hit}, \textit{correct} (less preferred name, synonym, hypernym/hyponym), or \textit{wrong} \label{tab:exp_errors_agreement}}
\end{table*}


\begin{table*}[t]
	\centering
	\small
	\begin{tabular}{llr@{~}|r@{~}r@{~}r@{~}r@{~}r@{~}||r@{~}|r@{~}r@{~}r@{~}r@{~}r@{~}}
		\toprule
		& & \multicolumn{6}{c}{All Test Images ($\#$)} 
		& \multicolumn{6}{c}{VG$\neq$MN Images ($\#$)}\\
		\toprule
		& &  same &  syn. &  syn. &  hyper. &  hypo. &  hyper. &  same &  syn. &  syn. &  hyper. &  hypo. &  hyper. \\
		& 	&  cluster &  & cluster & & & cluster 
			& cluster  &  & cluster & & & cluster \\
		\midrule
		FRCNN--VG1600 & VG     %        &                  0.95 &              0.0 &                0.01 &              0.01 &             0.03 &                 0.0 &                  0.96 &              0.0 &                 0.0 &              0.01 &             0.03 &                 0.0 \\
		 &                  94.6 &              0.0 &                 0.7 &               3.4 &              1.3 &                  0.0 &                  95.5 &              0.0 &                 0.0 &               3.0 &              1.5 &                  0.0 \\
		FRCNN--MN442 & VG %&                   0.97 &             0.01 &                0.02 &               0.0 &              0.0 &                 0.0 &                  0.97 &              0.0 &                0.03 &               0.0 &              0.0 &                 0.0 \\
		 &                  93.3 &              1.3 &                 2.0 &               1.3 &              2.0 &                  0.0 &                  93.8 &              0.0 &                 3.1 &               1.6 &              1.6 &                  0.0 \\
		\midrule \midrule
		FRCNN--VG1600--VGMN & MN %&                   1.0 &              0.0 &                 0.0 &               0.0 &              0.0 &                 0.0 &                   1.0 &              0.0 &                 0.0 &               0.0 &              0.0 &                 0.0 \\
		 &                  94.9 &              0.0 &                 0.0 &               2.0 &              3.0 &                  0.0 &                  96.2 &              0.0 &                 0.0 &               1.9 &              1.9 &                  0.0 \\
		\midrule
		ResNet101--VGMN & MN %&                  0.97 &              0.0 &                0.03 &               0.0 &              0.0 &                 0.0 &                  0.98 &              0.0 &                0.02 &               0.0 &              0.0 &                 0.0 \\
		 &                  83.9 &              0.0 &                 2.7 &               6.2 &              5.4 &                  1.8 &                  92.6 &              0.0 &                 1.9 &               1.9 &              3.7 &                  0.0 \\
		ResNet101--MN442 & MN %& 0.97 &              0.0 &                0.03 &               0.0 &              0.0 &                 0.0 &                  0.98 &              0.0 &                0.02 &               0.0 &              0.0 &                 0.0 \\
		  &                  86.4 &              0.0 &                 2.7 &               5.5 &              3.6 &                  1.8 &                  92.3 &              0.0 &                 1.9 &               1.9 &              3.8 &                  0.0 \\
		ResNet101--VGMN & VG %&                   0.98 &              0.0 &                0.02 &               0.0 &              0.0 &                 0.0 &                  0.98 &              0.0 &                0.02 &               0.0 &              0.0 &                 0.0 \\
		 &                  87.4 &              0.0 &                 2.2 &               1.5 &              8.1 &                  0.7 &                  92.5 &              0.0 &                 1.5 &               1.5 &              4.5 &                  0.0 \\
		ResNet101--MN442 & VG% &                  0.98 &              0.0 &                0.02 &               0.0 &              0.0 &                 0.0 &                  0.97 &              0.0 &                0.03 &               0.0 &              0.0 &                 0.0 \\		
		&                  88.9 &              0.0 &                 2.2 &               2.2 &              5.9 &                  0.7 &                  92.5 &              0.0 &                 3.0 &               1.5 &              3.0 &                  0.0 \\
		\bottomrule
	\end{tabular}
	
	\caption{Break-down of the results for the \textit{correct} name predictions. Proportions (in \%) of the \textit{correct} categories to all correctly classified instances.  \textit{hyponym}: $\hat{n}$ is a hyponym of the entry-level name. \textit{hypernym\_cl}: $\hat{n}$ is a hypernym of any of the valid names (cluster). \textit{Synonym} and \textit{synonym\_cl} are analogous. \label{tab:exp_details_correct}}
\end{table*}

\begin{table*}[t]
	\centering
	\small
	\begin{tabular}{ll|r@{~}|r@{~}r@{~}r@{~}r@{~}|r@{~}r@{~}||r@{~}|r@{~}r@{~}r@{~}r@{~}|r@{~}r@{~}}
		\toprule
		&& \multicolumn{7}{c}{All Test Images ($\#$)} 
		& \multicolumn{7}{c}{VG$\neq$MN Images ($\#$)}\\
		\toprule
		Model--Vocab & GTtrain  
		&  co- &  \multicolumn{4}{c}{other object}  &  error &  low 
		&  co- &  \multicolumn{4}{c}{other object}  &  error &  low \\
		& & hypo. & (vis. &  ling. &  box &  other)   & & count 
		&  hypo. & (vis. &  ling. &  box &  other) &   & count     \\
		 
	\midrule
	FRCNN--VG1600 & VG     &                 13.2 &             1.7 &                 0.0 &                  17.4 &            6.6 &           39.7 &             21.5 &                  5.7 &             5.7 &                 0.0 &                  17.1 &           14.3 &           42.9 &             14.3 \\
	FRCNN--MN442 & VG       &                 15.5 &             0.6 &                 0.0 &                  15.5 &            6.2 &           49.1 &             13.0 &                  5.9 &             2.0 &                 0.0 &                  13.7 &           11.8 &           54.9 &             11.8 \\
	\midrule \midrule
	FRCNN--VG1600--VGMN & MN &                 30.6 &             0.9 &                 0.0 &                  13.0 &            5.6 &           32.4 &             17.6 &                 16.7 &             2.8 &                 0.0 &                  16.7 &           11.1 &           41.7 &             11.1 \\
	\midrule
	ResNet101--VGMN & MN	&                 34.1 &             0.0 &                 0.0 &                  13.0 &            1.3 &           39.5 &             12.1 &                 19.6 &             0.0 &                 0.0 &                  12.5 &            5.4 &           51.8 &             10.7 \\
	ResNet101--MN442 & MN  &                 33.0 &             0.0 &                 0.0 &                  14.0 &            1.9 &           37.7 &             13.5 &                 21.1 &             0.0 &                 0.0 &                  15.8 &            7.0 &           43.9 &             12.3 \\
	ResNet101--VGMN & VG  &                 37.6 &             0.0 &                 0.0 &                   6.1 &            2.3 &           41.1 &             12.9 &                 24.4 &             0.0 &                 0.0 &                   3.5 &            5.8 &           45.3 &             20.9 \\
	ResNet101--MN442 & VG &                 34.0 &             0.0 &                 0.0 &                   7.5 &            2.8 &           41.9 &             13.8 &                 22.6 &             0.0 &                 0.0 &                   6.0 &            8.3 &           39.3 &             23.8 \\
	\bottomrule
\end{tabular}
\caption{Break-down of the results for the \textit{wrong} name predictions. Proportions (in \%) of the corresponding categories to all wrongly classified instances.  \label{tab:exp_details_wrong}}
\end{table*}

\fi


\subsection{[TBC] Generalization Ability: OpenImages}
\label{sect:exp_openimages}
Question: Can models trained towards \mn generalize to other, related datasets? Here: OpenImages. \cs{[Increase coverage with zero-shot learning?]}\

Models compared: FRCNN--VG1600 vs. FRCNN--MN442 vs. ResNet101--XX (best Vanilla).


\section{Conclusions}
\label{sec:conclusions}

%\bibliography{BLA}
%\bibliographystyle{acl_natbib}

%\appendix
%
\section{Appendices}
\label{sec:appendix}
Appendices are material that can be read, and include lemmas, formulas, proofs, and tables that are not critical to the reading and understanding of the paper.
Appendices should be \textbf{uploaded as supplementary material} when submitting the paper for review.
Upon acceptance, the appendices come after the references, as shown here.

\paragraph{\LaTeX-specific details:}
Use {\small\verb|\appendix|} before any appendix section to switch the section numbering over to letters.


\section{Supplemental Material}
\label{sec:supplemental}
Supplementary material may report preprocessing decisions, model parameters, and other details necessary for the replication of the experiments reported in the paper.
Seemingly small preprocessing decisions can sometimes make a large difference in performance, so it is crucial to record such decisions to precisely characterize state-of-the-art methods.


\begin{figure*}[t]
  \centering
  \includegraphics[width=\textwidth]{images/verification-interface.pdf}
  \caption{A screenshot of our verification task interface. Up to ten names could be shown for an image in this way (the number of available colors in the second task would increase accordingly).}
  \label{fig:verification-interface}
\end{figure*}

\bibliographystyle{acl_natbib}
\bibliography{naming}
\end{document}

%%% Local Variables:
%%% mode: latex
%%% TeX-master: "acl2020_main"
%%% End:
