

\cs{@gbt @mw}
\subsection{Overview of ManyNames}
\label{sect:mn_overview}

\subsection{Verification of Annotations}
\label{sect:mn_verification}

\subsection{Analysis}
\label{sect:mn_analysis}
\paragraph{Of ManyNames:} 
\cs{+ @sz?}\\
\cs{Here goes what has not been discussed in LREC paper, because it focusses on entry-level names. }
\cs{Results aimed to show that we need \underline{multiple} annotations per object \underline{instance} to get the entry-level name}
\begin{itemize}
	\item Statistics wrt entry-level of object class != entry-level of its instances to (i.e.,\ entry-levels cannot be derived on class-level)
	\item Plot of how most frequent name changes in relation to the number of responses collected (e.g., phase 0 vs. 1 vs. 2 vs. 3). (i.e.,\ entry-level names cannot be derived from a few annotators)
	\item Coverage of set of topN MN names (MN442) wrt all VG objects \cs{@\sz}
	\item ...
\end{itemize}


\paragraph{Of Verification Annotations}
\cs{@gbt @mw}
\begin{itemize}
	\item ...
	\item ...
	\item For the instances where VG!=topMN: Percentage of instances where the VG name is among the responses of an \textit{alternative object} (as given by clustering).
\end{itemize}

